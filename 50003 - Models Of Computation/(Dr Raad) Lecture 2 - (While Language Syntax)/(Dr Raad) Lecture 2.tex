\documentclass{report}
    \title{50003 - Models of Computation - (Dr Raad) Lecture 2}
    \author{Oliver Killane}
    \date{15/10/21}

%===========================COMMON FORMAT & COMMANDS===========================
% This file contains commands and format to be used by every module, and is 
% included in all files.
%===============================================================================

%====================================IMPORTS====================================
\usepackage[a4paper, total={6in, 8in}]{geometry}
\usepackage{graphicx, amssymb, amsfonts, amsmath, xcolor, listings, tcolorbox, multirow, hyperref}
%===============================================================================

%====================================IMAGES=====================================
\graphicspath{{image/}}

% \centerimage{options}{image}
\newcommand{\centerimage}[2]{\begin{center}
    \includegraphics[#1]{#2}
\end{center}}
%===============================================================================

%==============================SYNTAX HIGHLIGHTING==============================
\newcommand{\fun}[1]{\textcolor{blue}{\textbf{#1}}}
\newcommand{\file}[1]{\textcolor{green}{\textbf{#1}}}
\newcommand{\struct}[1]{\textcolor{orange}{\textbf{#1}}}
\newcommand{\var}[1]{\textcolor{purple}{\textbf{#1}}}
\newcommand{\const}[1]{\textcolor{red}{\textbf{#1}}}
%===============================================================================

%=================================CODE LISTINGS=================================
\definecolor{codebackdrop}{gray}{0.9}
\definecolor{commentgreen}{rgb}{0,0.6,0}
\lstset{
    inputpath=code, 
    commentstyle=\color{commentgreen},
    keywordstyle=\color{blue}, 
    backgroundcolor=\color{codebackdrop}, 
    basicstyle=\footnotesize,
    frame=single,
    numbers=left,
    stepnumber=1,
    showstringspaces=false,
    breaklines=true,
    postbreak=\mbox{\textcolor{red}{$\hookrightarrow$}\space}
}

% Create a code listing for a single line
% \codeline{language}{line}{file}
\newcommand{\codeline}[3]{\lstinputlisting[language=#1, firstline = #2, lastline = #2]{#3}}

% Create a code listing for multiple lines
% \codeline{language}{start}{end}{file}
\newcommand{\codelines}[4]{\lstinputlisting[language=#1, firstline = #2, lastline = #3]{#4}}


% Create a code listing for a given language & file
% \codelist{language}{file}
\newcommand{\codelist}[2]{\lstinputlisting[language=#1]{#2}}
%===============================================================================

%================================TEXT STRUCTURES================================
% Marka a word as bold
% \keyword{important word}
\newcommand{\keyword}[1]{\textbf{#1}}

% Creates a section in italics
% \question{question in italics}
\newcommand{\question}[1]{\textit{#1} \\ }

% Creates a box with title for side notes.
% \sidenote{title}{contents}
\newcommand{\sidenote}[2]{\begin{tcolorbox}[title=#1]#2\end{tcolorbox}}

\newcommand{\termdef}[2]{\begin{tcolorbox}[title=Definition: #1, colframe = blue]#2\end{tcolorbox}}

% Creates an item in an itemize or enumerate, with a paragraph after
% \begin{itemize}
%     \bullpara{title}{contents}
% \end{itemize}
\newcommand{\bullpara}[2]{\item \textbf{#1} \ #2}

% Creates a compact list (very small gaps between items)
% \compitem{
%     \item item 1
%     \item item 2
%     \item ...
% }
\newcommand{\compitem}[1]{\begin{itemize}\setlength\itemsep{-0.5em}#1\end{itemize}}
\newcommand{\compenum}[1]{\begin{enumerate}\setlength\itemsep{-0.5em}#1\end{enumerate}}

% Creates a link to the lecture for use at the start of the notes document
\newcommand{\lectlink}[1]{\sidenote{Lecture Recording}{
    Lecture recording is available \href{#1}{here}
}}
%===============================================================================

%================================STYLIZED PROOFS================================
%For ease in writing stylized proofs with numbers
\newcommand{\stepno}[1]{\textcolor{red}{\textbf{#1}}}

\newenvironment{proof}
{\begin{center}\begin{tabular}{r l l }}
{\end{tabular}\end{center}}

%\proofstep{step}{workings}{description}
\newcommand{\proofstep}[3]{\stepno{(#1)} & #2 & #3 \\}
%===============================================================================

%==============================UNFINISHED SECTION===============================
\newcommand{\unfinished}{\begin{huge} \textcolor{red}{\textbf{UNFINISHED!!!}} \end{huge}}
%===============================================================================

%================================CONFIGURATIONS================================
\newcommand{\config}[2]{\langle #1, #2 \rangle}
%==============================================================================

%==============================BIG STEP SEMANTICS==============================
\newcommand{\bigstep}[4]{\text{(#1)}\dfrac{#2}{#3 \Downarrow #4}}
\newcommand{\bigstepdef}[5]{\bigstep{#1}{#2}{#3}{#4} \ #5}
%==============================================================================

%=============================SMALL STEP SEMANTICS=============================
\newcommand{\smallstep}[4]{\text{(#1)}\dfrac{#2}{#3 \to #4}}
\newcommand{\smallstepdef}[5]{\smallstep{#1}{#2}{#3}{#4} \ #5}
%==============================================================================

\newcommand{\whilest}[3]{\text{(#1)}\dfrac{#2}{#3}}
\newcommand{\whilestdef}[4]{\whilest{#1}{#2}{#3} \ #4}
%==============================================================================

%===================================COMMANDS===================================
\newcommand{\while}[2]{\text{while } #1 \text{ do } #2}
\newcommand{\cond}[3]{\text{if } #1 \text{ then } #2 \text{ else } #3}
\newcommand{\doret}[2]{\text{do } #1 \text{ return } #2}
%==============================================================================

\begin{document}
\maketitle
\lectlink{https://imperial.cloud.panopto.eu/Panopto/Pages/Viewer.aspx?id=a3951f13-3a34-44bc-ba10-adc300d0077f}

\section*{Syntax of a while Language}
We can define a simple while language (if, else, while loops) to build programs from \& to analyse.
\begin{center}
	\begin{tabular}{r c l}
		$B \in Bool$ & ::= & $true | false | E = E | E < E | B \& B | \neg B \dots$                   \\
		$E \in Exp$  & ::= & $x | n | E + E | E \times E | \dots$                                     \\
		$C \in Com$  & ::= & $x :=E | if \ B \ then \ C \ else \ C | C;C | skip | while \ B \ do \ C$ \\
	\end{tabular}
\end{center}
Where $x \in Var$ ranges over variable identifiers, and $n \in \mathbb{N}$ ranges over natural numbers.
\\
\\ We can also define simple expressions (\keyword{SimpleExp}) to work on:
\[E \in SimpleExp \ ::= \ n | E + E | E \times E | \dots\]
\subsubsection*{Operational Semantics for SimpleExp}
\begin{itemize}
	\bullpara{Small-Step}{
		Also called structural, gives a method for evaluating an expression step-by-step.
	}
	\bullpara{Big-Step}{
		Also called Natural, ignores intermediate steps and gives result immediately.
	}
\end{itemize}
\subsubsection*{Big Step Semantics of SimpleExp}
The properties OF $\Downarrow$ are:
\begin{itemize}
	\bullpara{Determinacy}{
		For all $E, n_1$ and $n_2$ if $E \Downarrow n_1$ and $E \Downarrow n_2$ then $n_1 = n_2$
	}
	\bullpara{Totality}{
		For all $E$ there exists an $n$ such that $E \Downarrow n$.
		\\
		\\ We can break this with loops in matching, e.g
		\[\bigstepdef{B-NON-TOTAL}{}{true}{true}{}\]
		As a result, on hitting true will not stop.
	}
\end{itemize}
\subsubsection*{Small Step Semantics of SimpleExp}
Given a realtion $\to$ we can define a new relation $\leftarrow *$ such that:
\[E \leftarrow * E' \text{ holds if and only if } E = E' \text{ or there is some finite sequence } E \to E_1 \to E_3 \to \dots \to E_k \to E' \]
\begin{itemize}
	\bullpara{Normal Form}{
		$E$ is in its normal form (irreducable) if there is no $E'$ such that $E \to E'$
		\\
		\\ In \keyword{SimpleExp} the normal form is the natural numbers.
	}
	\bullpara{Determinacy}{
		For all $E, E_1, E_2$ if $E \to E_1$ and $E \to E_2$ then $E_1 = E_2$.
		\\
		\\ There is at most one next step.
	}
	\bullpara{Confluence}{
		For all $E, E_1, E_2$ if $E \to * E_1$ and $E \to * E_2$ then there exists some $E'$ such that $E_1 \to * E'$ and $E_2 \to* E'$.
		\\
		\\ Determinate $\rightarrow$ Confluent.
		\\
		\\ There are several evaluations paths, but they all get the same end result.
	}
	\bullpara{(Strong) Normalisation}{
		There are no infinite sequences of expressions $E_1 \to E_2 \to E_3 \to \dots$ such that for all $i$, $E_i \to E_{i+1}$.
		\\
		\\ Every evaluation path eventually reaches a normal form.
	}
\end{itemize}
Theorem: for all $E, n_1, n_2$, if $E \to * n_1$ and $E \to * n_2$ then $n_1 = n_2$.
\end{document}
