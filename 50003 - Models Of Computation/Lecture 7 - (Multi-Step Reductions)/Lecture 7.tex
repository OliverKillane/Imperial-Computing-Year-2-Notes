\documentclass{report}
    \title{50003 - Models of Computation - Lecture 7}
    \author{Oliver Killane}
    \date{02/11/21}

%===========================COMMON FORMAT & COMMANDS===========================
% This file contains commands and format to be used by every module, and is 
% included in all files.
%===============================================================================

%====================================IMPORTS====================================
\usepackage[a4paper, total={6in, 8in}]{geometry}
\usepackage{graphicx, amssymb, amsfonts, amsmath, xcolor, listings, tcolorbox, multirow, hyperref}
%===============================================================================

%====================================IMAGES=====================================
\graphicspath{{image/}}

% \centerimage{options}{image}
\newcommand{\centerimage}[2]{\begin{center}
    \includegraphics[#1]{#2}
\end{center}}
%===============================================================================

%=================================CODE LISTINGS=================================
\definecolor{codebackdrop}{gray}{0.9}
\definecolor{commentgreen}{rgb}{0,0.6,0}
\lstset{
    inputpath=code, 
    commentstyle=\color{commentgreen},
    keywordstyle=\color{blue}, 
    backgroundcolor=\color{codebackdrop}, 
    basicstyle=\footnotesize,
    frame=single,
    numbers=left,
    stepnumber=1,
    showstringspaces=false,
    breaklines=true,
    postbreak=\mbox{\textcolor{red}{$\hookrightarrow$}\space}
}

% Create a code listing for a single line
% \codeline{language}{line}{file}
\newcommand{\codeline}[3]{\lstinputlisting[language=#1, firstline = #2, lastline = #2]{#3}}

% Create a code listing for a given language & file
% \codelist{language}{file}
\newcommand{\codelist}[2]{\lstinputlisting[language=#1]{#2}}
%===============================================================================

%================================TEXT STRUCTURES================================
% Marka a word as bold
% \keyword{important word}
\newcommand{\keyword}[1]{\textbf{#1}}

% Creates a section in italics
% \question{question in italics}
\newcommand{\question}[1]{\textit{#1} \\ }

% Creates a box with title for side notes.
% \sidenote{title}{contents}
\newcommand{\sidenote}[2]{\begin{tcolorbox}[title=#1]#2\end{tcolorbox}}

% Creates an item in an itemize or enumerate, with a paragraph after
% \begin{itemize}
%     \bullpara{title}{contents}
% \end{itemize}
\newcommand{\bullpara}[2]{\item \textbf{#1} \ #2}

% Creates a compact list (very small gaps between items)
% \compitem{
%     \item item 1
%     \item item 2
%     \item ...
% }
\newcommand{\compitem}[1]{\begin{itemize}\setlength\itemsep{-0.5em}#1\end{itemize}}

% Creates a link to the lecture for use at the start of the notes document
\newcommand{\lectlink}[1]{\sidenote{Lecture Recording}{
    Lecture recording is available \href{#1}{here}
}}
%===============================================================================


%================================CONFIGURATIONS================================
\newcommand{\config}[2]{\langle #1, #2 \rangle}
%==============================================================================

%==============================BIG STEP SEMANTICS==============================
\newcommand{\bigstep}[4]{\text{(#1)}\dfrac{#2}{#3 \Downarrow #4}}
\newcommand{\bigstepdef}[5]{\bigstep{#1}{#2}{#3}{#4} \ #5}
%==============================================================================

%=============================SMALL STEP SEMANTICS=============================
\newcommand{\smallstep}[4]{\text{(#1)}\dfrac{#2}{#3 \to #4}}
\newcommand{\smallstepdef}[5]{\smallstep{#1}{#2}{#3}{#4} \ #5}
%==============================================================================

\newcommand{\whilest}[3]{\text{(#1)}\dfrac{#2}{#3}}
\newcommand{\whilestdef}[4]{\whilest{#1}{#2}{#3} \ #4}
%==============================================================================

%===================================COMMANDS===================================
\newcommand{\while}[2]{\text{while } #1 \text{ do } #2}
\newcommand{\cond}[3]{\text{if } #1 \text{ then } #2 \text{ else } #3}
\newcommand{\doret}[2]{\text{do } #1 \text{ return } #2}

\newcommand{\instrlabel}[1]{\text{\textcolor{teal}{$L_{#1}$}}}
\newcommand{\reglabel}[1]{\text{\textcolor{orange}{$R_{#1}$}}}
\newcommand{\regtemp}[1]{\text{\textcolor{orange}{$#1$}}}
\newcommand{\instr}[2]{\instrlabel{#1} : & #2 \\}
\newcommand{\dec}[3]{\reglabel{#1}^- \to \instrlabel{#2}, \instrlabel{#3}}
\newcommand{\inc}[2]{\reglabel{#1}^+ \to \instrlabel{#2}}
\newcommand{\halt}{\text{\textcolor{red}{\textbf{HALT}}}}

\newcommand{\lambapp}[2]{#1 \ #2}
\newcommand{\lambappb}[2]{(#1) \ (#2)}
\newcommand{\lambfun}[2]{\lambda #1 \ . \ #2}

\makeatletter
\newcommand{\lambarg}[1]{%
	#1 \checknextarglambarg}
\newcommand{\checknextarglambarg}{\@ifnextchar\bgroup{\gobblenextarglambarg}{}}
\newcommand{\gobblenextarglambarg}[1]{ \ #1 \@ifnextchar\bgroup{\gobblenextarglambarg}{}}

\newcommand{\chnum}[1]{\underline{#1}}
% Variable argument regconfig \regconfig{label no}{reg val}{reg val}...
\makeatletter
\newcommand{\regconfig}[2]{%
	$#1$ & $#2$\checknextarg}
\newcommand{\checknextarg}{\@ifnextchar\bgroup{\gobblenextarg}{\\}}
\newcommand{\gobblenextarg}[1]{ & $#1$\@ifnextchar\bgroup{\gobblenextarg}{\\}}
%==============================================================================

\begin{document}
\maketitle
\lectlink{https://imperial.cloud.panopto.eu/Panopto/Pages/Viewer.aspx?id=3862d74f-cd34-4faf-ba7b-add300e0c3a7}

\subsection*{Note for reader}
We will reference to state by set $State \triangleq (Var \rightharpoonup \mathbb{N})$.

\subsection*{Lemmas}
\sidenote{Lemma}{
	A small proven proposition that can be used in a proof. Used to make the proof smaller.
	\\
	\\ Also know as an "auxiliary theorem" or "helper theorem".
}
\sidenote{Corollary}{
	A theorem connected by a short proof to another existing theorem.
	\\
	\\ If B is can be easily deduced from A (or is evident in A's proof) then B is a corollary of A.
}
\subsubsection*{Lemmas}
\begin{enumerate}
	\item $\forall r \in \mathbb{N}. \forall E_1,E_1',E_2 \in SimpleExp. [E_1 \to^r E_1' \Rightarrow (E_1 + E_2) \to^r (E_1' + E_2)]$
	\item $\forall r,n \in \mathbb{N}. \forall E_2,E_2' \in SimpleExp . [E_2 \to^r E_2' \Rightarrow (n + E_2) \to^r (n + E_2')]$
\end{enumerate}
\subsubsection*{Corollaries}
\begin{enumerate}
	\item $\forall n_1 \in \mathbb{N} . \forall E_1, E_2 \in SimpleExp . [E_1 \to^* n_1 \Rightarrow (E_1 + E_2) \to^* (n_1 + E_2)]$
	\item $\forall n_1, n_2 \in \mathbb{N} . \forall E_2 \in SimpleExp . [E_2 \to^* n_2 \Rightarrow (n_1 + E_2) \to^* (n_1 + n_2)]$
	\item $\forall n,n_1, n_2, \in \mathbb{N} . \forall E_1, E_2 \in SimpleExp . [E_1 \to^* n_1 \land E_2 \to^* n_2 \land n = n_1 + n_2 \Rightarrow (E_1 + E_2) \to^* n]$
\end{enumerate}

\subsection*{Connecting $\Downarrow$ and $\to^*$ for SimpleExp}
\[\forall E \in SimpleExp, n \in \mathbb{N}.[E \Downarrow n \Leftrightarrow E \to^* n]\]
We prove each direction of implication separately. First we prove by induction over $E$ using the property $P$:
\[P(E) =^{def} \forall n \in \mathbb{N}.[E \Downarrow n \Rightarrow E \to^* n]\]
\subsubsection*{Base Case}
Take arbitrary $m \in \mathbb{N}$ to show $P(m)$ = $m \Downarrow n \Rightarrow m \to^* n$.
\begin{center}
	\begin{tabular}{l l l}
		(1) & Assume $m \Downarrow n$ &                                  \\
		(2) & $m = n$                 & (From Inversion of $\Downarrow$) \\
		(3) & $m \to^* n$             & (By 2 and definition of $\to^*$) \\
	\end{tabular}
\end{center}
\subsubsection*{Inductive Step}
Take some arbitrary $E, E_1, E_2$ such that $E = E_1 + E_2$.
\\ Inductive Hypothesis
\[\forall n_1 \in \mathbb{N} . [E_1 \Downarrow n_1 \Rightarrow E_1 \to^* n_1]\]
\[\forall n_2 \in \mathbb{N} . [E_2 \Downarrow n_2 \Rightarrow E_2 \to^* n_2]\]
To show $P(E)$: $\forall n \in \mathbb{N} . [(E_1 + E_2) \Downarrow n \Rightarrow (E_1 + E_2) \to^* n]$.
\begin{center}
	\begin{tabular}{l l l}
		(1) & Assume $(E_1 + E_2) \Downarrow n$                                                 &                               \\
		(2) & $\exists n_1, n_2 \in \mathbb{N} . [E_1 \Downarrow n_1 \land E_2 \Downarrow n_2]$ & (By 1 \& definition of B-ADD) \\
		(3) & $E_1 \to^* n_1$                                                                   & (By 2 \& IH)                  \\
		(4) & $E_2 \to^* n_2$                                                                   & (By 2 \& IH)                  \\
		(5) & Chose some $n \in \mathbb{N}$ such that $n = n_1 + n_2$                           &                               \\
		(6) & $(E_1 + E_2) \to^* n$                                                             & (By 3,4,5 Corollary 3)        \\
		(7) & $E \to^* n$                                                                       & (By 6, definition of $E$)     \\
	\end{tabular}
\end{center}
Hence assuming $E \Downarrow n$ implies $E \to^* n$, so $P(E)$.
\\
Next we work the other way, to show:
\[\forall E \in SimpleExp . \forall n \in \mathbb{N}.[E \to^* n \Rightarrow E \Downarrow n ]\]
\begin{center}
	\begin{tabular}{l l l}
		(1) & Take arbitrary $E \in SimplExp$ such that $E \to^* n$   & (Initial setup)                            \\
		(2) & Take some $m \in \mathbb{N}$ such that $E \Downarrow m$ & (By totality of $\Downarrow$)              \\
		(3) & $n = m$                                                 & (By 1,2 \& uniqueness of result for $\to$) \\
		(4) & $E \Downarrow n$                                        & (By 3)                                     \\
	\end{tabular}
\end{center}
It is also possible to prove this without using normalisation and determinacy, by induction on $E$.
\subsection*{Multi-Step Reductions}
\textbf{Lemma:}
\[\forall r \in \mathbb{N}. \forall E_1, E'_1, E_2 . [E_1 \to^r E'_1 \Rightarrow (E_1 + E_2) \to^r (E'_1 + E_2)]\]
To prove $\forall r \in \mathbb{N} . [P(r)]$ by induction on $r$:
\subsubsection*{Base Case}
\begin{itemize}
	\item Base case is $r = 0$.
	\item Prove that $P(0)$ holds.
\end{itemize}
\subsubsection*{Inductive  Step}
\begin{itemize}
	\item Inductive Case is $r = k + 1$ for arbitrary $k \in \mathbb{N}$.
	\item Inductive hypothesis is $P(k)$.
	\item Prove $P(k + 1)$ using inductive hypothesis.
\end{itemize}
\subsubsection*{Proof of the Lemma}
By induction on $r$:
\textbf{Base Case:}
Take some arbitrary $E_1, E_1', E_2 \in SimpleExp$ such that $E_1 \to^0 E_1'$.
\begin{center}
	\begin{tabular}{r c l}
		(1) & $E_1 = E_1'$                     & (By definition of $\to^0$) \\
		(2) & $(E_1 + E_2) = (E_1' + E_2)$     & (By 1)                     \\
		(3) & $(E_1 + E_2) \to^0 (E_1' + E_2)$ & (By definition of $\to^0$) \\
	\end{tabular}
\end{center}
\textbf{Inductive Step:}
Take arbitrary $k \in \mathbb{N}$ such that $P(k)$
\begin{center}
	\begin{tabular}{r c l}
		(1) & Take arbitrary $E_1, E_1', E_2$ such that $E_1 \to E'_1$ & (Initial setup)                  \\
		(2) & Take arbitrary $E_1''$ such that $E_1'' \to E_1'$        &                                  \\
		(3) & $(E_1 + E_2) \to^k (E_1'' + E_2)$                        & (By 2 \& IH)                     \\
		(4) & $(E_1'' + E_2) \to (E_1' + E_2)$                         & (By 2 \& rule S-LEFT)            \\
		(5) & $(E_1 + E_2) \to^{k+1} (E_1' + E_2)$                     & (3,4, definition of $\to^{k+1}$) \\
	\end{tabular}
\end{center}
\subsection*{Determinacy of $\to$ for Exp}
We extend simple expressions configurations of the form $\config{E}{s}$.
\[E \in Exp ::= n | x | E + E | \dots\]

Determinacy:
\[\forall E,E_1,E_2 \in Exp . \forall s, s_1, s_2 \in State . [\config{E}{s} \to \config{E_1}{s_1} \land \config{E}{s} \to \config{E_2}{s_2} \Rightarrow \config{E_1}{s_1} = \config{E_2}{s_2}]\]
We prove this using property $P$:
\[P(E,s) \triangleq \forall E_1,E_2 \in Exp . \forall s_1, s_2 \in State . [\config{E}{s} \to \config{E_1}{s_1} \land \config{E}{s} \to \config{E_2}{s_2} \Rightarrow \config{E_1}{s_1} = \config{E_2}{s_2}]\]
\subsubsection*{Base Case: $E = x$}
Take arbitrary $n \in \mathbb{N}$ and $s \in State$ to show $P(n,s)$
\begin{center}
	\begin{tabular}{r c l}
		(1) & take $E_1 \in Exp$, $s_1 \in State$ such that $\config{n}{s} \to \config{E_1}{s_1}$ & (Initial setup)                            \\
		(2) & take $E_2 \in Exp$, $s_2 \in State$ such that $\config{n}{s} \to \config{E_2}{s_2}$ & (Initial setup)                            \\
		(3) & $n = E_1 \land s = s_1$                                                             & (By 1 \& inversion on definition of E.NUM) \\
		(4) & $n = E_2 \land s = s_2$                                                             & (By 2 \& inversion on definition of E.NUM) \\
		(5) & $E_1 = E_2 \land s_1 = s_2$                                                         & (By 3 \& 4)                                \\
		(6) & $\config{E_1}{s_1} = \config{E_2}{s_2}$                                             & (By 5 \& definition of configurations)     \\
	\end{tabular}
\end{center}
\subsubsection*{Base Case: $E = x$}
Take arbitrary $x \in Var$ and $s \in State$ to show $P(n,s)$
\begin{center}
	\begin{tabular}{r c l}
		(1) & take $E_1 \in \mathbb{N}$, $s_1 \in State$ such that $\config{x}{s} \to \config{E_1}{s_1}$ & (Initial setup)                            \\
		(2) & take $E_2 \in \mathbb{N}$, $s_2 \in State$ such that $\config{x}{s} \to \config{E_2}{s_2}$ & (Initial setup)                            \\
		(3) & $E_1 = s(x) \land s_1 = s$                                                                 & (By 1 \& inversion on definition of E.VAR) \\
		(3) & $E_2 = s(x) \land s_2 = s$                                                                 & (By 2 \& inversion on definition of E.VAR) \\
		(5) & $E_1 = E_2 \land s_1 = s_2$                                                                & (By 3 \& 4)                                \\
		(6) & $\config{E_1}{s_1} = \config{E_2}{s_2}$                                                    & (By 5 \& definition of configurations)     \\
	\end{tabular}
\end{center}

\dots Inductive Step \dots

\subsection*{Syntax of Commands}
\[C \in Com ::= x := E | \cond{B}{C}{C} | C;C | skip | \while{B}{C}\]
\begin{itemize}
	\bullpara{Determinacy}{
		\[\forall C, C_1, C_2 \in Com . \forall s, s_1, s_2 \in State . [\config{C}{s} \to_c \config{C_1}{s_1} \land \config{C}{s} \to_c \config{C_2}{s_2} \Rightarrow \config{C_1}{s_1} = \config{C_2}{s_2}]\]
	}
	\bullpara{Confluence}{
		\[\forall C, C_1, C_2 \in Com . \forall s, s_1, s_2 \in State . [\config{C}{s} \to_c^* \config{C_1}{s_1} \land \config{C}{s} \to_c^* \config{C_2}{s_2} \Rightarrow \exists C' \in Com . \exists s' \in State . [\config{C_1}{s_1} \to_c^* \config{C'}{s'} \land \config{C_2}{s_2} \to_c^* \config{C'}{s'}]]\]
	}
	\bullpara{Unique Answer}{
		\[\forall C \in Com . s_1 s_2 \in State . [\config{C}{s} \to_c^* \config{skip}{s_1} \land \config{C}{s} \to_c^* \config{skip}{s_2} \Rightarrow s_1 = s_2]\]
	}
	\bullpara{No Normalisation}{
		\\ There exist derivations of infinite length for while.
	}
\end{itemize}
\subsection*{Connecting $\Downarrow$ and $\to^*$ for While}

\begin{enumerate}
	\item $\forall E, n \in Exp. \forall s, s' \in State . [\config{E}{s} \Downarrow_e \config{n}{s'} \Leftrightarrow \config{E}{s} \to_e^* \config{n}{s'}]$
	\item $\forall B, b \in Bool. \forall s, s' \in State . [\config{B}{s} \Downarrow_b \config{b}{s'} \Leftrightarrow \config{B}{s} \to_b^* \config{b}{s'}]$
	\item $\forall C \in Com. \forall s, s' \in State . [\config{C}{s} \Downarrow_c \langle s' \rangle \Leftrightarrow \config{C}{s} \to_c^* \config{skip}{s'}]$
\end{enumerate}
For $Exp$ and $Bool$ we have proofs by induction on the structure of expressions/booleans.
\\
\\ For $\Downarrow_c$ it is more complex as the $\Downarrow_c \Leftarrow \to_c^*$ cannot be proven using totality. Instead \keyword{complete/strong induction} on length of $\to_c^*$ is used.

\end{document}
