\documentclass{report}
    \title{50003 - Models of Computation - Lecture 5}
    \author{Oliver Killane}
    \date{25/10/21}

%===========================COMMON FORMAT & COMMANDS===========================
% This file contains commands and format to be used by every module, and is 
% included in all files.
%===============================================================================

%====================================IMPORTS====================================
\usepackage[a4paper, total={6in, 8in}]{geometry}
\usepackage{graphicx, amssymb, amsfonts, amsmath, xcolor, listings, tcolorbox, multirow, hyperref}
%===============================================================================

%====================================IMAGES=====================================
\graphicspath{{image/}}

% \centerimage{options}{image}
\newcommand{\centerimage}[2]{\begin{center}
    \includegraphics[#1]{#2}
\end{center}}
%===============================================================================

%==============================SYNTAX HIGHLIGHTING==============================
\newcommand{\fun}[1]{\textcolor{blue}{\textbf{#1}}}
\newcommand{\file}[1]{\textcolor{green}{\textbf{#1}}}
\newcommand{\struct}[1]{\textcolor{orange}{\textbf{#1}}}
\newcommand{\var}[1]{\textcolor{purple}{\textbf{#1}}}
\newcommand{\const}[1]{\textcolor{red}{\textbf{#1}}}
%===============================================================================

%=================================CODE LISTINGS=================================
\definecolor{codebackdrop}{gray}{0.9}
\definecolor{commentgreen}{rgb}{0,0.6,0}
\lstset{
    inputpath=code, 
    commentstyle=\color{commentgreen},
    keywordstyle=\color{blue}, 
    backgroundcolor=\color{codebackdrop}, 
    basicstyle=\footnotesize,
    frame=single,
    numbers=left,
    stepnumber=1,
    showstringspaces=false,
    breaklines=true,
    postbreak=\mbox{\textcolor{red}{$\hookrightarrow$}\space}
}

% Create a code listing for a single line
% \codeline{language}{line}{file}
\newcommand{\codeline}[3]{\lstinputlisting[language=#1, firstline = #2, lastline = #2]{#3}}

% Create a code listing for multiple lines
% \codeline{language}{start}{end}{file}
\newcommand{\codelines}[4]{\lstinputlisting[language=#1, firstline = #2, lastline = #3]{#4}}


% Create a code listing for a given language & file
% \codelist{language}{file}
\newcommand{\codelist}[2]{\lstinputlisting[language=#1]{#2}}
%===============================================================================

%================================TEXT STRUCTURES================================
% Marka a word as bold
% \keyword{important word}
\newcommand{\keyword}[1]{\textbf{#1}}

% Creates a section in italics
% \question{question in italics}
\newcommand{\question}[1]{\textit{#1} \\ }

% Creates a box with title for side notes.
% \sidenote{title}{contents}
\newcommand{\sidenote}[2]{\begin{tcolorbox}[title=#1]#2\end{tcolorbox}}

\newcommand{\termdef}[2]{\begin{tcolorbox}[title=Definition: #1, colframe = blue]#2\end{tcolorbox}}

% Creates an item in an itemize or enumerate, with a paragraph after
% \begin{itemize}
%     \bullpara{title}{contents}
% \end{itemize}
\newcommand{\bullpara}[2]{\item \textbf{#1} \ #2}

% Creates a compact list (very small gaps between items)
% \compitem{
%     \item item 1
%     \item item 2
%     \item ...
% }
\newcommand{\compitem}[1]{\begin{itemize}\setlength\itemsep{-0.5em}#1\end{itemize}}
\newcommand{\compenum}[1]{\begin{enumerate}\setlength\itemsep{-0.5em}#1\end{enumerate}}

% Creates a link to the lecture for use at the start of the notes document
\newcommand{\lectlink}[1]{\sidenote{Lecture Recording}{
    Lecture recording is available \href{#1}{here}
}}
%===============================================================================

%================================STYLIZED PROOFS================================
%For ease in writing stylized proofs with numbers
\newcommand{\stepno}[1]{\textcolor{red}{\textbf{#1}}}

\newenvironment{proof}
{\begin{center}\begin{tabular}{r l l }}
{\end{tabular}\end{center}}

%\proofstep{step}{workings}{description}
\newcommand{\proofstep}[3]{\stepno{(#1)} & #2 & #3 \\}
%===============================================================================

%==============================UNFINISHED SECTION===============================
\newcommand{\unfinished}{\begin{huge} \textcolor{red}{\textbf{UNFINISHED!!!}} \end{huge}}
%===============================================================================

%================================CONFIGURATIONS================================
\newcommand{\config}[2]{\langle #1, #2 \rangle}
%==============================================================================

%==============================BIG STEP SEMANTICS==============================
\newcommand{\bigstep}[4]{\text{(#1)}\dfrac{#2}{#3 \Downarrow #4}}
\newcommand{\bigstepdef}[5]{\bigstep{#1}{#2}{#3}{#4} \ #5}
%==============================================================================

%=============================SMALL STEP SEMANTICS=============================
\newcommand{\smallstep}[4]{\text{(#1)}\dfrac{#2}{#3 \to #4}}
\newcommand{\smallstepdef}[5]{\smallstep{#1}{#2}{#3}{#4} \ #5}
%==============================================================================

\newcommand{\whilest}[3]{\text{(#1)}\dfrac{#2}{#3}}
\newcommand{\whilestdef}[4]{\whilest{#1}{#2}{#3} \ #4}
%==============================================================================

%===================================COMMANDS===================================
\newcommand{\while}[2]{\text{while } #1 \text{ do } #2}
\newcommand{\cond}[3]{\text{if } #1 \text{ then } #2 \text{ else } #3}
\newcommand{\doret}[2]{\text{do } #1 \text{ return } #2}
%==============================================================================

\begin{document}
    \maketitle
    \lectlink{https://imperial.cloud.panopto.eu/Panopto/Pages/Viewer.aspx?id=14a819ba-3c2a-4eb6-aec0-adcc00b79830}

    \section*{Structural Induction}
        Structural induction is used for reasoning about collections of objects, which are:
        \compitem{
            \item structured in a well defined way
            \item finite but can be arbitrarily large and complex
        }
        We can use this is reason about:
        \compitem{
            \item natural numbers
            \item data structures (lists, trees, etc)
            \item programs (can be large, but are finite)
            \item derivations of assertions like $E \Downarrow 4$ (finite trees of axioms and rules)
        }
    \section*{Structural Induction over Natural Numbers}
        \[\mathbb{N} \in Nat ::= zero| succ(\mathbb{N})\]
        To prove a property $P(\mathbb{N})$ holds, for every number $N \in Nat$ by induction on structure $\mathbb{N}$:
        \begin{itemize}
            \bullpara{Base Case}{Prove $P(zero)$}
            \bullpara{Inductive Case}{Inductive Case is $P(Succ(K))$ where $P(K)$ holds}
        \end{itemize}
        For example, we can prove the property:
        \[plus(\mathbb{N}, zero) = \mathbb{N}\]
        \begin{itemize}
            \bullpara{Base Case}{
                \\ Show $plus(zero, zero) = zero$
                \begin{center}
                    \begin{tabular}{c r c l c}
                        (1) & LHS & = & $plus(zero, zero)$ & \\
                        (2) & & = & $zero$ & (By definition of $plus$) \\
                        (3) & & = & RHS & (As Required) \\
                    \end{tabular}
                \end{center}}
            \bullpara{Inductive Case}{
                \\ $N = succ(K)$
                \\ Inductive Hypothesis $plus(K, zero) = K$
                \\ Show $plus(succ(K), zero) = succ(K)$
                \begin{center}
                    \begin{tabular}{c r c l c}
                        (1) & LHS & = & $plus(succ(K), zero)$ \\
                        (2) & & = & $succ(plus(K, zero))$ & (By definition of $plus$) \\
                        (3) & & = & $succ(K)$ & (By Inductive Hypothesis) \\
                        (4) & & = & RHS & (As Required) \\
                    \end{tabular}
                \end{center}
            }
        \end{itemize}
        Mathematics induction is a special case of structural induction:
        \[P(0) \land [\forall k \in \mathbb{N}. P(k) \Rightarrow P(k + 1)]\]
        In the exam you may use $P(0)$ and $P(K+1)$ rather than $P(zero)$ and $P(succ(k))$ to save time.
    \section*{Binary Tree Example}
        \[bTree \in BinaryTree ::= Node | Branch(bTree, bTree)\]
        We can define a function $leaves$:
        \[leaves(Node) = 1\]
        \[leaves(Branch(T_1, T_2)) = 1 + leaves(T_1) + leaves(T_2)\]
        Or $branches$:
        \[branches(Node) = 0\]
        \[branches(Branch(T_1,T_2)) = branches(T_1) + branches(T_2)\]
        \sidenote{Exercise}{
            Prove By induction that $leaves(T) = branches(T) + 1$
        }
    \section*{Induction over SimpleExp}
        \[E \in SimpleExp ::= n | E + E | E \times E | \dots\]
        where $n \in N$.
        \subsubsection*{Properties of $\Downarrow$}
            \begin{itemize}
                \bullpara{Determinacy}{
                    \\ A simple expression can only evaluate to one answer.
                    \[E \Downarrow n_1 \land E \Downarrow n_2 \rightarrow n_1 = n_2\]
                }
                \bullpara{Totality}{
                    \\ A simple expression evaluates to at least one answer.
                    \[\forall E \in SimpleExp .\exists n \in \mathbb{N} .[E \Downarrow n]\]
                }
            \end{itemize}






        

\end{document}
