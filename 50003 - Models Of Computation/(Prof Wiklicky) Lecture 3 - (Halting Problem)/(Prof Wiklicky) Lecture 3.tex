\documentclass{report}
    \title{50003 - Models Of Computation - (Prof Wiklicky) Lecture 3}
    \author{Oliver Killane}
    \date{31/03/22}
%===========================COMMON FORMAT & COMMANDS===========================
% This file contains commands and format to be used by every module, and is 
% included in all files.
%===============================================================================

%====================================IMPORTS====================================
\usepackage[a4paper, total={6in, 8in}]{geometry}
\usepackage{graphicx, amssymb, amsfonts, amsmath, xcolor, listings, tcolorbox, multirow, hyperref}
%===============================================================================

%====================================IMAGES=====================================
\graphicspath{{image/}}

% \centerimage{options}{image}
\newcommand{\centerimage}[2]{\begin{center}
    \includegraphics[#1]{#2}
\end{center}}
%===============================================================================

%=================================CODE LISTINGS=================================
\definecolor{codebackdrop}{gray}{0.9}
\definecolor{commentgreen}{rgb}{0,0.6,0}
\lstset{
    inputpath=code, 
    commentstyle=\color{commentgreen},
    keywordstyle=\color{blue}, 
    backgroundcolor=\color{codebackdrop}, 
    basicstyle=\footnotesize,
    frame=single,
    numbers=left,
    stepnumber=1,
    showstringspaces=false,
    breaklines=true,
    postbreak=\mbox{\textcolor{red}{$\hookrightarrow$}\space}
}

% Create a code listing for a single line
% \codeline{language}{line}{file}
\newcommand{\codeline}[3]{\lstinputlisting[language=#1, firstline = #2, lastline = #2]{#3}}

% Create a code listing for a given language & file
% \codelist{language}{file}
\newcommand{\codelist}[2]{\lstinputlisting[language=#1]{#2}}
%===============================================================================

%================================TEXT STRUCTURES================================
% Marka a word as bold
% \keyword{important word}
\newcommand{\keyword}[1]{\textbf{#1}}

% Creates a section in italics
% \question{question in italics}
\newcommand{\question}[1]{\textit{#1} \\ }

% Creates a box with title for side notes.
% \sidenote{title}{contents}
\newcommand{\sidenote}[2]{\begin{tcolorbox}[title=#1]#2\end{tcolorbox}}

% Creates an item in an itemize or enumerate, with a paragraph after
% \begin{itemize}
%     \bullpara{title}{contents}
% \end{itemize}
\newcommand{\bullpara}[2]{\item \textbf{#1} \ #2}

% Creates a compact list (very small gaps between items)
% \compitem{
%     \item item 1
%     \item item 2
%     \item ...
% }
\newcommand{\compitem}[1]{\begin{itemize}\setlength\itemsep{-0.5em}#1\end{itemize}}

% Creates a link to the lecture for use at the start of the notes document
\newcommand{\lectlink}[1]{\sidenote{Lecture Recording}{
    Lecture recording is available \href{#1}{here}
}}
%===============================================================================


%================================CONFIGURATIONS================================
\newcommand{\config}[2]{\langle #1, #2 \rangle}
%==============================================================================

%==============================BIG STEP SEMANTICS==============================
\newcommand{\bigstep}[4]{\text{(#1)}\dfrac{#2}{#3 \Downarrow #4}}
\newcommand{\bigstepdef}[5]{\bigstep{#1}{#2}{#3}{#4} \ #5}
%==============================================================================

%=============================SMALL STEP SEMANTICS=============================
\newcommand{\smallstep}[4]{\text{(#1)}\dfrac{#2}{#3 \to #4}}
\newcommand{\smallstepdef}[5]{\smallstep{#1}{#2}{#3}{#4} \ #5}
%==============================================================================

\newcommand{\whilest}[3]{\text{(#1)}\dfrac{#2}{#3}}
\newcommand{\whilestdef}[4]{\whilest{#1}{#2}{#3} \ #4}
%==============================================================================

%===================================COMMANDS===================================
\newcommand{\while}[2]{\text{while } #1 \text{ do } #2}
\newcommand{\cond}[3]{\text{if } #1 \text{ then } #2 \text{ else } #3}
\newcommand{\doret}[2]{\text{do } #1 \text{ return } #2}

\newcommand{\instrlabel}[1]{\text{\textcolor{teal}{$L_{#1}$}}}
\newcommand{\reglabel}[1]{\text{\textcolor{orange}{$R_{#1}$}}}
\newcommand{\regtemp}[1]{\text{\textcolor{orange}{$#1$}}}
\newcommand{\instr}[2]{\instrlabel{#1} : & #2 \\}
\newcommand{\dec}[3]{\reglabel{#1}^- \to \instrlabel{#2}, \instrlabel{#3}}
\newcommand{\inc}[2]{\reglabel{#1}^+ \to \instrlabel{#2}}
\newcommand{\halt}{\text{\textcolor{red}{\textbf{HALT}}}}

\newcommand{\lambapp}[2]{#1 \ #2}
\newcommand{\lambappb}[2]{(#1) \ (#2)}
\newcommand{\lambfun}[2]{\lambda #1 \ . \ #2}

\makeatletter
\newcommand{\lambarg}[1]{%
	#1 \checknextarglambarg}
\newcommand{\checknextarglambarg}{\@ifnextchar\bgroup{\gobblenextarglambarg}{}}
\newcommand{\gobblenextarglambarg}[1]{ \ #1 \@ifnextchar\bgroup{\gobblenextarglambarg}{}}

\newcommand{\chnum}[1]{\underline{#1}}
% Variable argument regconfig \regconfig{label no}{reg val}{reg val}...
\makeatletter
\newcommand{\regconfig}[2]{%
	$#1$ & $#2$\checknextarg}
\newcommand{\checknextarg}{\@ifnextchar\bgroup{\gobblenextarg}{\\}}
\newcommand{\gobblenextarg}[1]{ & $#1$\@ifnextchar\bgroup{\gobblenextarg}{\\}}
%==============================================================================
\begin{document}
    \maketitle
    \lectlink{https://imperial.cloud.panopto.eu/Panopto/Pages/Viewer.aspx?id=aa77fc96-d37a-469b-8982-aded0092e925}

    \section*{Halting Problem for Register Machines}
        A register machine $H$ decides the halting problem if for all $e, a_1, \dots, a_n \in \mathbb{N}$:
        \[\begin{matrix}
            \reglabel{0} = 0 & \reglabel{1} = e & \reglabel{2} = \ulcorner [a_1, \dots, a_n]\urcorner & \reglabel{3..} = 0 \\
        \end{matrix}\]
        And where $H$ halt with the state as follows:
        \[\reglabel{0} = \begin{cases}
            1 & \text{Register machine encoded as $e$ halts when started with $\reglabel{0} = 0, \reglabel{1} = a_1, \dots, \reglabel{n} = a_n$} \\
            0 & otherwise \\
        \end{cases}\]
        We can prove that there is no such machine $H$ through a contradiction.
        \centerimage{width=0.9\textwidth}{halting problem}
        Hence when we run $C$ with the argument $C$ we get a contradiction.
        \compitem{
            \bullpara{$C(C)$ Halts}{Then $C$ with $\reglabel{1} = \ulcorner C \urcorner$ as an argument does not halt, which is a contradiction}
            \bullpara{$C(C)$ Does not Halt}{Then $C$ with $\reglabel{1} = \ulcorner C \urcorner$ as an argument halts, which is a contradiction}
        }
    \section*{Computable Functions}
        \subsection*{Enumerating the Computable Functions}  
            \termdef{Onto (Surjective)}{
                Each element in the codomain is mapped to by at least one element in the domain.
                \[\forall y \in Y. \  \exists x \in X. \  [f(x) = y] \Rightarrow f \text{ is onto}\]
            }
            For each $e \in \mathbb{N}$, $\varphi_e \in \mathbb{N} \rightharpoonup \mathbb{N}$ (partial function computed by $program(e)$):
            \[\varphi_e(x) = y \Leftrightarrow program(e) \text{ with } \reglabel{0} = 0 \land \reglabel{1} = x \text{ halts with } \reglabel{0} = y\]
            Hence for a given program $\in \mathbb{N}$ we can get the computable partial function of the program.
            \[e \mapsto \varphi_e\]
            Therefore the above mapping represents an \keyword{onto/surjective} function from $\mathbb{N}$ to all computable partial functions from $\mathbb{N} \rightharpoonup \mathbb{N}$.
        \subsection*{Uncomputable Functions}
            \sidenote{$\uparrow$ and $\downarrow$}{
                As in Prof Wicklicky's first lecture, for $f: X \rightharpoonup Y$ (partial function from $X$ to $Y$):
                \[\begin{split}
                    f(x)\uparrow & \triangleq \neg \exists y \in Y . \ [f(x) = y] \\
                    f(x)\downarrow & \triangleq  \exists y \in Y . \ [f(x) = y] \\
                \end{split}\]
            }
            Hence we can attempt to define a function to determine if a function halts.
            \[f \in \mathbb{N} \rightharpoonup \mathbb{N} \triangleq \{(x,0) | \varphi_x(x)\uparrow\} \triangleq f(x) = \begin{cases}
                0 & \varphi_x(x)\uparrow \\
                undefined & \varphi_x(x)\downarrow
            \end{cases}\]
            However we run into the halting problem:
            \\
            \\ Assume $f$ is computable, then $f = \varphi_e$ for some $e \in \mathbb{N}$.
            \compitem{
                \bullpara{if $\varphi_e(e)\uparrow$}{by definition of $f$, $\varphi_e(e) = 0$ so $\varphi_e(e)\downarrow$ which is a contradiction}
                \bullpara{if $\varphi_e(e)\downarrow$}{by definition of $f$, $f(e)\uparrow$, and hence as $f = \varphi_e$, $\varphi_e\uparrow$ which is a contradiction}
            }
            Here we have ended up with the halting problem being uncomputable.
        
        \subsection*{Undecidable Set of Numbers}
            Given a set $S \subseteq \mathbb{N}$, its characteristic function is:
            \[\chi_S \in \mathbb{N} \to \mathbb{N} \ \ \chi_S(x) \triangleq \begin{cases}
                1 & x \in S \\
                0 & x \not\in S \\
            \end{cases}\]
            $S$ is \keyword{register machine decidable} if its characteristic function is a register machine computable function.
            \\
            \\ $S$ is decidable iff there is a register machine $M$ such that for all $x \in \mathbb{N}$ when run with $\reglabel{0} = 0, \reglabel{1} = x$ and $\reglabel{2..} = 0$ it eventually halts with:
            \compitem{
                \bullpara{$\reglabel{0} = 1$}{ if and only if $x \in S$}
                \bullpara{$\reglabel{0} = 1$}{ if and only if $x \not\in S$}
            }
            Hence we are effectively asking if a register machine exists that can determine if any number is in some set $S$.
            \\
            \\ We can then define subsets of $\mathbb{N}$ that are decidable/undecidable.

            \subsubsection*{The set of functions mapping $0$ is undecidable}
                Given a set:
                \[S_0 \triangleq \{e | \varphi_e(0)\downarrow\}\]
                Hence we are finding the set of the indexes (numbers representing register machines) that halt on input $0$.
                \\
                \\ If such a machine exists, we can use it to create a register machine to solve the halting problem. Hence this is a contradiction, so the set is undecidable.

            \subsubsection*{The set of total functions is undecidable}
                Take set $S_1 \subseteq \mathbb{N}$:
                \[S_1 \triangleq \{e | \varphi_e\text{total function}\}\]
                If such a register machine exists to compute $\chi_{S_1}$, we can create another register machine, simply checking $0$. Hence as from the previous example, there is no register machine to determine $S_0$, there is none to determine $S_1$.

\end{document}
