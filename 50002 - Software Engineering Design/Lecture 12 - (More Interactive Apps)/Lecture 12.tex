\documentclass{report}
    \title{50002 - Software Engineering Design - Lecture 12}
    \author{Oliver Killane}
    \date{22/11/21}

%===========================COMMON FORMAT & COMMANDS===========================
% This file contains commands and format to be used by every module, and is 
% included in all files.
%===============================================================================

%====================================IMPORTS====================================
\usepackage[a4paper, total={6in, 8in}]{geometry}
\usepackage{graphicx, amssymb, amsfonts, amsmath, xcolor, listings, tcolorbox, multirow, hyperref}
%===============================================================================

%====================================IMAGES=====================================
\graphicspath{{image/}}

% \centerimage{options}{image}
\newcommand{\centerimage}[2]{\begin{center}
    \includegraphics[#1]{#2}
\end{center}}
%===============================================================================

%=================================CODE LISTINGS=================================
\definecolor{codebackdrop}{gray}{0.9}
\definecolor{commentgreen}{rgb}{0,0.6,0}
\lstset{
    inputpath=code, 
    commentstyle=\color{commentgreen},
    keywordstyle=\color{blue}, 
    backgroundcolor=\color{codebackdrop}, 
    basicstyle=\footnotesize,
    frame=single,
    numbers=left,
    stepnumber=1,
    showstringspaces=false,
    breaklines=true,
    postbreak=\mbox{\textcolor{red}{$\hookrightarrow$}\space}
}

% Create a code listing for a single line
% \codeline{language}{line}{file}
\newcommand{\codeline}[3]{\lstinputlisting[language=#1, firstline = #2, lastline = #2]{#3}}

% Create a code listing for a given language & file
% \codelist{language}{file}
\newcommand{\codelist}[2]{\lstinputlisting[language=#1]{#2}}
%===============================================================================

%================================TEXT STRUCTURES================================
% Marka a word as bold
% \keyword{important word}
\newcommand{\keyword}[1]{\textbf{#1}}

% Creates a section in italics
% \question{question in italics}
\newcommand{\question}[1]{\textit{#1} \\ }

% Creates a box with title for side notes.
% \sidenote{title}{contents}
\newcommand{\sidenote}[2]{\begin{tcolorbox}[title=#1]#2\end{tcolorbox}}

% Creates an item in an itemize or enumerate, with a paragraph after
% \begin{itemize}
%     \bullpara{title}{contents}
% \end{itemize}
\newcommand{\bullpara}[2]{\item \textbf{#1} \ #2}

% Creates a compact list (very small gaps between items)
% \compitem{
%     \item item 1
%     \item item 2
%     \item ...
% }
\newcommand{\compitem}[1]{\begin{itemize}\setlength\itemsep{-0.5em}#1\end{itemize}}

% Creates a link to the lecture for use at the start of the notes document
\newcommand{\lectlink}[1]{\sidenote{Lecture Recording}{
    Lecture recording is available \href{#1}{here}
}}
%===============================================================================


\begin{document}
    \maketitle
    \lectlink{https://imperial.cloud.panopto.eu/Panopto/Pages/Viewer.aspx?id=f8370610-159a-41b1-b7d1-ade500f91cef}

    \section*{Testing}
        By using the \keyword{Model-View-Controller} pattern we can make it easy to test the logic of our application, as we can test the model in isolation.
        \\
        \\ However we may want to test the entire application to ensure not only that the view and controller components function, but that there are no bugs from integrating them together.
        \\
        \\ To do this we do a \keyword{System Test} which interacts with an instance of our application.
        \centerimage{width=\textwidth}{system test.png}
        The driver interacts with the framework used for displaying graphics in order to get value of objects, and pass commands.
    
    \section*{Window Licker}
        A driver used for testing Java based UIs, to use it we:
        \begin{enumerate}
            \bullpara{Create driver code}{
                \\ To make it easier to access buttons, text firleds etc we can create functions to make use of WindowLick's library to get components and interact with them.
                \codelist{Java}{driver setup.java}
            }
            \bullpara{Create Test Code}{
                \\ Using the driver to control, add test code.
                \codelist{Java}{test code.java}
            }
            \bullpara{Run Tests}{
                \\ Can run the tests in a headless mode, or display the GUI \& let the \keyword{WindowLick} driver control interactions.
            }
        \end{enumerate}
    
    \section*{Web Application}
        We applications use the browser to view, and usually have the controller as the dominant object.
        \centerimage{width=0.8\textwidth}{webapp.png}
        For webapp we can use a tool such as \keyword{Selenium} which automates browser control. 
        \\
        \\ We can use \keyword{Selenium} to:
        \compitem{
            \item connect to the webapp.
            \item Get handles for html components using tags, names, if necessary xpaths etc
            \item Interact with components.
        }
        It offers many drivers for interacting using different browsers, as well as headless modes, and can run many browsers in parallel to speed up large test suites.
\end{document}
