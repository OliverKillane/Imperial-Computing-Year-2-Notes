\documentclass{report}
    \title{50002 - Software Engineering Design - Software Engineering Design}
    \author{Oliver Killane}
    \date{02/01/22}
%===========================COMMON FORMAT & COMMANDS===========================
% This file contains commands and format to be used by every module, and is 
% included in all files.
%===============================================================================

%====================================IMPORTS====================================
\usepackage[a4paper, total={6in, 8in}]{geometry}
\usepackage{graphicx, amssymb, amsfonts, amsmath, xcolor, listings, tcolorbox, multirow, hyperref}
%===============================================================================

%====================================IMAGES=====================================
\graphicspath{{image/}}

% \centerimage{options}{image}
\newcommand{\centerimage}[2]{\begin{center}
    \includegraphics[#1]{#2}
\end{center}}
%===============================================================================

%=================================CODE LISTINGS=================================
\definecolor{codebackdrop}{gray}{0.9}
\definecolor{commentgreen}{rgb}{0,0.6,0}
\lstset{
    inputpath=code, 
    commentstyle=\color{commentgreen},
    keywordstyle=\color{blue}, 
    backgroundcolor=\color{codebackdrop}, 
    basicstyle=\footnotesize,
    frame=single,
    numbers=left,
    stepnumber=1,
    showstringspaces=false,
    breaklines=true,
    postbreak=\mbox{\textcolor{red}{$\hookrightarrow$}\space}
}

% Create a code listing for a single line
% \codeline{language}{line}{file}
\newcommand{\codeline}[3]{\lstinputlisting[language=#1, firstline = #2, lastline = #2]{#3}}

% Create a code listing for a given language & file
% \codelist{language}{file}
\newcommand{\codelist}[2]{\lstinputlisting[language=#1]{#2}}
%===============================================================================

%================================TEXT STRUCTURES================================
% Marka a word as bold
% \keyword{important word}
\newcommand{\keyword}[1]{\textbf{#1}}

% Creates a section in italics
% \question{question in italics}
\newcommand{\question}[1]{\textit{#1} \\ }

% Creates a box with title for side notes.
% \sidenote{title}{contents}
\newcommand{\sidenote}[2]{\begin{tcolorbox}[title=#1]#2\end{tcolorbox}}

% Creates an item in an itemize or enumerate, with a paragraph after
% \begin{itemize}
%     \bullpara{title}{contents}
% \end{itemize}
\newcommand{\bullpara}[2]{\item \textbf{#1} \ #2}

% Creates a compact list (very small gaps between items)
% \compitem{
%     \item item 1
%     \item item 2
%     \item ...
% }
\newcommand{\compitem}[1]{\begin{itemize}\setlength\itemsep{-0.5em}#1\end{itemize}}

% Creates a link to the lecture for use at the start of the notes document
\newcommand{\lectlink}[1]{\sidenote{Lecture Recording}{
    Lecture recording is available \href{#1}{here}
}}
%===============================================================================

\begin{document}
\maketitle

\section*{Test Driven Development}
\subsection*{Design Principles}
\compitem{
	\bullpara{Simple Design}{
		\\ Smaller codebase is easier to reason about, add new features to.}
	\bullpara{Correct Behaviour}{
		\\ Thorough testing (e.g unit testing, automated test suites) used to ensure program behaves as required \& is bug free.
	}
	\bullpara{Reduce Duplication}{
		\\ Duplication leads to a larger codebase, greater potential for bugs (changing one instance but nmot another), and makes code changes more arduous (to find and change duplications).
	}
	\bullpara{Maximum Clarity}{
		\\ Code must be easy to read \& understand (e.g descriptive naming schemes, commenting, documentation)
	}
}

\subsection*{Waterfall Development}
\centerimage{width=\textwidth}{Test Driven Development/waterfall.png}
\compitem {
	\item Often each stage is managed by a separate team.
	\item Difficult to correct issues caused by previous stages (e.g altering design when discovering an issue in coding, or fixing major code when finding bugs in testing).
}

\subsection*{Test Driven Development}
\centerimage{width=\textwidth}{Test Driven Development/test driven development.png}
Software developed in a cycle.
\compitem{
	\item Tests define the public api \& vice-versa.
	\item Programming to satisfy tests requires altering program internals (implementation details).
	\item Refactoring alters the structure but not behaviour (structural design), keepuing the code in a green state.
}
\codelist{Java}{Test Driven Development/junit example.java}

\end{document}
