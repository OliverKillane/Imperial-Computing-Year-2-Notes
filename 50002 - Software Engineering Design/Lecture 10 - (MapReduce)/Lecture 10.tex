\documentclass{report}
    \title{50002 - Software Engineering Design - Lecture 10}
    \author{Oliver Killane}
    \date{14/11/21}

%===========================COMMON FORMAT & COMMANDS===========================
% This file contains commands and format to be used by every module, and is 
% included in all files.
%===============================================================================

%====================================IMPORTS====================================
\usepackage[a4paper, total={6in, 8in}]{geometry}
\usepackage{graphicx, amssymb, amsfonts, amsmath, xcolor, listings, tcolorbox, multirow, hyperref}
%===============================================================================

%====================================IMAGES=====================================
\graphicspath{{image/}}

% \centerimage{options}{image}
\newcommand{\centerimage}[2]{\begin{center}
    \includegraphics[#1]{#2}
\end{center}}
%===============================================================================

%=================================CODE LISTINGS=================================
\definecolor{codebackdrop}{gray}{0.9}
\definecolor{commentgreen}{rgb}{0,0.6,0}
\lstset{
    inputpath=code, 
    commentstyle=\color{commentgreen},
    keywordstyle=\color{blue}, 
    backgroundcolor=\color{codebackdrop}, 
    basicstyle=\footnotesize,
    frame=single,
    numbers=left,
    stepnumber=1,
    showstringspaces=false,
    breaklines=true,
    postbreak=\mbox{\textcolor{red}{$\hookrightarrow$}\space}
}

% Create a code listing for a single line
% \codeline{language}{line}{file}
\newcommand{\codeline}[3]{\lstinputlisting[language=#1, firstline = #2, lastline = #2]{#3}}

% Create a code listing for a given language & file
% \codelist{language}{file}
\newcommand{\codelist}[2]{\lstinputlisting[language=#1]{#2}}
%===============================================================================

%================================TEXT STRUCTURES================================
% Marka a word as bold
% \keyword{important word}
\newcommand{\keyword}[1]{\textbf{#1}}

% Creates a section in italics
% \question{question in italics}
\newcommand{\question}[1]{\textit{#1} \\ }

% Creates a box with title for side notes.
% \sidenote{title}{contents}
\newcommand{\sidenote}[2]{\begin{tcolorbox}[title=#1]#2\end{tcolorbox}}

% Creates an item in an itemize or enumerate, with a paragraph after
% \begin{itemize}
%     \bullpara{title}{contents}
% \end{itemize}
\newcommand{\bullpara}[2]{\item \textbf{#1} \ #2}

% Creates a compact list (very small gaps between items)
% \compitem{
%     \item item 1
%     \item item 2
%     \item ...
% }
\newcommand{\compitem}[1]{\begin{itemize}\setlength\itemsep{-0.5em}#1\end{itemize}}

% Creates a link to the lecture for use at the start of the notes document
\newcommand{\lectlink}[1]{\sidenote{Lecture Recording}{
    Lecture recording is available \href{#1}{here}
}}
%===============================================================================


\lstdefinelanguage{JavaScript}{
  keywords={break, case, catch, continue, debugger, default, delete, do, else, false, finally, for, function, if, in, instanceof, new, null, return, switch, this, throw, true, try, typeof, var, void, while, with},
  morecomment=[l]{//},
  morecomment=[s]{/*}{*/},
  morestring=[b]',
  morestring=[b]",
  ndkeywords={class, export, boolean, throw, implements, import, this},
  keywordstyle=\color{blue}\bfseries,
  ndkeywordstyle=\color{darkgray}\bfseries,
  identifierstyle=\color{black},
  commentstyle=\color{purple}\ttfamily,
  stringstyle=\color{red}\ttfamily,
  sensitive=true
}

\begin{document}
\maketitle
\lectlink{https://imperial.cloud.panopto.eu/Panopto/Pages/Viewer.aspx?id=0fc33d52-21b7-4577-a6c4-adde00f851cc}

\section*{MapReduce}
A system built by Google to work on very large datasets, allowing data processing tasks to be spread over many computers in a farm.
\\
The main difficulties win distributed computing considered were:
\compitem{
	\item If many projects are changed to be distributed, then lots of code must be changed, much will be duplicated \& there is potential for bugs.
	\item Ordinging interactions and sharing data across many networked computers is difficult.
	\item Probability of a hardware failure grows with the amount of hardware.
}
MapReduce takes tasks written by programmers, and distributises it across a number of systems, handling failures \& can be easily used through an \keyword{API}.
Furthermore any improvements to the system will immediately be reflected in all projects using it.
\\
\\ Google wanted to index the world wide web, this is a large problem that is best suited to highly distributed programming.
\codelist{Javascript}{simple mapreduce.js}

\centerimage{width=0.6\textwidth}{mapreduce.png}
The shuffle stage is important, to increase performance we want to transmit/copy the minimum amount of data between computers.
\\
\\ For \keyword{MapReduce} the signatures are different to lists \& integers:
\[map: (k_1,v_1) \to list(k_2, v_2)\]
\[reduce: list(k_2, list(v_2)) \to list(k_2, v_2)\]
For example:
We have $k_1 = \text{line number}$ and $v_1 = \text{line of text}$
\begin{center}
	\begin{tabular}{c l}
		\textbf{Key} & \textbf{Value}                                                                                       \\
		1            & I think that it's extraordinarily important that we in computer science keep fun in computing.       \\
		2            & When it started out, it was an awful lot of fun. Of course, the paying customers got shafted every   \\
		3            & now and then, and after a while we began to take their complaints seriously. We began to feel as     \\
		4            & if we really were responsible for the successful, error-free perfect use of these machines. I don't  \\
		5            & think we are. I think we're responsible for stretching them, setting them off in new directions, and \\
		6            & keeping fun in the house. I hope the field of computer science never loses its sense of fun          \\
	\end{tabular}
\end{center}
$map (k_1, v_1) \to [(k_2, v_2)]$ where $k_2 = \text{word}$ and $v_2 = \text{1}$.
\codelist{Python}{example map.py}
The we reduce with $+$ for each key:
\\ $reduce [(k_2, v_2)] \to [(k_2, [v_2])]$:
\codelist{Python}{example reduce.py}

\subsubsection*{Alternatives}
Other projects have been developed to perform a similar role to \keyword{MapReduce} such as \keyword{Hadoop}.

\section*{Java Example}
\codelist{Java}{Hadoop java.java}
\end{document}
