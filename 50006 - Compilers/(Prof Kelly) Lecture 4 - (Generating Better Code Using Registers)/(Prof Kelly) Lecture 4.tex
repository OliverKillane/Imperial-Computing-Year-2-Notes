\documentclass{report}
    \title{50006 - Compilers - (Prof Kelly) Lecture 4}
    \author{Oliver Killane}
    \date{11/01/22}
%===========================COMMON FORMAT & COMMANDS===========================
% This file contains commands and format to be used by every module, and is 
% included in all files.
%===============================================================================

%====================================IMPORTS====================================
\usepackage[a4paper, total={6in, 8in}]{geometry}
\usepackage{graphicx, amssymb, amsfonts, amsmath, xcolor, listings, tcolorbox, multirow, hyperref}
%===============================================================================

%====================================IMAGES=====================================
\graphicspath{{image/}}

% \centerimage{options}{image}
\newcommand{\centerimage}[2]{\begin{center}
    \includegraphics[#1]{#2}
\end{center}}
%===============================================================================

%=================================CODE LISTINGS=================================
\definecolor{codebackdrop}{gray}{0.9}
\definecolor{commentgreen}{rgb}{0,0.6,0}
\lstset{
    inputpath=code, 
    commentstyle=\color{commentgreen},
    keywordstyle=\color{blue}, 
    backgroundcolor=\color{codebackdrop}, 
    basicstyle=\footnotesize,
    frame=single,
    numbers=left,
    stepnumber=1,
    showstringspaces=false,
    breaklines=true,
    postbreak=\mbox{\textcolor{red}{$\hookrightarrow$}\space}
}

% Create a code listing for a single line
% \codeline{language}{line}{file}
\newcommand{\codeline}[3]{\lstinputlisting[language=#1, firstline = #2, lastline = #2]{#3}}

% Create a code listing for a given language & file
% \codelist{language}{file}
\newcommand{\codelist}[2]{\lstinputlisting[language=#1]{#2}}
%===============================================================================

%================================TEXT STRUCTURES================================
% Marka a word as bold
% \keyword{important word}
\newcommand{\keyword}[1]{\textbf{#1}}

% Creates a section in italics
% \question{question in italics}
\newcommand{\question}[1]{\textit{#1} \\ }

% Creates a box with title for side notes.
% \sidenote{title}{contents}
\newcommand{\sidenote}[2]{\begin{tcolorbox}[title=#1]#2\end{tcolorbox}}

% Creates an item in an itemize or enumerate, with a paragraph after
% \begin{itemize}
%     \bullpara{title}{contents}
% \end{itemize}
\newcommand{\bullpara}[2]{\item \textbf{#1} \ #2}

% Creates a compact list (very small gaps between items)
% \compitem{
%     \item item 1
%     \item item 2
%     \item ...
% }
\newcommand{\compitem}[1]{\begin{itemize}\setlength\itemsep{-0.5em}#1\end{itemize}}

% Creates a link to the lecture for use at the start of the notes document
\newcommand{\lectlink}[1]{\sidenote{Lecture Recording}{
    Lecture recording is available \href{#1}{here}
}}
%===============================================================================


\newcommand{\hot}[1]{\textcolor{red}{#1}}
\newcommand{\old}[1]{\textcolor{blue}{#1}}

\begin{document}
    \maketitle
    \lectlink{https://imperial.cloud.panopto.eu/Panopto/Pages/Viewer.aspx?id=53a89316-6ad3-4461-8f57-ae14010081b6}

    \section*{Unbounded Register Use}
        We will generate code for arithmetic expressions that:
        \compitem {
            \item Assumes there will always be enough registers.
            \item Handles the case when we run out of registers.
            \item Translates expressions while minimising number of registers required.
        }
        \codelines{Haskell}{17}{30}{register machine.hs}
        We can take an input such as:
        \[(100 * 3) + ((200 * 2) + 300) + (400 + (500 * 3))\]
        


        \begin{center}
            \begin{tabular}{c l | l l l l}
                & \multirow{2}{*}{Instruction} & \multicolumn{4}{c}{Stack Slot} \\
                & & 3 & 2 & 1 & 0 \\
                \hline
                0 & PushImm 100 &       &          &          & \hot{100}\\ 
                1 & PushImm 3   &       &          &   \hot{3}&       100\\ 
                2 & Mul         &       &          &  \old{3} & \hot{300}\\ 
                3 & PushImm 200 &       &          & \hot{200}&       300\\ 
                4 & PushImm 2   &       &\hot{2}   &       200&       300\\ 
                5 & Mul         &       &   \old{2}& \hot{400}&       300\\ 
                6 & PushImm 300 &       &\hot{300} &       400&       300\\ 
                7 & Add         &       & \old{300}& \hot{700}&       300\\ 
                8 & Add         &       & \old{300}& \old{700}&\hot{1000}\\ 
                9 & PushImm 400 &       & \old{300}& \hot{400}&      1000\\ 
               10 & PushImm 500 &       &\hot{500} &       400&      1000\\ 
               11 & PushImm 3   &\hot{3}&       500&       400&      1000\\ 
               12 & Mul         &\old{3}&\hot{1500}&       400&      1000\\ 
               13 & Add         &\old{3}&\old{1500}&\hot{1900}&      1000\\ 
               14 & Add         &\old{3}&\old{1500}&\old{1900}&\hot{2900}\\
            \end{tabular}
        \end{center}
        We can use the placement of values in the stack (relative to the initial stack pointer) to assign registers. Assigning each slot to a register.
        \\
        \\ We want to provide the translator with the register to place the result in, it can use any higher registers.
        \begin{center}
            \begin{tabular}{c l | l l l l}
                & \multirow{2}{*}{Instruction} & \multicolumn{4}{c}{Register} \\
                & & R3 & R2 & R1 & R0 \\
                \hline
                0 &LoadImm R0 100&       &          &          & \hot{100}\\ 
                1 &LoadImm R1 3  &       &          &   \hot{3}&       100\\ 
                2 &Mul R0 R1     &       &          &  \old{3} & \hot{300}\\ 
                3 &LoadImm R1 200&       &          & \hot{200}&       300\\ 
                4 &LoadImm R2 2  &       &\hot{2}   &       200&       300\\ 
                5 &Mul R1 R2     &       &   \old{2}& \hot{400}&       300\\ 
                6 &LoadImm R2 300&       &\hot{300} &       400&       300\\ 
                7 &Add R1 R2     &       & \old{300}& \hot{700}&       300\\ 
                8 &Add R0 R1     &       & \old{300}& \old{700}&\hot{1000}\\ 
                9 &LoadImm R1 400&       & \old{300}& \hot{400}&      1000\\ 
               10 &LoadImm R2 500&       &\hot{500} &       400&      1000\\ 
               11 &LoadImm R3 3  &\hot{3}&       500&       400&      1000\\ 
               12 &Mul R2 R3     &\old{3}&\hot{1500}&       400&      1000\\ 
               13 &Add R1 R2     &\old{3}&\old{1500}&\hot{1900}&      1000\\ 
               14 &Add R0 R1     &\old{3}&\old{1500}&\old{1900}&\hot{2900}\\
            \end{tabular}
        \end{center}
        \subsection*{Register Improvements}
            We could improve our generated code if there were instructions available for in place constant application.
            \\ \begin{minipage}[t]{0.4\textwidth}
                \centerline{\textbf{IA32 with Immediate Operand}}
                \codelist{{[x86masm]Assembler}}{in place.s}
            \end{minipage}
            \hfill
            \begin{minipage}[t]{0.4\textwidth}
                \centerline{\textbf{Our Simple Assembly}}
                \codelist{Python}{our code.pesudo}
            \end{minipage}

            \begin{center}
                \begin{tabular}{c l | l l l l}
                    & \multirow{2}{*}{Instruction} & \multicolumn{4}{c}{Register} \\
                    & & R3 & R2 & R1 & R0 \\
                    \hline
                    0 & LoadImm 0 3   &     &        &          &   \hot{3} \\
                    1 & MulImm 0 100  &     &        &          & \hot{300} \\
                    3 & LoadImm 1 2   &     &        &   \hot{2}&       300 \\
                    4 & MulImm 1 200  &     &        & \hot{400}&       300 \\
                    5 & AddImm 1 300  &     &        & \hot{700}&       300 \\
                    6 & Add 0 1       &     &        &\old{700} &\hot{1000} \\
                    7 & LoadImm 1 3   &     &        &   \hot{3}&      1000 \\
                    8 & MulImm 1 500  &     &        &\hot{1500}&      1000 \\
                    9 & AddImm 1 400  &     &        &\hot{1900}&      1000 \\
                   10 & Add 0 1       &     &        &\old{1900}&\hot{2900} \\
                \end{tabular}
            \end{center}
        
        \subsection*{Code for Translation}
            \codelines{Haskell}{40}{70}{register machine.hs}
    
    \section*{Bounded Number of Registers}
        \subsection*{Accumulator Machine}
            Has a single register (\keyword{Accumulator}) upon which arithmetic instructions can be applied.
            \codelines{Haskell}{17}{28}{accumulator machine.hs}
            \begin{center}
                \begin{tabular}{c l | l l l | l}
                    & \multirow{2}{*}{Instruction} & \multirow{2}{*}{Acc} &\multicolumn{2}{c}{Stack} \\
                    & & & 1 & 0 \\
                    \hline
                    0 &LoadImm 500  & \hot{500}&         &           \\
                    1 &MulImm 3     &\hot{1500}&         &           \\
                    3 &AddImm 400   &\hot{1900}&         &           \\
                    4 &Push         &\old{1900}&         &\hot{1900} \\
                    5 &LoadImm 200  & \hot{200}&         &      1900 \\
                    6 &MulImm 2     & \hot{400}&         &      1900 \\
                    7 &AddImm 300   & \hot{700}&         &      1900 \\
                    8 &Push         & \old{700}&\hot{700}&      1900 \\
                    9 &LoadImm 100  & \hot{100}&      700&      1900 \\
                   10 &MulImm 3     & \hot{300}&      700&      1900 \\
                   11 &Add          &\hot{1000}&\old{700}&      1900 \\
                   12 &Add          &\hot{2900}&\old{700}&\old{1900} \\
                \end{tabular}
            \end{center}

            \codelines{Haskell}{37}{66}{accumulator machine.hs}

        \subsection*{Limited Register Set}
            One solution is to combine the register and accumulator strategies
            \codelines{Haskell}{17}{32}{limited register machine.hs}
            \compitem{
                \item When free register sremain, use the register machine strategy.
                \item When the limit is reached (one register left to use as accumulator), switch to accumulator strategy.
            }
            This results in most expressions using full benefit of registers, while very large expressions can still be correctly executed.
            \codelines{Haskell}{42}{82}{limited register machine.hs}
        
        \

\end{document}
