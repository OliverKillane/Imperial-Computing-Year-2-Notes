\documentclass{report}
    \title{50006 - Compilers - Lecture 1}
    \author{Oliver Killane}
    \date{05/01/22}
%===========================COMMON FORMAT & COMMANDS===========================
% This file contains commands and format to be used by every module, and is 
% included in all files.
%===============================================================================

%====================================IMPORTS====================================
\usepackage[a4paper, total={6in, 8in}]{geometry}
\usepackage{graphicx, amssymb, amsfonts, amsmath, xcolor, listings, tcolorbox, multirow, hyperref}
%===============================================================================

%====================================IMAGES=====================================
\graphicspath{{image/}}

% \centerimage{options}{image}
\newcommand{\centerimage}[2]{\begin{center}
    \includegraphics[#1]{#2}
\end{center}}
%===============================================================================

%=================================CODE LISTINGS=================================
\definecolor{codebackdrop}{gray}{0.9}
\definecolor{commentgreen}{rgb}{0,0.6,0}
\lstset{
    inputpath=code, 
    commentstyle=\color{commentgreen},
    keywordstyle=\color{blue}, 
    backgroundcolor=\color{codebackdrop}, 
    basicstyle=\footnotesize,
    frame=single,
    numbers=left,
    stepnumber=1,
    showstringspaces=false,
    breaklines=true,
    postbreak=\mbox{\textcolor{red}{$\hookrightarrow$}\space}
}

% Create a code listing for a single line
% \codeline{language}{line}{file}
\newcommand{\codeline}[3]{\lstinputlisting[language=#1, firstline = #2, lastline = #2]{#3}}

% Create a code listing for a given language & file
% \codelist{language}{file}
\newcommand{\codelist}[2]{\lstinputlisting[language=#1]{#2}}
%===============================================================================

%================================TEXT STRUCTURES================================
% Marka a word as bold
% \keyword{important word}
\newcommand{\keyword}[1]{\textbf{#1}}

% Creates a section in italics
% \question{question in italics}
\newcommand{\question}[1]{\textit{#1} \\ }

% Creates a box with title for side notes.
% \sidenote{title}{contents}
\newcommand{\sidenote}[2]{\begin{tcolorbox}[title=#1]#2\end{tcolorbox}}

% Creates an item in an itemize or enumerate, with a paragraph after
% \begin{itemize}
%     \bullpara{title}{contents}
% \end{itemize}
\newcommand{\bullpara}[2]{\item \textbf{#1} \ #2}

% Creates a compact list (very small gaps between items)
% \compitem{
%     \item item 1
%     \item item 2
%     \item ...
% }
\newcommand{\compitem}[1]{\begin{itemize}\setlength\itemsep{-0.5em}#1\end{itemize}}

% Creates a link to the lecture for use at the start of the notes document
\newcommand{\lectlink}[1]{\sidenote{Lecture Recording}{
    Lecture recording is available \href{#1}{here}
}}
%===============================================================================

\begin{document}
\maketitle
\lectlink{https://imperial.cloud.panopto.eu/Panopto/Pages/Viewer.aspx?id=fe733829-d36e-4b7f-89d6-ae1400ff2570}

\section*{Course Introduction}
\subsection*{Lecturers}
\compitem{
	\bullpara{Paul Kelly}{
		\compitem {
			\item Introduction to syntax analysis \& toy compiler
			\item Code Generation
			\item Generating better code using registers
			\item Register Allocation
			\item Optimisation and data flow analysis
			\item Loop optimisations
		}
	}
	\bullpara{Naranker Dulay}{
		\compitem {
			\item Lexical Analysis (characters to tokens)
			\item LR (Bottom-up) parsers (tokens to AST)
			\item LL (top-down parsing)
			\item Semantic Analysis, checking program validity
			\item Runtime organisation (memory)
		}
	}
}

\subsection*{Materials}
\compitem{
	\item \href{https://scientia.doc.ic.ac.uk/2122/modules/50006/resources}{Scientia}
	\item Course Textbook
	\item Course Webpages for \href{http://www.doc.ic.ac.uk/~phjk/Compilers}{Paul Kelly} and \href{http://www.doc.ic.ac.uk/~nd/compilers}{Naranker Dulay}
}

Recommended Textbooks:
\begin{itemize}
	\bullpara{The Dragon Book: 'Compilers: Principles, Techniques and Tools'}{
		\\ The definitive book on compilers. Called the dragon book due to its front illustration (allusion to taming the dragon of compiler complexity).
		\\
		\\ Has all required information and lots more (very thick book). \href{https://www.amazon.co.uk/Compilers-Principles-Techniques-Alfred-Aho-dp-0321486811/dp/0321486811/ref=dp_ob_title_bk}{(Amazon link)}
	}
	\bullpara{'Modern Compiler Implementation in Java'}{
		\\ Details a java compiler implementation, with reasoning behind design choiced. \href{https://www.amazon.co.uk/Modern-Compiler-Implementation-Andrew-Appel/dp/052182060X}{(Amazon link)}
	}
	\bullpara{Most Recommended: 'Engineering a Compiler'}{
		\\ Most closely follows course \& well reviewed. \href{https://www.amazon.co.uk/Engineering-Compiler-Keith-Cooper/dp/012088478X}{(Amazon link)}
	}
\end{itemize}

\section*{Compilers Introduction}
A particular class of program called \keyword{language processors}. Compilers can be:
\compitem{
	\bullpara{Processes}{programs written in some language.}
	\bullpara{Writes}{programs in some language.}
	\bullpara{Translates}{programs from one language to an equivalent program in another. (e.g from hier to lower level language)}
}
A tool to enable programs written using high-level concepts to be translated into a low-level implementation.
\centerimage{width=0.8\textwidth}{compiler.png}
\compitem{
	\item Typically from a high level language to a lower level one.
	\item Error messages ensure source program can be fixed easily when failing to compile.
	\item Can compiler from \& to a high-level language (e.g compile to C, or between languages)
	\item Analyse the source program to provide the programmer useful information.
}
\subsection*{Basic Compiler Structure}
\centerimage{width=\textwidth}{compiler structure.png}

\begin{itemize}
	\bullpara{Input}{Take program written in some language.}
	\bullpara{Analysis} {
		\\ Create an internal representation (IR) of the source which encodes its meaning (semantics).
		\\
		\\ This often starts with a tree (abstract syntax tree), but graphs may be used to represent control flow.
		\compitem{
			\bullpara{Lexical Analysis}{ \\ Tokenization, converting text to a series of tokens representing keywords, identifiers, etc.}
			\bullpara{Syntax Analysis}{ \\ Analysis of the nested structure of a program (e.g if-else within if-else within loop block).}
			\bullpara{Semantic Analysis}{ \\ Analysing the meaning of a program such as type checking, determining overloading (e.g is $+$ between variables correct \& what action (e.g add integers, or concantenate strings)). }
		}
	}
	\bullpara{Synthesis} {
		\\ Use the IR to create the program in the target language (retaining the same meaning/semantics).
		\\
		\\ The IR may be analysed and transformed to provide extra information to the programmer (e.g warnings) as well as to optimise (e.g inlining, loop hoisting, loop unrolling, removing dead code).
		\compitem {
			\bullpara{Intermediate Code Generation}{ \\ Encodes more information, and is used in sphisticated compilers for optimisation. For example determinign loop invariant code (code that can be hoisted out of the loop \& only be evaluated once).}
			\bullpara{Optimisation}{ \\ Improving code performance without altering its meaning. }
		}
	}
	\bullpara{Output} {Output the target program. (e.g x86 assembly or Java bytecode)}
\end{itemize}


\subsection*{Symbol Table}
Contains information on all declared identifiers, and is used to check its use (e.g type checking) and generate code for access.
\compitem{
	\item Name
	\item Type of identifier (function, class, variable, constant)
	\item Size (Memory to reserve \& where (e.g stack, data segment, heap))
}

\begin{minipage}[t]{0.45\textwidth}
	\centerline{Code}
	\codelist{C}{C code.c}
\end{minipage}
\hfill
\begin{minipage}[t]{0.45\textwidth}
	\centerline{GCC symbol table}
	\lstinputlisting[basicstyle=\tiny]{C code symbol table.txt}
\end{minipage}

\sidenote{GCC Symbol Table}{
	Simply compile, and then use \fun{nm} on the resulting binary to get a basic representation of GCC's symbol table (that is included with the executable).
}

The symbol table is used in different ways through different phases of the compiler:
\centerimage{width=0.8\textwidth}{compiler phases.png}
\begin{minipage}[t]{0.45\textwidth}
	\centerline{Code}
	\codelist{C}{basic example.c}
	\sidenote{Boilerplate}{
		Not lots of boilerplate assembly is ommited, if you try this locally you will see this.
	}
\end{minipage}
\begin{minipage}[t]{0.1\textwidth}
	\huge
	\[\Rightarrow\]
\end{minipage}
\begin{minipage}[t]{0.45\textwidth}
	\centerline{Assembly}
	\codelist{{[x86masm]Assembler}}{basic example.s}
	\sidenote{GodBolt}{
		The \href{https://godbolt.org/}{compiler explorer} is a great tool for generating \& exploring assembly generated by compilers.
	}
\end{minipage}
\section*{Compiler Phases}
We can show the phases of the compiler through an example. Converting \keyword{C} code into \keyword{AT\&T/GAS} assembly.
\[int_{space}a;_{newline}int_{space}b;_{newline} \ _{newline}void_{space}text\_fun(void)_{space}\{_{newline}a_{space}=_{space}b+42;_{newline}\}\]
[INTtok,
\\ IDENTtok(pointer to 'a' entry in symbol table),
\\ SEMICOLONtok,
\\ INTtok,
\\ IDENTtok(pointer to 'b' entry in symbol table),
\\ SEMICOLONtok,
\\ VOIDtok,
\\ IDENTtok(pointer to 'test\_fun' entry in symbol table),
\\ LBRACKETtok,
\\ VOIDtok,
\\ RBRACKETtok,
\\ LCURLYBRACKETtok,
\\ IDENTtok(pointer to 'a' entry in symbol table),
\\ EQtok,
\\ IDENTtok(pointer to 'b' entry in symbol table),
\\ PLUStok,
\\ CONSTINTtok(123),
\\ SEMICOLONtok,
\\ RCURLYBRACKETtok]


\subsection*{Lexical Analysis}
Also known as tokenization. The input sequence of characters is tokenized and any new identifiers added to the symbol table during tokenization.
\\
\\ Tokens to identifiers include pointers to their corresponding entry in the symbol table.


\subsection*{Syntax Analysis}
Also known as parsing. Using a grammatical structure to generate an \keyword{Abstract Syntax Tree} from the sequence of tokens provided by the previous stage.
\compenum {
	\item Ensure the structure is correct, report an error otherwise.
	\item Build \keyword{Abstract Syntax Tree} from tokens to represent program.
}
The grammar of the language can be expressed in a notation such as \keyword{Backus-Naur Form}. This is much the same as the definitions used for the simple while language in \keyword{Models of Computation}.
\\
\\ A rudimentary example could be:
\begin{center}
	\begin{tabular}{r c l}
		$Cond$ & ::= & $'true' \ | \ 'false' \ | \ Expr == Expr \ | \ Expr < Expr \ | \ Cond \& Cond \ | \ \neg Cond \dots$                            \\
		$Expr$ & ::= & $Var \ | \ Const \ | \ Expr + Expr \ | \ Expr \times Expr \dots$                                                                \\
		$Stat$ & ::= & $x :=Expr \ | \ 'if' \ Cond \ 'then' \ Stat \ 'else' \ Stat \ | \ Stat; \ Stat \ | \ 'skip' \ | \ 'while' \ Cond \ 'do' \ Stat$ \\
	\end{tabular}
\end{center}
The \keyword{Abstract Syntax Tree} is implemented as a tree of linked objects. It must be designed to be efficient to construct \& contain all require information for subsequent stages.
\\
\\ The \keyword{Abstract Syntax Tree} for the basic C program would be something like:
\centerimage{width=\textwidth}{basic example ast.png}

\section*{Domain Specific Languages}
Many programming languages are \keyword{domain specific}, meaning they are designed for a specific use.
\compitem{
	\bullpara{Latex}{Creating documents}
	\bullpara{Markdown}{Simple text files with image, font support (e.g readmes)}
	\bullpara{YAML \& TOML}{Configuration files}
	\bullpara{Verilog}{Digital Circuit Design}
	\bullpara{R}{Statistics}
	\bullpara{Simulink}{Dynamic System Modelling}
	\bullpara{Yacc}{(Yet another compiler-compiler) parser generator language}
}
There are also many language processors that are not compilers:
\compitem{
	\bullpara{FindBugs}{Finds bugs in java code}
	\bullpara{Pylint}{Linter for python}
	\bullpara{Valgrind}{Detecting bugs \& memory leaks}
}

\end{document}