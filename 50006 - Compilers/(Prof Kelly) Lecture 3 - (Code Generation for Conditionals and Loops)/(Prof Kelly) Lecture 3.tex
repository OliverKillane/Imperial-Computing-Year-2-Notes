\documentclass{report}
    \title{50006 - Compilers - (Prof Kelly) Lecture 3}
    \author{Oliver Killane}
    \date{10/01/22}
%===========================COMMON FORMAT & COMMANDS===========================
% This file contains commands and format to be used by every module, and is 
% included in all files.
%===============================================================================

%====================================IMPORTS====================================
\usepackage[a4paper, total={6in, 8in}]{geometry}
\usepackage{graphicx, amssymb, amsfonts, amsmath, xcolor, listings, tcolorbox, multirow, hyperref}
%===============================================================================

%====================================IMAGES=====================================
\graphicspath{{image/}}

% \centerimage{options}{image}
\newcommand{\centerimage}[2]{\begin{center}
    \includegraphics[#1]{#2}
\end{center}}
%===============================================================================

%=================================CODE LISTINGS=================================
\definecolor{codebackdrop}{gray}{0.9}
\definecolor{commentgreen}{rgb}{0,0.6,0}
\lstset{
    inputpath=code, 
    commentstyle=\color{commentgreen},
    keywordstyle=\color{blue}, 
    backgroundcolor=\color{codebackdrop}, 
    basicstyle=\footnotesize,
    frame=single,
    numbers=left,
    stepnumber=1,
    showstringspaces=false,
    breaklines=true,
    postbreak=\mbox{\textcolor{red}{$\hookrightarrow$}\space}
}

% Create a code listing for a single line
% \codeline{language}{line}{file}
\newcommand{\codeline}[3]{\lstinputlisting[language=#1, firstline = #2, lastline = #2]{#3}}

% Create a code listing for a given language & file
% \codelist{language}{file}
\newcommand{\codelist}[2]{\lstinputlisting[language=#1]{#2}}
%===============================================================================

%================================TEXT STRUCTURES================================
% Marka a word as bold
% \keyword{important word}
\newcommand{\keyword}[1]{\textbf{#1}}

% Creates a section in italics
% \question{question in italics}
\newcommand{\question}[1]{\textit{#1} \\ }

% Creates a box with title for side notes.
% \sidenote{title}{contents}
\newcommand{\sidenote}[2]{\begin{tcolorbox}[title=#1]#2\end{tcolorbox}}

% Creates an item in an itemize or enumerate, with a paragraph after
% \begin{itemize}
%     \bullpara{title}{contents}
% \end{itemize}
\newcommand{\bullpara}[2]{\item \textbf{#1} \ #2}

% Creates a compact list (very small gaps between items)
% \compitem{
%     \item item 1
%     \item item 2
%     \item ...
% }
\newcommand{\compitem}[1]{\begin{itemize}\setlength\itemsep{-0.5em}#1\end{itemize}}

% Creates a link to the lecture for use at the start of the notes document
\newcommand{\lectlink}[1]{\sidenote{Lecture Recording}{
    Lecture recording is available \href{#1}{here}
}}
%===============================================================================

\begin{document}
\maketitle
\lectlink{https://imperial.cloud.panopto.eu/Panopto/Pages/Viewer.aspx?id=9e39f043-07bf-41bc-97ea-ae1401000fb5}

\section*{Simple Programming Language}
The grammar is expressed as:
\begin{center}
	\begin{tabular}{r l}
		$stat \to$  & $ident$ ':=' $exp$ $|$ $stat$ ';' $stat$ $|$ 'for' $ident$ 'from' $exp$ 'to' $exp$ 'do' $stat$ 'od \\
		$exp \to$   & $exp$ $binop$ $exp$ $|$ $unop$ $exp$ $|$ $ident$ $|$ $num$                                         \\
		$binop \to$ & '+' $|$ '-' $|$ '*' $|$ '/' $|$                                                                    \\
		$unop \to$  & '-'                                                                                                \\
	\end{tabular}
\end{center}

And abstract syntax tree as:
\codelines{Haskell}{1}{21}{stack compiler.hs}

\section*{Target Stack Machine}
\centerimage{width=0.8\textwidth}{stack machine.png}
Assembly instructions (Some are directives/pesudoinstructions):
\codelines{Haskell}{23}{34}{stack compiler.hs}

Pesudocode for execution behaviour:
\codelist{Python}{microcode.pesudo}

\begin{minipage}[t]{0.4\textwidth}
	\centerline{Typical Assembly}
	\codelist{Python}{assembly.pesudo}
\end{minipage}
\hfill
\begin{minipage}[t]{0.4\textwidth}
	\centerline{Compiler Code Generated}
	\codelist{Haskell}{code gen rep.hs}
\end{minipage}
The define directive (and assembly label) are directives to make the linker/assembler convert jumps to the label into jumps to the memory address of the instruction immediately after the label.

\section*{Translation (Naive Implementation)}
\codelist{haskell}{stack compiler.hs}

\section*{Intermediate Representations}
\compitem {
	\bullpara{Abstract Syntax Tree}{ Usually the first intermediate representation.
		Can include statements, operations and expressions in a uniform way (simple data structure)
		\\ Useful for sophisticated instruction selections and register allocation.
	}
	\bullpara{Flattened Control Flow Graph}{ Represents assembler-level code
		\\ Order of opeerations defines control flow, useful for loop-invariant code motion.
	}
	\bullpara{Dependency Based Graphs}{ More complex, used by most modern compilers.
		\\ Used for optimisations, can create 'static single assignment' graphs to deal with dependencies on mutable data.
	}
}

\end{document}
