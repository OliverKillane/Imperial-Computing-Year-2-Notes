\documentclass{report}
    \title{50008 - Probability and Statistics - Lecture 3}
    \author{Oliver Killane}
    \date{24/01/22}
%===========================COMMON FORMAT & COMMANDS===========================
% This file contains commands and format to be used by every module, and is 
% included in all files.
%===============================================================================

%====================================IMPORTS====================================
\usepackage[a4paper, total={6in, 8in}]{geometry}
\usepackage{graphicx, amssymb, amsfonts, amsmath, xcolor, listings, tcolorbox, multirow, hyperref}
%===============================================================================

%====================================IMAGES=====================================
\graphicspath{{image/}}

% \centerimage{options}{image}
\newcommand{\centerimage}[2]{\begin{center}
    \includegraphics[#1]{#2}
\end{center}}
%===============================================================================

%=================================CODE LISTINGS=================================
\definecolor{codebackdrop}{gray}{0.9}
\definecolor{commentgreen}{rgb}{0,0.6,0}
\lstset{
    inputpath=code, 
    commentstyle=\color{commentgreen},
    keywordstyle=\color{blue}, 
    backgroundcolor=\color{codebackdrop}, 
    basicstyle=\footnotesize,
    frame=single,
    numbers=left,
    stepnumber=1,
    showstringspaces=false,
    breaklines=true,
    postbreak=\mbox{\textcolor{red}{$\hookrightarrow$}\space}
}

% Create a code listing for a single line
% \codeline{language}{line}{file}
\newcommand{\codeline}[3]{\lstinputlisting[language=#1, firstline = #2, lastline = #2]{#3}}

% Create a code listing for a given language & file
% \codelist{language}{file}
\newcommand{\codelist}[2]{\lstinputlisting[language=#1]{#2}}
%===============================================================================

%================================TEXT STRUCTURES================================
% Marka a word as bold
% \keyword{important word}
\newcommand{\keyword}[1]{\textbf{#1}}

% Creates a section in italics
% \question{question in italics}
\newcommand{\question}[1]{\textit{#1} \\ }

% Creates a box with title for side notes.
% \sidenote{title}{contents}
\newcommand{\sidenote}[2]{\begin{tcolorbox}[title=#1]#2\end{tcolorbox}}

% Creates an item in an itemize or enumerate, with a paragraph after
% \begin{itemize}
%     \bullpara{title}{contents}
% \end{itemize}
\newcommand{\bullpara}[2]{\item \textbf{#1} \ #2}

% Creates a compact list (very small gaps between items)
% \compitem{
%     \item item 1
%     \item item 2
%     \item ...
% }
\newcommand{\compitem}[1]{\begin{itemize}\setlength\itemsep{-0.5em}#1\end{itemize}}

% Creates a link to the lecture for use at the start of the notes document
\newcommand{\lectlink}[1]{\sidenote{Lecture Recording}{
    Lecture recording is available \href{#1}{here}
}}
%===============================================================================

\begin{document}
\maketitle

\section*{Continuous Random Variables}
\lectlink{https://imperial.cloud.panopto.eu/Panopto/Pages/Viewer.aspx?id=1f9edcbf-e281-4147-aa01-ae250144e653}
For continuous random variables we want to track quantities in $\mathbb{R}$ (e.g temperature, volume, other probabilities).
\sidenote{Induced Probability Terms}{
	\[S_x = \{s \in S | X(s) \leq x\}\]
	\[P_X(X \leq x) = P(S_x) = F_X(x)\]
	$S_x$ is the elements of the sample space up to and including $x$. Hence the probability of getting $S_x$ is the cumulative probability.
	\centerimage{width=0.5\textwidth}{cumulative sx.png}
}
\termdef{Probability Density Function}{
	For a random variable $X : S \to \mathbb{R}$ the induced probability is defined as:
	\[P_X((-\infty, x]) = P(S_X) = F_X(x)\]
	A variable $X$ is \keyword{absolutely continuous} if $\exists f_X : \mathbb{R} \to \mathbb{R}$ such that:
	\[F_X(x) = \int_{u = -\infty}^xf_X(u)du\]
	\[f_x(x) = F'(x) = \cfrac{d}{dx}F_X(x)\]
	Where $f_X$ is the \keyword{probability density function} (\keyword{pdf}).
	\\
	\\ To find probability that $X \in (a,b]$:
	\[P_X(a < X \leq b) = P_X(X \leq b) - P_X(X \leq a) = F_X(b) - F_X(a) = \int_a^bf_X(x)dx\]
	\centerimage{width=0.7\textwidth}{probability density function.png}
	\compitem{
		\item We can use $<$ and $\leq$ interchangeably as $P(X = x) = 0 \Leftrightarrow P(X \leq x) \equiv P(X < x)$.
		\item Probability of any event is zero: $P_X(X = y) = 0$, any elementary event $\{x\}$ where $x \in \mathbb{R}$ has zero probability.
		\item However the sum of a range of events probabilities is not zero.
		\item Hence the range of a continuous random variable is uncountable (i.e as $\mathbb{R}$ is also).
	}
	\[\forall x \in \mathbb{R}. f_X(x)>= 0 \ \text{ and } \ \int_{-\infty}^{\infty}f_X(x)d_x = 1\]
}
\example{Defining a continuous random variable}{
	Given some continuous random variable $x$ with a probability density function given as:
	\[f(x) = \begin{cases}
			cx^2 & 0 < x < 3 \\
			0    & otherwise \\
		\end{cases}\]
	For some unknown constant $c$
	\\
	\\ To find the value of $c$ we use the requirement that the cumulative distribution must sum to $1$:
	\[\int_0^3 cx^2 = 1 \leadsto [\cfrac{cx^3}{3}]^3_0 = 1 \leadsto (9c) - 0 = 1 \leadsto c = \sfrac{1}{9}\]
	Hence:
	\[f(x) = \begin{cases}
			\cfrac{x^2}{9} & 0 < x < 3 \\
			0              & otherwise \\
		\end{cases}\]
	Hence we can specify the cumulative probability distribution as:
	\[F(x) = \begin{cases}
			0               & x \leq 0  \\
			\cfrac{x^3}{27} & 0 < x < 3 \\
			1               & x \geq 0  \\
		\end{cases}\]
	We can then calculate probabilities using the cumulative distribution:
	\[P(1 < X < 2) = F(2) - F(1) = \cfrac{2^3}{27} - \cfrac{1^3}{27} = \cfrac{7}{27} \approx 0.259\]
}

\section*{Mean, Variance and Quantiles}
\lectlink{https://imperial.cloud.panopto.eu/Panopto/Pages/Viewer.aspx?id=5f81f775-460e-4a02-9f6e-ae250144e688}
\termdef{Expected (Continuous)}{
	The \keyword{mean} or \keyword{expected} of a continuous random variable $X$:
	\[\mu_X = E_X(X) = \int_{-\infty}^{\infty}xf_X(x)dx\]
	For a function of interest that is applied to the random variable $g : \mathbb{R} \to \mathbb{R}$:
	\[E_X(g(X))  \int_{-\infty}^{\infty}g(x)f_X(x)dx\]
	\compitem {
		\item $E(aX + b) = aE(X) + b$
		\item $E(g(X) + h(X)) = E(g(X)) + E(h(X))$
	}
}
\termdef{Variance (Continuous)}{
	The variance of a continuous random variable $X$:
	\[\sigma^2_X = Var_X(X) = E((X - \mu_X)^2) = \int_{-\infty}^{\infty}(x - \mu_X)^2f_X(x)dx\]
	We can show this as:
	\[
		\begin{split}
			Var_X(X) & = \int_{-\infty}^{\infty}x^2f_X(x)ds - \mu_X^2 \\
			& = E(X^2) - (E(X))^2
		\end{split}\]
	For a linear transformation:
	\[Var(aX + b) = a^2Var(X)\]
}
\termdef{Quartiles} {
The lower, upper quartiles and median are points
\\
\\ For a continuous random variable $X$, we define the \keyword{$\alpha$-Quantile} $Q_X(\alpha)$ where $0 \leq \alpha \leq 1$ as the lowest $X$ such that:
\[P(X \leq Q_X(\alpha)) = \alpha \ \text{ or in other words } \ Q_X(\alpha) = F_X^{-1}(\alpha)\]
\centerimage{width=0.9\textwidth}{quantiles.png}
Using $Q_X$ we can define some standard quantiles:
\compitem{
	\bullpara{Quartiles}{Lower Quartile ($\alpha = \sfrac{1}{4}$), Median ($\alpha = \sfrac{1}{2}$) and Upper Quartile ($\alpha = \sfrac{3}{4}$)}
	\bullpara{Percentiles}{The $n$th percentile: $\alpha = \cfrac{n}{100}$}
}
}
\example{Basic continuous random variable}{
	Given continuous random variable $X$:
	\[f(x) = \begin{cases}
			\cfrac{x^2}{9} & 0 < x < 3 \\
			0              & otherwise \\
		\end{cases}\]
	We can calculate the expected:
	\[\begin{split}
			E(X) & = \int_{-\infty}^{\infty}xf(x)dx \\
			& = \int_{-\infty}^{0}xf(x)dx + \int_{0}^{3}xf(x)dx + \int_{3}^{\infty}xf(x)dx \\
			& = \int_{-\infty}^{0}x \times 0 dx + \int_{0}^{3}xf(x)dx + \int_{3}^{\infty}x \times 0 dx \\
			& = \int_{0}^{3}xf(x)dx = \int_{0}^{3}\cfrac{x^3}{9}dx = \begin{bmatrix}
				\cfrac{x^4}{36}
			\end{bmatrix}^3_0 \\
			& = \cfrac{9}{4} = 2.25\\
		\end{split}\]
	We can calculate the variance:
	\[\begin{split}
			Var(X) & = \int_{-\infty}^{\infty}x^2f(x)dx - \mu_X^2 \\
			& = \int_{-\infty}^{0}x^2f_(x)dx + \int_{0}^{3}x^2f(x)dx + \int_{3}^{\infty}x^2f(x)dx - \mu_X^2 \\
			& = \int_{0}^{3}x^2f(x)dx - \mu_X^2 = \int_{0}^{3}\cfrac{x^5}{9}dx - \mu_X^2\\
			& = 27 - \mu_X^2 = 27 - 2.25 = 24.75 \\
		\end{split}\]
	we can calculate the median, we ignore the range $x > 3$ as the median must be below this.
	\[\begin{split}
			0.5 &= \int_{-\infty}^{x}f(y)dy = \int_{-\infty}^{0}f(y)dy + \int_{0}^{x}f(y)dy = \int_{0}^{x}f(y)dy\\
			0.5 &= \int_{0}^{x}\cfrac{y^2}{9} = \begin{bmatrix}
				\cfrac{y^3}{27}
			\end{bmatrix}_0^x = \cfrac{x^3}{27}\\
			x &= \sqrt[3]{0.5 \times 27} \approx 2.38 \\
		\end{split}\]
}
\section*{Notable Continuous Distributions}
\lectlink{https://imperial.cloud.panopto.eu/Panopto/Pages/Viewer.aspx?id=aa5f8101-f025-4df8-b6d2-ae250144e6ab}
\termdef{Continuous Uniform Distribution}{
	A continuous random variable with equal probability of being any value within a range:
	\\
	\\ For $X \thicksim U(a,b)$:
	\begin{center}
		\begin{tabular}{c | c | c | c}
			\textbf{PDF}                         & \textbf{CDF}                          & \textbf{Expected}      & \textbf{Variance}                \\
			$f_X(x) = \begin{cases}
					\cfrac{1}{b - a} & a < x < b \\
					0                & otherwise \\
				\end{cases}$ & $F_X(x) = \begin{cases}
					0                    & x \leq a  \\
					\cfrac{x - a}{b - a} & a < x < b \\
					1                    & x \geq b  \\
				\end{cases}$ & $\mu = \cfrac{a+b}{2}$ & $\sigma^2 = \cfrac{(b-a)^2}{12}$ \\
		\end{tabular}
	\end{center}
	The standard uniform distribution is defined as $X \thicksim U(0,1)$:
	\begin{center}
		\begin{tabular}{c | c | c | c}
			\textbf{PDF}                          & \textbf{CDF}                          & \textbf{Expected}    & \textbf{Variance}          \\
			$f_X(x) = \begin{cases}
					1 & 0 < x < 1 \\
					0 & otherwise \\
				\end{cases}$ & $F_X(x) = \begin{cases}
					0 & x \leq a  \\
					x & a < x < b \\
					1 & x \geq b  \\
				\end{cases}$ & $\mu = \sfrac{1}{2}$ & $\sigma^2 = \sfrac{1}{12}$ \\
		\end{tabular}
	\end{center}
	Other uniform distributions can be mapped linearly to the standard uniform.
	\example{Mapping to Standard Uniform}{
		Given $X \thicksim U(2,5)$ find the expected, variance and median.
		\\
		\\ Take $Y \thicksim U(0,1)$, $X = 3 \times Y + 2$.
		\begin{center}
			\begin{tabular}{l c c c}
				\textbf{Distribution} & \textbf{Expected} & \textbf{Variance} & \textbf{Median} \\
				$Y$                   & $0.5$             & $\sfrac{1}{12}$   & $0.5$           \\
				$X$                   & $3.5$             & $\sfrac{3}{4}$    & $3.5$           \\
			\end{tabular}
		\end{center}
	}
}
\termdef{Exponential Distribution}{
	Given a rate of events $\lambda$, what is the probability of waiting $X$ time for the event to occur.
	\\
	\\ For $X \thicksim Exponential(\lambda)$ or $X \thicksim Exp(\lambda)$ where $\lambda > 0$:
	\begin{center}
		\begin{tabular}{c | c | c | c}
			\textbf{PDF}                                        & \textbf{CDF}                                    & \textbf{Expected}            & \textbf{Variance}                 \\
			$f_X(x) = \lambda e^{- \lambda x}$ where $x \geq 0$ & $F_X(x) = 1 - e^{- \lambda x}$ where $x \geq 0$ & $\mu_X = \cfrac{1}{\lambda}$ & $\sigma^2 = \cfrac{1}{\lambda^2}$ \\
		\end{tabular}
	\end{center}
	The distribution has the \keyword{Lack of memory property}, namely the time waited already does not affect the next part of the distribution (same shape).
	\[P(X > x + t | X > t) = \cfrac{P(X > x + t \cap X > t)}{P(X > t)} = \cfrac{P(X > x + t)}{P(X > t)} = \cfrac{e^{-\lambda(x+t)}}{e^{-\lambda t}} = e^{-\lambda x} = P(X > x)\]
	\[P(X > x + t | X > t) =  P(X > x)\]
	This distribution can be combined with Poisson. Given $X \thicksim Poisson(\lambda)$ (events occurring in a given time frame), the time between events is modelled by $X \thicksim Exponential(\lambda)$ (interval time for one event).
	\\
	\\ There is a variant with $\theta$ as the parameter for the distribution where $\theta = \cfrac{1}{\lambda}$.
}
\termdef{Normal Distribution}{
	Given a mean value ($\mu$) and a variance ($\sigma^2$) from the mean the symmetrical distribution is a \keyword{Normal Distribution}.
	\\
	\\ For $X \thicksim Normal(\mu, \sigma^2)$ or $X \thicksim N(\mu, \sigma^2)$ where $\sigma > 0$:
	\begin{center}
		\begin{tabular}{c | c }
			\textbf{PDF}                                                            & \textbf{CDF}                                                                                \\
			$f_X(x) = \cfrac{1}{\sigma \sqrt{2 \pi}}exp \begin{Bmatrix}
					-\cfrac{(x - \mu)^2}{2\sigma^2} \\
				\end{Bmatrix}$ & $F_X(x) = \cfrac{1}{\sigma \sqrt{2 \pi}} \int_{-\infty}^x exp\begin{Bmatrix}
					- \cfrac{(t - \mu)^2}{2 \sigma^2}
				\end{Bmatrix} dt$ \\
		\end{tabular}
	\end{center}
	The \keyword{Standard/Unit Normal Distribution} is $X \thicksim N(0, 1)$:
	\begin{center}
		\begin{tabular}{c | c }
			\textbf{PDF}                                                      & \textbf{CDF}                                                                \\
			$\phi(x) = \cfrac{1}{\sqrt{2 \pi}}exp \begin{Bmatrix}
					- \cfrac{1}{2}x^2
				\end{Bmatrix}$ & $\Phi(x) =  \cfrac{1}{\sqrt{2 \pi}} \int_{-\infty}^xe^{-\cfrac{t^2}{2}} dt$ \\
		\end{tabular}
	\end{center}
	We can apply linear functions:
	\[X \thicksim N(\mu, \sigma^2) \rightarrow \text{ and } aX + b \thicksim N(a \mu + b, a^2\sigma^2)\]
	Hence we can use the \keyword{Standard Normal Distribution}:
	\[X \thicksim N{\mu, \sigma^2} \Rightarrow \cfrac{X - \mu}{\sigma} \thicksim N(0,1) \text{ and hence } P(X \leq x) = \Phi(\cfrac{x - \mu}{\sigma})\]
}
\termdef{Lognormal Distribution}{
	Given $X \thicksim N(\mu, \sigma^2)$ and $Y = e^X$ we can compute the \keyword{PDF} of $Y$:
	\[f_Y(y) = \cfrac{1}{\sigma y \sqrt{2 \pi}} exp \begin{bmatrix}
			- \cfrac{(\log y - \mu)^2}{2 \sigma^2}
		\end{bmatrix}\]
}
\section*{Central Limit Theorem}
\lectlink{https://imperial.cloud.panopto.eu/Panopto/Pages/Viewer.aspx?id=650275b8-77fd-4c2a-9c98-ae250144e6f3}
\termdef{Moment Generating Function}{
The moment generating function $M_X$ for a continuous random variable $X$ is:
\[M_X(t) = E(e^{tX}) = \int_{-\infty}^{\infty}e^{tx}f_X(x)dx\]
Assuming the calculus within the $E(\dots)$ is valid, the $n$th moment is given by:
\[E[X^n] = \left.\cfrac{d^nM_x(t)}{dt^n}\right|_{t=0}\]
If the integral does not exist, the \keyword{characteristic function} $\phi_X(t) = M_X(\iota t)$ can be used ($\iota$ is imaginary unit).
}

\example{Expected and Variance}{
	\[\begin{split}
			E[X] & = \left.\cfrac{dM_x(t)}{dt}\right|_{t=0} \\
			& = \left.\cfrac{dE[e^{tX}]}{dt}\right|_{t=0} \\
			& = \left.\cfrac{d \int_{-\infty}^{\infty}e^{tx}f_X(x)dx}{dt}\right|_{t=0} \\
			& = \left.\int_{-\infty}^{\infty}xe^{tx}f_X(x)dx\right|_{t=0} \\
			& = \int_{-\infty}^{\infty}xe^{0x}f_X(x)dx \\
			& = \int_{-\infty}^{\infty} x f_X(x) dx\\
		\end{split}\]

	\[\begin{split}
			E[X^2] & = \left.\cfrac{d^2M_x(t)}{dt^2}\right|_{t=0} \\
			& = \left.\cfrac{d^2E(e^{tX})}{dt^2}\right|_{t=0} \\
			& = \left.\cfrac{d^2 \int_{-\infty}^{\infty}e^{tx}f_X(x)dx}{dt^2}\right|_{t=0} \\
			& = \left.\cfrac{d \int_{-\infty}^{\infty}xe^{tx}f_X(x)dx}{dt}\right|_{t=0} \\
			& = \left.\int_{-\infty}^{\infty}x^2e^{tx}f_X(x)dx\right|_{t=0} \\
			& = \int_{-\infty}^{\infty}x^2e^{0x}f_X(x)dx \\
			& = \int_{-\infty}^{\infty}x^2f_X(x)dx \\
		\end{split}\]

	\[Var[X] =  E[X^2] - (E[X])^2\]
}

\subsection*{Product of Random Variables}
Given independent random variables $Z_1, Z_2, \dots, Z_n$:
\[E[\prod_{i=1}^nZ_i] = \prod_{i=1}^nE[Z_i]\]
The sum of the random variables is the products of their \keyword{Moment Generating Functions}.
\[M_{Z_1 + Z_2}(t) = E[e^{t(Z_1 + Z_2)}] = E[e^{tZ_1}e^{tZ_2}] = E[e^{tZ_1}]E[e^{tZ_2}] = M_{Z_1}(t)M_{Z_2}(t)\]
\[S_n = \sum_{i=1}^nZ_i \Rightarrow M_{S_n}(t) = \prod_{j=1}^nM{X_j}(t)\]

\subsection*{Central Limit Theorem}
\termdef{Central Limit Theorem}{
	Given $X_1, X_2, \dots, X_n$ are independent and identically distributed random variables from any distribution with mean $\mu$ and finite variance $\sigma^2$.
	\[S_n = \sum_{i=1}^nX_i\]
	Hence we have a distribution with a known expected and variance, so can form a \keyword{Normal Distribution}.
	\[\begin{matrix}
			Y = S_n                                 & E(Y) = n \mu & Var(Y) = n \sigma^2 \\
			Y = S_n - n \mu                         & E(Y) = 0     & Var(Y) = n \sigma^2 \\
			Y = \cfrac{S_n - n \mu}{\sqrt{n}\sigma} & E(X) = 0     & Var(X) = 1          \\
		\end{matrix}\]
	$Y$ can now be used to approximate a \keyword{Standard Normal Distribution}.
	\[\lim_{n \to \infty}\cfrac{S_n - n \mu}{\sqrt{n}\sigma} \thicksim N(0,1)\]
	This implies that for large (but finite n):
	\[\overline{X} \approx N(\mu, \cfrac{\sigma^2}{n}) \ \text{ and } \ \sum_{i=1}^n X_i \approx N(n \mu, n \sigma^2)\]
	Where $\overline{X}$ is the average value of the random variables $\cfrac{\sum_{i=0}^n X_i}{n}$.
	\\
	\\ The approximation holds for all distributions (including discrete), and is exact when the random variables are from the same \keyword{normal distribution}.
}
\subsubsection*{An attempt at CLT proof}
Given the random variables $X_1, X_2, \dots, X_n$ we can standardize and get their sum:
\[Z_n = \cfrac{S_n - n \mu}{\sqrt{n} \sigma} = \cfrac{\sum_{i=1}^nX_i - n\mu}{\sqrt{n} \sigma} = \cfrac{\sum_{i=1}^n(X_i - \mu)}{\sqrt{n} \sigma} = \sum_{i=1}^n \cfrac{Y_i}{\sqrt{n}\sigma} \ \text{ where } \ Y_i = X_i - \mu\]
The moment generating function of $Z_n$ is the product of the \keyword{moment generating functions} of the $Y$ (all identically distributed, so identical \keyword{MGFs}).
\[M_{Z_n}(t) = \begin{pmatrix}
		M_Y \begin{pmatrix}
			\cfrac{t}{\sqrt{n}\sigma}
		\end{pmatrix}
	\end{pmatrix} \ \text{ where } M_y \text{ is the moment generating function for all } Y_i\]
We can then expand the $M_Y$ around 0 using Taylor's Theorem:
\[M_Y(t) = M_Y(0) + M'_Y(0)t + \sfrac{1}{2}M''_Y(0)t^2 + O(t^3)\]
$O(t^3)$ is the error term of our approximation, as this is for higher powers, it has a small effect so can be ignored
\\
\\ The derivatives of the \keyword{MFG} are:
\[M'_Y(0) = E(Y_i) = 0 \text{ due to shift performed earlier and } M''_Y(0) = E(Y_i^2) = \sigma^2 + E(Y_i)^2 = = \sigma^2 + 0 = \sigma^2\]
Hence we can derive:
\[M_Y(t) = 1 + \cfrac{\sigma^2t^2}{2} + O(t^3)\]
Hence we can scale $t$, and ignore the error term for simplicity:
\[M_Y \begin{pmatrix}
		\cfrac{t}{\sqrt{n}\sigma}
	\end{pmatrix} = 1 + \cfrac{t^2}{2n}\]
As the error term gets very small, we can use limits to get an approximation for $M_{Z_n}(t)$.
\[lim_{n \to \infty} M_{Z_n}(t) = \lim_{n \to \infty}(1 + \cfrac{t^2}{2n} + O(n^{-\sfrac{3}{2}}))^n = e^{\sfrac{t^2}{2}}\]
Note that $\lim_{m \to \infty}(1 + \cfrac{x}{m})^m = e^x$.
\example{Coin Tossing}{
	Consider a set of count tosses, each are Bernoulli discrete random variables (take values 0 or 1).
	\[X_1, X_2, X_3, \dots, X_n \text{ where } \mu = p \text{ and } \sigma^2 = p(1-p)\]
	The total score of toin tosses can be modelled as a binomail distribution:
	\[\sum_{i=1}^nX_i \text{ is } X \thicksim Binomial(n,p) \text{ with } E(X) = np \text{ and } Var(X) = np(1-p)\]
	For large $n$ can also model it as a normal distribution:
	\[\sum_{i=1}^nX_i \text{ is } X \thicksim N(n \mu, n \sigma^2) \equiv N(np, n p(1-p)) \]
	As the number of events (coin tosses) tends to infinity, the distributions tend to look identical.
}

\end{document}
