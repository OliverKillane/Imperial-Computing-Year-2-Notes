\documentclass{report}
    \title{50008 - Probability and Statistics - Lecture 2}
    \author{Oliver Killane}
    \date{17/01/22}
%===========================COMMON FORMAT & COMMANDS===========================
% This file contains commands and format to be used by every module, and is 
% included in all files.
%===============================================================================

%====================================IMPORTS====================================
\usepackage[a4paper, total={6in, 8in}]{geometry}
\usepackage{graphicx, amssymb, amsfonts, amsmath, xcolor, listings, tcolorbox, multirow, hyperref}
%===============================================================================

%====================================IMAGES=====================================
\graphicspath{{image/}}

% \centerimage{options}{image}
\newcommand{\centerimage}[2]{\begin{center}
    \includegraphics[#1]{#2}
\end{center}}
%===============================================================================

%=================================CODE LISTINGS=================================
\definecolor{codebackdrop}{gray}{0.9}
\definecolor{commentgreen}{rgb}{0,0.6,0}
\lstset{
    inputpath=code, 
    commentstyle=\color{commentgreen},
    keywordstyle=\color{blue}, 
    backgroundcolor=\color{codebackdrop}, 
    basicstyle=\footnotesize,
    frame=single,
    numbers=left,
    stepnumber=1,
    showstringspaces=false,
    breaklines=true,
    postbreak=\mbox{\textcolor{red}{$\hookrightarrow$}\space}
}

% Create a code listing for a single line
% \codeline{language}{line}{file}
\newcommand{\codeline}[3]{\lstinputlisting[language=#1, firstline = #2, lastline = #2]{#3}}

% Create a code listing for a given language & file
% \codelist{language}{file}
\newcommand{\codelist}[2]{\lstinputlisting[language=#1]{#2}}
%===============================================================================

%================================TEXT STRUCTURES================================
% Marka a word as bold
% \keyword{important word}
\newcommand{\keyword}[1]{\textbf{#1}}

% Creates a section in italics
% \question{question in italics}
\newcommand{\question}[1]{\textit{#1} \\ }

% Creates a box with title for side notes.
% \sidenote{title}{contents}
\newcommand{\sidenote}[2]{\begin{tcolorbox}[title=#1]#2\end{tcolorbox}}

% Creates an item in an itemize or enumerate, with a paragraph after
% \begin{itemize}
%     \bullpara{title}{contents}
% \end{itemize}
\newcommand{\bullpara}[2]{\item \textbf{#1} \ #2}

% Creates a compact list (very small gaps between items)
% \compitem{
%     \item item 1
%     \item item 2
%     \item ...
% }
\newcommand{\compitem}[1]{\begin{itemize}\setlength\itemsep{-0.5em}#1\end{itemize}}

% Creates a link to the lecture for use at the start of the notes document
\newcommand{\lectlink}[1]{\sidenote{Lecture Recording}{
    Lecture recording is available \href{#1}{here}
}}
%===============================================================================

\begin{document}
\maketitle

\section*{Introduction}
\lectlink{https://imperial.cloud.panopto.eu/Panopto/Pages/Viewer.aspx?id=91891a82-a5fd-4ad8-8889-ae20006c07de}
\termdef{Probability Space}{
	\[(S, \mathcal{F}, P )\]
	Models a random experiment where probability measure $P(E)$ is defined on subsets $E \subseteq S$ belonging to sigma algebra $\mathcal{F}$.
}

Within a sample space we can study quantities that are a function of randomly occurring events (e.g temperature, exchange rates, gambling scores).

\termdef{Random Variable}{
	A \keyword{random variable} is a mapping from the sample space to the real numbers, for example \keyword{random variable} $X$:
	\[X: S \to \mathbb{R}\]
	Each element in the sample space $s \in S$ is assigned to a numerical value by $X(s)$.
	\\
	\\ When referring to the value of a random variable we use its name, e.g $X$ in $P(5 < X \leq 30)$
	\compitem {
		\bullpara{Simple}{Finite set of possible outcomes. (e.g dice faces)}
		\bullpara{Discrete}{Countable outcomes/support/range. (e.g distance (m))}
		\bullpara{Continuous}{Can be a continuous range (e.g temp)}
	}
}

\example{Single Fair Dice Roll}{
	$S = \{ 1,2,3,4,5,6 \}$, for any $s \in S. P( \{s \} ) = \cfrac{1}{6}$.
	\\ We can define random variable $X$ such that:
	\[X(1) = 1, X(2) = 2, X(3) = 3, X(4) = 4, X(5) = 5, X(6) = 6\]
	Then we can use $X$:
	\[P_X(1 < X \leq 5) = P(\{2,3,4,5\}) = \sfrac{2}{3}\]
	\[P_X(X \in \{2,3\}) = P(\{2,3\}) = \sfrac{1}{3}\]
	We can also define random variable $Y$ such that:
	\[Y(\epsilon) = \begin{cases}
			0 & \epsilon \text{ is odd}  \\
			1 & \epsilon \text{ is even} \\
		\end{cases}\]
	And hence:
	\[P_Y(Y = 0) = P(\{1,3,5\}) = \sfrac{1}{2}\]
}

\subsection*{Induced Probability}
The probability measure $P$ defined on a sample space $S$ induces a probability distribution on the random variable in $\mathbb{R}$ (distribution of its outcomes).
\[S_X = \{s \in S | X(s) \leq x\}\]
Such that:
\[P_X(X \geq x) = P(S_X)\]
Note that unless there is ambiguity, $P_X(\dots)$ will often be written as $P(\dots)$.
\example{Heads and Tails}{
	We define random variable $X : \{H,T\} \to \mathbb{R}$  over the \keyword{continuum} $\mathbb{R}$ such that:
	\[X(T) = 0 \text{ and } X(H) = 1\]
	\[S_X = \begin{cases}
			\emptyset & \text{if }x < 0         \\
			\{T\}     & \text{if } 0 \leq x < 1 \\
			\{H,T\}   & \text{if } x \geq 1
		\end{cases}\]
	X represents the number of heads flipped.
	\[P_X(X \leq x) = P(S_X) = \begin{cases}
			P(\emptyset) = 0        & \text{if } x < 0        \\
			P(\{T\}) = \sfrac{1}{2} & \text{if } 0 \leq x < 1 \\
			P(\{H,T\}) = 1          & \text{if } x \geq 1     \\
		\end{cases}\]
	Now we can use $X$ to compactly show probabilities.
	\[P_X(X =1) = \sfrac{1}{2}\]
}

\example{Multiple Coin Flips}{
	\[S = \{TTT,TTH,THT,HTT,THH,HHT,HTH,HHH\}\]
	We can define $X$ (number of heads):
	\[X(s) = \begin{cases}
			0 & s = TTT               \\
			1 & s \in \{TTH,THT,HTT\} \\
			2 & s \in \{THH,HHT,HTH\} \\
			3 & s = HHH               \\
		\end{cases}\]
	Hence given 3 coin tosses:
	\begin{center}
		\begin{tabular}{c l}
			$P_X(X > 1)$    & More than one head \\
			$P_X(X < 3)$    & Not all heads      \\
			$P_X(X \leq 1)$ & At least one head  \\
		\end{tabular}
	\end{center}
}

\termdef{Support/Range}{
	The set of all possible values of a random variable $X$:
	\[\mathbb{X} \equiv supp(X) \equiv X(S) = \{x \in \mathbb{R} | \exists s \in S . X(s) = x\}\]
	As $S$ contains all possible experiment outcomes, $supp(X)$ contains all possible values/outcomes for the random variables $X$.
	\[P_X(X \leq x) \text{ is defined for all } x \in supp(X)\]
}

\subsection*{Cumulative Distributions}
\termdef{Cumulative Distribution Function ($F_X$)}{
	The cumulative distribution function (cfd) of a random variable $X$ is the probability where X takes some value less than or equal to some $x$:
	\[F_X : \mathbb{R} \to [0,1] \text{ such that } F_X(x) = P_x(X \leq x)\]
	To be a valid cfd, 3 criteria must be met:
	\compenum{
		\bullpara{Probability between 0 and 1}{$\forall x \in \mathbb{R}.0 \leq F_X(x) \leq 1$}
		\bullpara{Monotonicity}{$\forall x_1,x_2 \in \mathbb{R} x_1 < x_2 \Rightarrow F_X(x_1) \leq F_X(x_2)$}
		\bullpara{Infinite Bounds}{$F_X(-\infty) = 0, F_X(\infty) = 1$}
	}
	For any random variable a \keyword{cfd} is right-continuous (a result of monotonicity).
	\[x_1 > x_2 > x_3 ... > x \Rightarrow F_X(x_1) >= F_X(x_2) >= ... >= F_X(x)\]
}

We can determine the probability over finite intervals using the cumulative distribution:
\[\text{for } (a,b] \subseteq \mathbb{R} \ P_X(a < X \leq b) = F_X(b) - F_X(a)\]

\section*{Distributions}
\lectlink{https://imperial.cloud.panopto.eu/Panopto/Pages/Viewer.aspx?id=fb3d1868-a486-4142-823f-ae20006c073c}

\termdef{Probability Mass Function ($p_X$)}{
	Also called \keyword{probability function} gives the probability that a discrete random variable is exactly equal to a value.
	\\
	\\ The sample space $S$ is mapped onto elements in the \keyword{support} of $X$ (one-to-one).
	\\
	\\ We can then partition the sample space into a countable, disjoint collection od event subsets:
	\[s \in E_i \Leftrightarrow X(s) = x_i, i = 1,2 \dots\]

	A probability mass function is valid if and only if:
	\compenum{
		\bullpara{No negative probabilities}{$\forall x \in supp(X). \ p_X(x) \geq 0$}
		\bullpara{Probabilities sum to 1}{$\sum_{x \in supp(x)}p_X(x) = 1$}
	}
}

\subsection*{Discrete Random Variable}
For a \keyword{discrete random variable} we define the probability mass function as:
\[p_X(x_i) = P(X = x_i) = P(E_i) \text{ where } x_i \in supp(X) \text{ and } x_i \text{ is the outcome of event } E_i\]
We can also define using \keyword{cfds}:
\[F_X(x_i) = \sum^{i}_{j = 1}p_X(x_j) \Leftrightarrow p_X(x_i) = F_X(x_i) - F_X(x_{i-1}) \ \text{ where } i = 2, 3 \dots\]
Or more simply:
\[p_X(x_i) = P_X(X = x_i) = P(X \leq x_i) - P(X \leq x_{i-1}) = F_X(x_i) - F_X(x_{i-1})\]
When graphed, $F_X$ is a monotonically increasing, stepped function with jumps at points in $S(X)$.
\example{Six Sided Dice}{
	Here we have $X$ representing the value of the dice roll. We can plot the cumulative distribution (showing probability a dice roll is less than or equal to a given value).
	\centerimage{width=0.8\textwidth}{Cumulative distribution.png}
}

Discrete CFDs have several properties:
\compitem {
	\bullpara{Limiting Cases}{
		\[\lim_{x \to -\infty}F_X(x) = 0 \ \ \lim_{x \to \infty} F_X(x) = 1\]
		At $\infty$ the whole set of outcomes is covered, probabilities sum to 1. At $-\infty$ none are covered.
	}
	\bullpara{Continuous from the right}{
		\[\text{For } x \in \mathbb{R} \lim_{h \to 0^+}F_X(x + h) = F_X(x)\]
		Moving from the right to the left the probability will reduce and tend towards the value.
	}
	\bullpara{Non-Decreasing}{
		\[a < b \Rightarrow F_X(a) \leq F_X(b)\]
		As it is cumulative, the value can only grow larger moving right.
	}
	\bullpara{Can cover a range}{
		\[\text{For }a < b. \ P(a < X \leq b) = F_X(b) - F_X(a)\]
	}
}

\termdef{Poisson Distribution}{
A discrete probability distribution expressing the probability of a given number of events occuring in a fixed time interval, given a constant mean.
\[Pois(\lambda) = \cfrac{\lambda^ke^{-\lambda}}{k!} \ \text{ where } k \text{ is the number of occurrences}\]
e.g What is the probability exactly 7 people buy pizzas at a stall in one hour, given on average is 4 people per hour?
\[X \approx Poisson(4)\]
For a poisson distribution the mean (expected) and variance are equal.
\[E(X) = Var(X)\]
\[P(X = 7) = \cfrac{4^7e^{-4}}{7!}\]
\centerimage{width=0.6\textwidth}{poisson pmf.png}
\centerimage{width=0.6\textwidth}{poisson cdf.png}
}

\subsection*{Link with Statistics}
We can consider a set of data as realisations of a random variable defined on some underlying population of the data.
\compitem{
	\item Frequency histogram is an empirical estimate for the \keyword{pmf}.
	\item Cumulative histogram is an empirical estimate of the \keyword{cdf}.
}

\subsection*{Expectation}
\termdef{Expected Value}{
	The expectation of a \keyword{discrete random variable} $X$ is:
	\[E_X(X) = \sum_{x}xp(x)\]
	Also referred to as $\mu_X$ it is the mean value of the distribution.
	\[E(g(X)) = \sum_xg(x)p_X(x)\]
	\[E(a \times X + b) = a \times E(X) + b\]
	\[E(a \times g(X) + b \times f(X)) = a \times E(g(X)) + b \times E(f(X))\]
	Given another distribution $Y$:
	\[E(X + Y) = E(X) + E(Y)\]
}
\example{Dice Rolls}{
	Given random variable $X$ representing the value of a dice roll:
	\[X(n) = n \text{ where } 1 \geq n \geq 6\]
	\[P(X = x) = \begin{cases}
			\sfrac{1}{6} & 1 \geq n \geq 6 \\
			0            & otherwise       \\
		\end{cases}\]
	We can get the expected as:
	\[E(X) = \sfrac{1}{6} \times 1 + \sfrac{1}{6} \times 2 + \sfrac{1}{6} \times 3 + \sfrac{1}{6} \times 4 + \sfrac{1}{6} \times 5 + \sfrac{1}{6} \times 6 = \sfrac{21}{6} = 3.5\]
	We can base scoring on the dice roll:
	\[score(x) = 4 \times x + 2\]
	Hence we can calculate that the expected score is $E(score(X)) = 4 \times 3.5 + 2 = 16$.
}

\example{Dice and Coins}{
	Given random variable $D$ of a fair dice, and fair coin $C$:
	\[P(D = x) = \begin{cases}
			\sfrac{1}{6} & 1 \geq n \geq 6 \\
			0            & otherwise       \\
		\end{cases} \ \text{ and } \ P(C = x) = \begin{cases}
			\sfrac{1}{2} & x \in \{H,T\} \\
			0            & otherwise     \\
		\end{cases}\]
	Given $score = dice \ roll + 1 \text{ if coin flip is heads}$ what is the expected score?
	\[E(D) = 3.5 \ \ E(C) = 0.5 \ \ E(score) = 3.5 + 2 * 0.5 = 4.5\]
}

\subsection*{Variance}
\lectlink{https://imperial.cloud.panopto.eu/Panopto/Pages/Viewer.aspx?id=9ae606e7-2887-440a-babd-ae20006c07a5}
\termdef{Moment}{
	A function which measures the shape of a function's graph.
	\\
	\\ The $n^{th}$ moment of a random variable is the expected value of its $n^{th}$ power:
	\[n^{th}\text{ moment of $X$} = \mu_X(n) = E(X^n) = \sum_xx^np(x)\]
	\compitem{
		\bullpara{First Moment}{The expected value.}
		\bullpara{Central Moment}{The variance ($E[(X - E(X))^2]$)}
		\bullpara{Standardized Moment}{The skew ($\cfrac{E(X - E(X))^3}{sd(X)^3}$)}
	}

}
\termdef{Variance}{
	The expected deviation from the expected/mean value.
	\[Var(X) = Var_X(X) = \sigma_X^2 = E[(X - E(X))^2] = E(X^2) - (E(X))^2\]
	Note that:
	\[Var(a \times X + b) = a^2Var(X)\]
}
\termdef{Standard Deviation}{
	The square root of the variance.
	\[\sigma_X = sd_X(X) = \sqrt{Var_X(X)}\]
}
\example{Dice Roll}{
	For a random variable representing a dice $X$:
	\[Var(X) = E(X^2) - (E(X^2)) = \sum_xx^2p(x) - (\sum_xxp(x))^2 = \sfrac{91}{6} - \sfrac{49}{4} = \sfrac{35}{12}\]
}
\termdef{Skewness}{
	A measure of asymmetry (the standardized moment):
	\[\gamma_1 = \cfrac{E(X - E(X))^3}{sd(X)^3} = \cfrac{E(X - \mu)}{\sigma^3} \text{ where } \mu = E(X), \sigma = Sd(X)\]
	\centerimage{width=0.8\textwidth}{skew.png}
}

\subsection*{Sum of Random Variables}
\lectlink{https://imperial.cloud.panopto.eu/Panopto/Pages/Viewer.aspx?id=0711a1ed-e922-4535-82b9-ae20006c077b}
Given random variables $X_1, X_2, \dots, X_n$ (not necessarily independent, and potentially from different distributions), the sum is:
\[\text{The sum }S_n = \sum_{i=1}^nX_i \ \text{ and the average is } \cfrac{S_n}{n}\]
(The sum of the outcomes from all random variables)
\\
\\ The expected/mean value of $S_n$ (expected value of the sum of all the random variables) is:
\[E(S_n) = \sum_{i=1}^nE(X_i) \ \text{ and } \ E(\cfrac{S_n}{n}) = \cfrac{\sum_{i=1}^nE(X_i)}{n}\]
\begin{itemize}
	\bullpara{All independent}{
		\[Var(S_n) = \sum_{i=1}^nVar(X_i) \ \text{ and } \ Var(\cfrac{S_n}{n}) = \cfrac{\sum_{i=1}^nVar(X_i)}{n^2}\]
	}
	\bullpara{All independent and Identically Distributed}
	\\ Given that for all $i$, $E(X_i) = \mu_X$ and $Var(X_i) = \sigma^2_X$:
	\[E(\cfrac{S_n}{n}) = \mu_X \ \text{ and } \ Var(\cfrac{S_n}{n}) = \cfrac{\sigma_X^2}{n}\]
\end{itemize}

\section*{Important Discrete Random Variables}
\lectlink{https://imperial.cloud.panopto.eu/Panopto/Pages/Viewer.aspx?id=6f6a5fe8-1229-44ab-a8f5-ae20006c23f0}
\termdef{Bernouli Distribution}{
	For an experiment with only two outcomes, encoded as $1$ and $0$.
	\\
	\\ For$X \thicksim Bernoulli(p)$ where $x \in S(X) = \{0,1\}$ and $0 \leq p \leq 1$:
	\begin{center}
		\begin{tabular}{c | c | c}
			\keyword{PMF}             & \keyword{Expected} & \keyword{Variance}           \\
			$p_X(x) = p^x(1-p)^{1-x}$ & $\mu = E(X) = p$   & $\sigma^2 = Var(X) = p(1-p)$ \\
		\end{tabular}
	\end{center}

}
\termdef{Binomial Distribution}{
	Given $n$ trials with two options, binomial models the number of outcomes. (e.g 3 coin tosses, number of ways to get 2 heads out of total outcomes).
	\\
	\\ For $X \thicksim Bionomial(n,p)$ where $X$ takes values $0,1,2, \dots, n$ and $0 \leq p \leq 1$:
	\begin{center}
		\begin{tabular}{c | c | c | c}
			\keyword{PMF}                          & \keyword{Expected} & \keyword{Variance}            & \keyword{Skewness}                          \\
			$p_X(x) = {n \choose x}p^x(1-p)^{n-x}$ & $\mu = E(X) = np$  & $\sigma^2 = Var(X) = np(1-p)$ & $\gamma_1 = \cfrac{1 - 2p}{\sqrt{np(1-p)}}$ \\
		\end{tabular}
	\end{center}

	Note that choice is: ${n \choose x} = \cfrac{n!}{x!(n-x)!}$
}
\termdef{Poisson Distribution}{
	Given a constant mean number of events per fixed itme interval, provides probabilities of different numbers of events occuring. (e.g sell on average 6 cookies an hour, what is the probability 10 cookies are sold in a given hour).
	\\
	\\ For $X \thicksim Poisson(\lambda)$ where $\lambda$ is the mean number of events and $\lambda > 0$:
	\begin{center}
		\begin{tabular}{c | c | c | c}
			\keyword{PMF}                                & \keyword{Expected}     & \keyword{Variance}            & \keyword{Skewness}                     \\
			$p_X(x) = \cfrac{e^{-\lambda}\lambda^x}{x!}$ & $\mu = E(X) = \lambda$ & $\sigma^2 = Var(X) = \lambda$ & $\gamma_1 = \cfrac{1}{\sqrt{\lambda}}$ \\
		\end{tabular}
	\end{center}
	Note that for poisson the skew is always positive (but decreases as $\lambda$ increases), and $E(X) \equiv Var(X)$.
}
\termdef{Geometric Distribution}{
	A potentially infinite number of trials to get an outcome (e.g attempts required to shoot a target, given probability of hit).
	\\
	\\ We can consider it infinite Bernoulli trials $X_1, X_2, \dots$, where $X = \{i | X_i = 1\}$ (X is number of attempts to get outcome $1$).
	\\
	\\ For $X \thicksim Geometric(p)$ where $X$ takes all values in $\mathbb{Z}^+ = \{1,2,\dots\}$ and $0 \leq p \leq 1$:
	\begin{center}
		\begin{tabular}{c | c | c | c}
			\keyword{PMF}             & \keyword{Expected}          & \keyword{Variance}                     & \keyword{Skewness}                   \\
			$p_X(x) = p(1-p)^{x - 1}$ & $\mu = E(X) = \cfrac{1}{p}$ & $\sigma^2 = Var(X) = \cfrac{1-p}{p^2}$ & $\gamma_1 = \cfrac{2-p}{\sqrt{1-p}}$ \\
		\end{tabular}
	\end{center}
	Alternatively we can consider the number of trials \textit{before} getting an outcome:
	\\ If $X \thicksim Geometric(P)$ consider $Y = X - 1$ where $Y$ takes values $\mathbb{N} = \{0,1,2,\dots\}$:
	\begin{center}
		\begin{tabular}{c | c | c | c}
			\keyword{PMF}         & \keyword{Expected}            & \keyword{Variance} & \keyword{Skewness} \\
			$p_Y(x) = p(1-p)^{y}$ & $\mu = E(Y) = \cfrac{1-p}{p}$ & Unchanged          & Unchanged          \\
		\end{tabular}
	\end{center}
}
\termdef{Discrete Uniform Distribution}{
	Where a discrete number of outcomes are equally likely (e.g fair dice, colour wheel).
	\\
	\\ For $X \thicksim U(\{1,2,\dots,n\})$:
	\begin{center}
		\begin{tabular}{c | c | c | c}
			\keyword{PMF}           & \keyword{Expected}            & \keyword{Variance}                        & \keyword{Skewness} \\
			$p_X(x) = \cfrac{1}{n}$ & $\mu = E(X) = \cfrac{n+1}{2}$ & $\sigma^2 = Var(X) = \cfrac{n^2 - 1}{12}$ & $\gamma_1 = 0$     \\
		\end{tabular}
	\end{center}
}
\subsection*{Poisson Limit Theorem}
We can use the \keyword{Binomial Distribution} to approximate the \keyword{Poisson Distribution}:
\[Poisson(\lambda) \approx Binomial(n,p) \text{ when } \lambda = np \text{ and } n \text{ is very large, } p \text{ is very small}\]
This is as for a \keyword{Poisson distribution} mean and variance are equal and for binomial, mean is $np$ and variance $np(1-p)$ so as $p$ gets smaller (and $n$ larger) $np \approx np(1-p)$.

\end{document}
