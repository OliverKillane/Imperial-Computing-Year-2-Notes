\documentclass{report}
    \title{50008 - Probability and Statistics - Lecture 1}
    \author{Oliver Killane}
    \date{11/01/22}
%===========================COMMON FORMAT & COMMANDS===========================
% This file contains commands and format to be used by every module, and is 
% included in all files.
%===============================================================================

%====================================IMPORTS====================================
\usepackage[a4paper, total={6in, 8in}]{geometry}
\usepackage{graphicx, amssymb, amsfonts, amsmath, xcolor, listings, tcolorbox, multirow, hyperref}
%===============================================================================

%====================================IMAGES=====================================
\graphicspath{{image/}}

% \centerimage{options}{image}
\newcommand{\centerimage}[2]{\begin{center}
    \includegraphics[#1]{#2}
\end{center}}
%===============================================================================

%=================================CODE LISTINGS=================================
\definecolor{codebackdrop}{gray}{0.9}
\definecolor{commentgreen}{rgb}{0,0.6,0}
\lstset{
    inputpath=code, 
    commentstyle=\color{commentgreen},
    keywordstyle=\color{blue}, 
    backgroundcolor=\color{codebackdrop}, 
    basicstyle=\footnotesize,
    frame=single,
    numbers=left,
    stepnumber=1,
    showstringspaces=false,
    breaklines=true,
    postbreak=\mbox{\textcolor{red}{$\hookrightarrow$}\space}
}

% Create a code listing for a single line
% \codeline{language}{line}{file}
\newcommand{\codeline}[3]{\lstinputlisting[language=#1, firstline = #2, lastline = #2]{#3}}

% Create a code listing for a given language & file
% \codelist{language}{file}
\newcommand{\codelist}[2]{\lstinputlisting[language=#1]{#2}}
%===============================================================================

%================================TEXT STRUCTURES================================
% Marka a word as bold
% \keyword{important word}
\newcommand{\keyword}[1]{\textbf{#1}}

% Creates a section in italics
% \question{question in italics}
\newcommand{\question}[1]{\textit{#1} \\ }

% Creates a box with title for side notes.
% \sidenote{title}{contents}
\newcommand{\sidenote}[2]{\begin{tcolorbox}[title=#1]#2\end{tcolorbox}}

% Creates an item in an itemize or enumerate, with a paragraph after
% \begin{itemize}
%     \bullpara{title}{contents}
% \end{itemize}
\newcommand{\bullpara}[2]{\item \textbf{#1} \ #2}

% Creates a compact list (very small gaps between items)
% \compitem{
%     \item item 1
%     \item item 2
%     \item ...
% }
\newcommand{\compitem}[1]{\begin{itemize}\setlength\itemsep{-0.5em}#1\end{itemize}}

% Creates a link to the lecture for use at the start of the notes document
\newcommand{\lectlink}[1]{\sidenote{Lecture Recording}{
    Lecture recording is available \href{#1}{here}
}}
%===============================================================================

\begin{document}
    \maketitle

    \section*{Elementary Probability Theory}
        \lectlink{https://imperial.cloud.panopto.eu/Panopto/Pages/Viewer.aspx?id=0804d63e-685c-49c5-9d1b-ae1800b0ba1d}

        Probability theory is a mathematical formalism to describe and quantify uncertainty.
        \\ 
        \\ Uses of probability include examples such as:
        \compitem {
            \item Finding distribution of runtimes \& memory usage for software.
            \item Response times for database queries.
            \item Failure rate of components in a datacenter.
        }

        \subsection*{Sample Spaces and Events}
            \lectlink{https://imperial.cloud.panopto.eu/Panopto/Pages/Viewer.aspx?id=b3d2d5a9-2820-498d-88b3-ae1800b0baa7}
            \termdef{Sample Space}{
                The set of all possible outcomes of a random experiment. The set is usually denoted with set notation, and can be finite, countably or uncountably infinite.
                \\
                \\ For example:
                \begin{center}
                    \begin{tabular}{l l}
                        \textbf{Experiment} & \textbf{Sample Space} \\
                        \hline
                        Coin Toss & $S = \{Heads, Tails\}$ \\
                        6-Sided Dice Roll & $S = \{1,2,3,4,5,6\}$ \\
                        2 Coin Tosses & $S = \{(H,H), (H,T), (T,H), (T,T)\}$ \\
                        Choice of Odd number & $S = \{ x \in \mathbb{N} | \exists y \in \mathbb{N}.[2y + 1 = x] \}$
                    \end{tabular}
                \end{center}
            }
            \termdef{Event}{
                Any subset of the sample space $E \subseteq S$ (a set of possible outcomes).
                \compitem{
                    \bullpara{null event($\emptyset$)}{Empty event, can be used for impossible events.}
                    \bullpara{universal event ($S$)}{Event contains entire sample space and is therefore certain.}
                    \bullpara{elementary events}{Singleton subsets of the sample space (contain one element).}
                }
                For example:
                \begin{center}
                    \begin{tabular}{l l l}
                        \textbf{Event} & \textbf{Set of Event} & \textbf{Sample Space} \\
                        \hline
                        6-Sided Dice Rolls 1 & $E = \{1\}$ & $S = \{1,2,3,4,5,6\}$ \\
                        6-Sided Dice Rolls Even & $E = \{2,4,6\}$ & $S = \{1,2,3,4,5,6\}$ \\
                        6-Sided Dice Rolls 7 & $E = \emptyset$ & $S = \{1,2,3,4,5,6\}$ \\
                        2 Coin toss get 2 Tails & $E = \{(T,T)\}$ & $S = \{(H,H), (H,T), (T,H), (T,T)\}$ \\
                        Random Natural Number is 4 & $E = \{4\}$ & $S = \mathbb{N}$ \\
                    \end{tabular}
                \end{center}
            }
            \centerimage{width=0.7\textwidth}{experiment.png}
            \compitem{
                \item If we perform a random experiment with outcome $S* \in S$. If $s* \in E$, then event $E$ has occurred.
                \item If $E$ has not occurred ($s* \not\in E$) then $s* \in \overline{E}$.
                \item The set $\{s*\}$ is an elementary event.
                \item Null event $\emptyset$ never occurs, the universal event $S$ always occurs.
            }
            \subsubsection*{Set Operations on Events}
                \begin{itemize}
                    \bullpara{Union / Or}{
                        \[\bigcup_iE_i = \{s \in S | \exists i .[s \in E_i] \} \]
                        Occurs if at least one of the events $E_i$ has occurred (has union of event sets).
                        \\
                        \\ If 4 is rolled on a 6-sided dice, then union of (is 3) and (is 4) occurred.
                    }
                    \bullpara{Intersection / And}{
                        \[\bigcap_iE_i = \{s \in S | \forall i. [s \in E_i]\}\]
                        Occurs if all the events $E_i$ occur.
                        \\
                        \\ If 4 is rolled on a 6-sided dice, the intersection of (is even) and (is 4) occurred.
                    }
                    \bullpara{Mutual Exlusion}{
                        \[E_1 \cap E_2 = \emptyset\]
                        If sets are disjoint, then they are mutually exclusive (cannot occur simultaneously).
                        \\
                        \\ For a 6-sided dice the events (is 4) and (is 6) arte mutually exclusive.
                    }
                \end{itemize}
                
        \subsection*{Probability}
            \lectlink{https://imperial.cloud.panopto.eu/Panopto/Pages/Viewer.aspx?id=1d8ad6f9-ad83-4625-a63a-ae1800b0bad5}
            When determining the probability of every subset $E \subseteq S$ occurring:
            \compitem{
                \bullpara{$S$ is Finite}{Can easily assign probabilites.}
                \bullpara{$S$ is countable}{Can assign probabilites.}
                \bullpara{$S$ is uncountably infinite}{
                    \\ Can initially assign some collection of subsets probabilities, but it then becomes impossible to define probabilities on reamining subsets.
                    \\
                    \\ Cannot make probabilities sum to $1$ with reasonably axioms.
                }
            }
            For this reason when defining a probability function on sample space $S$, we must define the collection of subsets we will measure.
            \\
            \\ The subsets are referred to as $\mathcal{F}$ and must be:
            \compenum{
                \item nonempty ($S \in \mathcal{F}$)
                \item closed under complements $E \in \mathcal{F} \Rightarrow \overline{E} \in \mathcal{F}$
                \item closed under countable union $E_1, E_2, \dots \in \mathcal{F} \Rightarrow \bigcup_iE_i \in \mathcal{F}$
            }
            A collection of sets is known as $\sigma$-algebra.
            \termdef{Probability Measure}{
                A function $P: \mathcal{F} \to [0,1]$ on the pair $(S,\mathcal{F})$ such that:
                \begin{center}
                    \begin{tabular}{l l}
                        Axiom 1. & $\forall E \in \mathcal{F}.[0 \leq P(E) \leq 1]$ \\
                        Axiom 2. & $P(S) = 1$ \\
                        Axiom 3. & Countably additive, for \textbf{disjoint} sets $E_1, E_2, \dots \in \mathcal{F}$: $P(\bigcup_iE_i) = \sum_iP(E_i)$
                    \end{tabular}
                \end{center}
                $P(E)$ provides the probability (between 0 and 1 inclusive) that a given event occurs.
            }
            From the axioms satisfied by a \keyword{probability measure} we can derive that:
            \compenum{
                \item $P(\overline{E}) = 1 - P(E)$
                \item $P(\emptyset) = 0$
                \item For any events $E_1$ and $E_2$: $P(E_1 \cup E_2) = P(E_1) + P(E_2) - P(E_1 \cap E_2)$
            }
        
        \subsection*{Interpretations of Probability}
            \lectlink{https://imperial.cloud.panopto.eu/Panopto/Pages/Viewer.aspx?id=2ef8d19d-1e83-4019-8e44-ae1800b0e0a1}
            \subsubsection*{Classical Interpretation}
                
                Given $S$ is finite and the \keyword{elementary events} are equally likely:
                \[P(E) = \cfrac{|E|}{|S|}\]                
                We can also extend this \keyword{uniform probability distribution} to infinite spaces by considering measures such as area, mass or volume.
                \centerimage{width=0.5\textwidth}{uniform area.png}
            
            \subsection*{Frequentist Interpretation}
                Through repeated observations of identical random experiments in which $E$ can occur, the proportion of experiments where $E$ occurs tends towards the probability of $E$.
                \\
                \\ At an infinite number experiments, the proportion of occurrences of $E$ is equal to $P(E)$.
                \sidenote{Central Limit Theorem}{
                    This can also be considered in terms of \keyword{central limit theorem}, where the greater the sample size taken from some distribution (with defined mean $\mu$), the closer the mean of the sample to the distribution's mean. (more readings results in less variance in the sample means as they converge on the distribution's mean)
                }
            
            \subsection*{Subjective Interpretation}
                Probability is the degree of belief held by an individual.
                \\
                \\ For example if gambling: \begin{tabular}{l l}
                    Option 1: $E$ occurs win $£1$, $\overline{E}$ occurs win $£0$ \\
                    Option 2: Regardless of outcome get $£P(E)$ \\
                \end{tabular}.
                \\
                \\ Either outcome, the gambler receives $£P(E)$. The value of $P(E)$ is the value for which the individual is indifferent about the choice between option 1 or 2. It is the \keyword{individuals probability} of event $E$ occurring.
            
        
        \subsection*{Joint Events and Conditional Probability}
            \lectlink{https://imperial.cloud.panopto.eu/Panopto/Pages/Viewer.aspx?id=291512d9-06e4-4795-a936-ae1800b1063f}
            We commonly need to consider \keyword{Join Events} (where two events occur at the same time).
            \termdef{Independent Events}{
                Two events are independent if the occurence of one does not affect the other. Given $E_1$ and $E_2$ are independent:
                \[E_1 \ and \ E_2 \ independent \Leftrightarrow P(E_1 \ occurrs \ and \ E_2 \ occurs) = P(E_1) \times P(E_2)\]
                More generally, the set of events $\{E_1, E_2, \dots\}$ are independent if for any finite subset $\{E_{i_1}, E_{i_2}, \dots, E_{i_n}\}$:
                \[p(\bigcap_{j=1}^{n}E_{i_j}) = \prod_{j=1}^n P(E_{i_j}) \]
                If $E_1$ and $E_2$ are independent, then so are $\overline{E_1}$ and $E_2$.
                \\
                \\For example with a coin toss, subsequent coin tosses do not effect the next coin toss's probability of heads.
            }
            
            We can show that if $E_1$ and $E_2$ are independent, so are $\overline{E_1}$ and $E_2$:
            \begin{proof}
                \proofstep{1}{$F = (E_1 \cap E_2) \cup (\overline{E_1} \cap E_2)$}{By set operations}
                \proofstep{2}{$P(E_2) = P(E_1 \cap E_2) + p(\overline{E_1} \cap E_2)$}{As \stepno{1} was a disjoint union, Axiom 3}
                \proofstep{3}{$P(\overline{E_1} \cap E_2) = P(E_2) - P(E_1 \cap E_2)$}{}
                \proofstep{4}{$P(\overline{E_1} \cap E_2) = P(E_2) - P(E_1) \times P(E_2)$}{}
                \proofstep{5}{$P(\overline{E_1} \cap E_2) = P(E_2) \times (1 - P(E_1)$}{}
                \proofstep{6}{$P(\overline{E_1} \cap E_2) = P(E_2) \times P(\overline{E_1})$}{By $P(\overline{E}) = 1 - P(E)$}
            \end{proof}
            We can show that $P(E_1 \cup E_2) = P(E_1) + P(E_2) - P(E_1 \cap E_2)$:
            \begin{proof}
                \proofstep{1}{$E_1 \cup E_2 = E_1 \cup (E_2 \cap \overline{E_1})$}{From set theory}
                \proofstep{2}{$P(E_1 \cup E_2) = P(E_1 \cup (E_2 \cap \overline{E_1}))$}{By Axiom 3}
                \proofstep{3}{$P(E_1 \cup E_2) = P(E_1) + P(E_2 \cap \overline{E_1})$}{}
                \proofstep{4}{$P(E_2 \cap \overline{E_1}) = P(E_2) - P(E_1 \cap E_2)$}{By \stepno{3} of the previous proof and as $E_1$ and $E_2$ are independent}
            \end{proof}
            \subsubsection*{For Example: Coin and 6-Sided Dice}
                We can construct a \keyword{Probability Table}:
                \begin{center}
                    \begin{tabular}{c c c c c c c c c}
                        \setlength{\tabcolsep}{3em}
                        & & \multicolumn{6}{c}{Dice} & \multirow{2}{*}{Totals} \\
                                            &     & 1               & 2               & 3               & 4               & 5               & 6              \\
                        \multirow{2}{*}{Coin} & H   & $\sfrac{1}{12}$ & $\sfrac{1}{12}$ & $\sfrac{1}{12}$ & $\sfrac{1}{12}$ & $\sfrac{1}{12}$ &$\sfrac{1}{12}$ & $\sfrac{1}{2}$ \\
                                            & T   & $\sfrac{1}{12}$ & $\sfrac{1}{12}$ & $\sfrac{1}{12}$ & $\sfrac{1}{12}$ & $\sfrac{1}{12}$ &$\sfrac{1}{12}$ & $\sfrac{1}{2}$ \\
                        \multicolumn{2}{c}{Totals} & $\sfrac{1}{6}$  & $\sfrac{1}{6}$  & $\sfrac{1}{6}$  & $\sfrac{1}{6}$  & $\sfrac{1}{6}$  & $\sfrac{1}{6}$ &                \\
                    \end{tabular}
                \end{center}
                We can determine the probability of any event by summing the probabilities of elementary events represented by cells in the table.
                \\
                \\ $P({H})$ is called a \keyword{marginal probability}, as it the probability of one event occurring irrespective of the other (the dice in this case).
                \\
                \\ $P({(H,3)})$ is called a \keyword{joint probability} as it involves both events (dice roll and the coin toss).

            \subsubsection*{For Example: Coin and 2 6-Sided Dice}
                A crooked die (called a top) has the same faces on either side.
                \\
                \\ We flip the coin, then if it is heads we use the normal die, else we use the top.
                \begin{center}
                    \begin{tabular}{c c c c c c c c c}
                        \setlength{\tabcolsep}{3em}
                        & & \multicolumn{6}{c}{Dice} & \multirow{2}{*}{Totals} \\
                                            &     & 1               & 2               & 3               & 4               & 5               & 6                 &                 \\
                        \multirow{2}{*}{Coin} & H   & $\sfrac{1}{12}$ & $\sfrac{1}{12}$ & $\sfrac{1}{12}$ & $\sfrac{1}{12}$ & $\sfrac{1}{12}$ & $\sfrac{1}{12}$ & $\sfrac{1}{2}$  \\
                                              & T   & $\sfrac{1}{6}$  & $0$             & $\sfrac{1}{6}$  & $0$             & $\sfrac{1}{6}$  & $0$             &  $\sfrac{1}{2}$ \\
                        \multicolumn{2}{c}{Totals}  & $\sfrac{1}{4}$  & $\sfrac{1}{12}$ & $\sfrac{1}{4}$  & $\sfrac{1}{12}$ & $\sfrac{1}{4}$  & $\sfrac{1}{12}$ &                 \\
                    \end{tabular}
                \end{center}
                We can now see that $P({(H,3)}) \neq P({H}) \times P({3})$ and hence they are dependent, as the dice roll depends on the coin toss.

        \subsection*{Conditional Probability}
            \lectlink{https://imperial.cloud.panopto.eu/Panopto/Pages/Viewer.aspx?id=71a90c30-38fd-4a9f-b02e-ae1800b0ba59}
            For two events $E$ and $F$ in \keyword{sample space} $S$, where $P(F) \neq 0$:
            \[P(E|F) = \cfrac{P(E \cap F)}{P(F)}\]
            Probability of $E$ given $F$ is the probability of both occurring over the probability of $F$.
            \sidenote{Independence}{
                If $E$ and $F$ are independent:
                \[P(E|F) = \cfrac{P(E \cap F)}{P(F)} = \cfrac{P(E) \times P(F)}{P(F)} = P(E)\]
            }
            \termdef{Conditional Independence}{
                $P(\bullet | F)$ defines a probability measure obeying the axioms of probability on set $F$ (When have just reduced $S$ to $F$).
                \\
                \\ Three events $E_1, E_2, F$ are conditionally independent if and only if:
                \[P(E_1 \cap E_2|F) = P(E_1|F) \times P(E_2|F)\]
            }


            \subsubsection*{Example: Rolling a Dice}
                What is the probability the dice rolls a $3$ given the dice rolls an odd number?
                \[P(\{3\}|\{1,3,5\}) = \cfrac{P(\{3\} \cap \{1,3,5\})}{P(\{1,3,5\})} = \cfrac{P(\{3\})}{P(\{1,3,5\})} = \cfrac{\sfrac{1}{6}}{\sfrac{1}{2}} = \cfrac{1}{3}\]

            \subsubsection*{Example: Rolling two Dice}
                Throw a die from each hand. What is the probability the die thrown from the left is larger than the die thrown from the right.
                \\
                \\ The sample space is:
                \[S = \begin{matrix}
                    \{(1,1),(1,2),(1,3),(1,4),(1,5),(1,6),(2,1),(2,2),(2,3),(2,4),(2,5),(2,6),(3,1),(3,2),(3,3),(3,4),(3,5),(3,6), \\
                    (4,1),(4,2),(4,3),(4,4),(4,5),(4,6),(5,1),(5,2),(5,3),(5,4),(5,5),(5,6),(6,1),(6,2),(6,3),(6,4),(6,5),(6,6)\} \\
                \end{matrix}\]
                We want the event such that the left value of the pair is larger.
                \\
                \\ For value $1$ there are $0$ possible, for $2$ there is $1$ and so on.
                \[(1:0),(2:1),(3:2),(4:3),(5:4),(6:5)\]
                Hence there are $0 + 1 + 2 + 3 + 4 + 5 = 15$ possible pairs with the left larger than the right.
                \[P(E) = \cfrac{15}{36} = \cfrac{5}{12}\]
                However if we know the left or right die, we can determine a new probability. For example if we know the left die is $4$ then we know there are $6$ pairs with the left as $4$, and $3$ of those pairs have a smaller right.
                \[P(E|{4}) = \cfrac{3}{6} = \cfrac{1}{2}\]
            
            \termdef{Bayes Theorem}{
                For two events $E$ and $F$ we have:
                \[P(E \cap F) = P(F) \times P(E|F) = P(F) \times \cfrac{P(E \cap F)}{P(F)} = P(E) \times P(F|E) = P(E) \times \cfrac{P(E \cap F)}{P(E)}\]
                Hence we can deduce:
                \[P(E|F) = \cfrac{P(E) \times P(F|E)}{P(F)}\]
                \centerimage{width=0.6\textwidth}{bayes theorem.png}
            }

            \termdef{Partition Rule}{
                Given a set of events $\{F_1, F_2, \dots\}$ which forms a partition of $S$ (disjoint sets that contain all of $F$).
                \\
                \\ For any event $E \subseteq S$:
                \[P(E) = \sum_iP(E|F_i) \times P(F_i)\]
                \centerimage{width=0.8\textwidth}{partition rule.png}
                Proof:
                \begin{proof}
                    \proofstep{1}{$E = E \cap S = E \cap \bigcup_iF_i = \bigcup_i(E \cap F_i)$}{By set theory and disjointness of partitions.}
                    \proofstep{2}{$P(E) = P(\bigcup_i(E \cap F_i))$}{}
                    \proofstep{3}{$P(E) = \sum_i{P(E \cap F_i)}$}{By axiom 3 and disjointness of partitions.}
                    \proofstep{4}{$P(E) = \sum_iP(E|F_i) \times P(F_i)$}{}
                \end{proof}
            }
            
            \termdef{Law of Total Probability}{
                Given some event $E$ and events $\{F_1, F_2, \dots\}$:
                \[P(E) = \sum_i{P(E \cap F_i)}\]
                For example the 6-Sided dice, $E = {H}$ and $F = [\{1\},\{2\},\{3\},\{4\},\{5\},\{6\}]$, the marginal probability is the same as the sum of all cells in row $H$.
            }

            Using complement as a partition we can deduce that:
            \[P(E) = P(E \cap F) + P(E \cap \overline{F})\]
            \[P(E) = P(E|F) \times P(F) + P(E|\overline{F}) \times P(\overline{F})\]
        
        \subsection*{Terminology Recap}
            \begin{itemize}
                \bullpara{Conditional Probabilities}{Of the form $P(E | F)$.}
                \bullpara{Joint Probabilities}{Of the form $P(E \cap F)$.}
                \bullpara{Marginal Probabilities}{Of the form $P(E)$.}
            \end{itemize}



            
            

                


            


\end{document}
