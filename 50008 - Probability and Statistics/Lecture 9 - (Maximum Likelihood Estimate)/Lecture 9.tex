\documentclass{report}
    \title{50008 - Probability and Statistics - Lecture 9}
    \author{Oliver Killane}
    \date{08/03/22}
%===========================COMMON FORMAT & COMMANDS===========================
% This file contains commands and format to be used by every module, and is 
% included in all files.
%===============================================================================

%====================================IMPORTS====================================
\usepackage[a4paper, total={6in, 8in}]{geometry}
\usepackage{graphicx, amssymb, amsfonts, amsmath, xcolor, listings, tcolorbox, multirow, hyperref}
%===============================================================================

%====================================IMAGES=====================================
\graphicspath{{image/}}

% \centerimage{options}{image}
\newcommand{\centerimage}[2]{\begin{center}
    \includegraphics[#1]{#2}
\end{center}}
%===============================================================================

%=================================CODE LISTINGS=================================
\definecolor{codebackdrop}{gray}{0.9}
\definecolor{commentgreen}{rgb}{0,0.6,0}
\lstset{
    inputpath=code, 
    commentstyle=\color{commentgreen},
    keywordstyle=\color{blue}, 
    backgroundcolor=\color{codebackdrop}, 
    basicstyle=\footnotesize,
    frame=single,
    numbers=left,
    stepnumber=1,
    showstringspaces=false,
    breaklines=true,
    postbreak=\mbox{\textcolor{red}{$\hookrightarrow$}\space}
}

% Create a code listing for a single line
% \codeline{language}{line}{file}
\newcommand{\codeline}[3]{\lstinputlisting[language=#1, firstline = #2, lastline = #2]{#3}}

% Create a code listing for a given language & file
% \codelist{language}{file}
\newcommand{\codelist}[2]{\lstinputlisting[language=#1]{#2}}
%===============================================================================

%================================TEXT STRUCTURES================================
% Marka a word as bold
% \keyword{important word}
\newcommand{\keyword}[1]{\textbf{#1}}

% Creates a section in italics
% \question{question in italics}
\newcommand{\question}[1]{\textit{#1} \\ }

% Creates a box with title for side notes.
% \sidenote{title}{contents}
\newcommand{\sidenote}[2]{\begin{tcolorbox}[title=#1]#2\end{tcolorbox}}

% Creates an item in an itemize or enumerate, with a paragraph after
% \begin{itemize}
%     \bullpara{title}{contents}
% \end{itemize}
\newcommand{\bullpara}[2]{\item \textbf{#1} \ #2}

% Creates a compact list (very small gaps between items)
% \compitem{
%     \item item 1
%     \item item 2
%     \item ...
% }
\newcommand{\compitem}[1]{\begin{itemize}\setlength\itemsep{-0.5em}#1\end{itemize}}

% Creates a link to the lecture for use at the start of the notes document
\newcommand{\lectlink}[1]{\sidenote{Lecture Recording}{
    Lecture recording is available \href{#1}{here}
}}
%===============================================================================

\begin{document}
\maketitle
\lectlink{https://imperial.cloud.panopto.eu/Panopto/Pages/Viewer.aspx?id=fa26c926-d5ca-4ff6-8186-ae4b010239df}
\section*{Maximum Likelihood Estimate}
Given some distribution with an unknown parameter $\theta$:
\[X \thicksim Distribution(\dots \theta \dots)\]
And a sample taken from the distribution $\underline{X}$:
\[\underline{X} = (X_1, X_2, \dots, X_n)\]
We want to know the value of $\theta$ for which the likelihood of the sample occurring is highest.
\termdef{Likelihood Function}{
	The likelihood of some observations $x_1, x_2, \dots, X_n$ occurring given some $\theta$ are:
	\[\begin{split}
			L(\theta) &= P(x_1, x_2, \dots, x_n|\theta) \\
			&= \prod_{i=1}^n f(x_i|\theta)
		\end{split}\]
	This is as $f$ is the \keyword{probability mass function}, and as each observation is independent we can multiply their probabilities.
}
\termdef{Log Likelihood Function}{
	Used more often than likelihood (easier to work with, and converts decimal small values to large negative values - avoids floating point errors)
	\[l(\theta) = \ln L(\theta)\]
}
To do this, we construct the likelihood (or log likelihood) function from the distribution and sample in term of $\theta$.
\\
\\ Then we can differentiate the function to determine the value of $\theta$ for the maximum.
\\
\\ This value of $\theta$ is the \keyword{Maximum Likelihood Estimate} ($\hat{\theta}$).

\section*{Common Maximum Likelihood Estimates}
Given a sample $\underline{x} = (x_1, x_2, \dots, x_n)$, we can use formulas for the maximum likelihood.

\subsection*{Exponential Distribution}
\[X \thicksim Exp(\theta) \Rightarrow f(x) = \theta e^{-\theta x}\]
First we determine the \keyword{likelihood} in terms of $\theta$.
\[\begin{split}
		L(\theta) &= \prod_{i=1}^n f(x_i) \\
		&= \prod_{i=1}^n \theta e^{-\theta x_i} \\
		&= \theta^n\prod_{i=1}^n e^{-\theta x_i} \\
		&= \theta^n e^{-\theta\sum_{i=1}^n x_i} \\
	\end{split}\]
Next we can derive the \keyword{log likelihood}
\[\begin{split}
		l(\theta) &= \ln L(\theta) \\
		&= \ln \left(\theta^n e^{-\theta \sum_{i=1}^nx_i} \right)\\
		&= n\ln \theta -\theta\sum_{i=1}^n x_i \\
	\end{split}\]
Next we can differentiate and set equal to zero:
\[\begin{split}
		\cfrac{dl(\theta)}{d\theta} &= n\cfrac{1}{\theta} - \sum_{i=1}^nx_i = 0\\
		0 &= \cfrac{n}{\theta} - \sum_{i=1}^nx_i \\
		\sum_{i=1}^nx_i &= \cfrac{n}{\theta} \\
		\theta &= \cfrac{n}{\sum_{i=1}^nx_i} \\
	\end{split}\]
Hence the maximum likelihood estimator is the reciprocal of the mean of the sample.
\[\hat{\theta} = \sfrac{1}{\overline{x}}\]
\subsection*{Geometric Distribution}
\[X \thicksim Geo(\theta) \Rightarrow f(x) =\theta(1-\theta)^{x - 1}\]
\[\begin{split}
		L(\theta) &= \prod_{i=1}^n f(x_i) \\
		& \prod_{i=1}^n \theta(1-\theta)^{x_i - 1} \\
		& \theta^n \prod_{i=1}^n (1-\theta)^{x_i - 1} \\
		& \theta^n (1 - \theta)^{\sum_{i=1}^n(x_i - 1)} \\
		& \theta^n (1 - \theta)^{\left(\sum_{i=1}^nx_i\right) - n} \\
	\end{split}\]
Now we find the \keyword{log likelihood}.
\[\begin{split}
		l(\theta) &= \ln L(\theta) \\
		&= \ln \left( \theta^n (1 - \theta)^{\left(\sum_{i=1}^nx_i\right) - n} \right) \\
		&= n\ln\theta +\left(\left(\sum_{i=1}^nx_i\right) - n\right)\ln\left(1 - \theta\right) \\
	\end{split}\]
Now we differentiate, and set equal to zero to find $\hat{\theta}$.
\[\begin{split}
		\cfrac{dl(\theta)}{d\theta} &= \cfrac{n}{\theta} + \left(\left(\sum_{i=1}^nx_i\right) - n\right)\cfrac{1}{\theta - 1} = 0 \\
		0 &=\cfrac{n(\theta - 1)}{\theta(\theta - 1)} + \left(\left(\sum_{i=1}^nx_i\right) - n\right)\cfrac{\theta}{\theta(\theta - 1)} \\
		0 &=n(\theta - 1)+ \left(\left(\sum_{i=1}^nx_i\right) - n\right)\theta \\
		0 &=n\theta - n + \left(\left(\sum_{i=1}^nx_i\right) - n\right)\theta \\
		n &=\left(\sum_{i=1}^nx_i\right)\theta \\
		\cfrac{n}{\sum_{i=1}^nx_i} &=\theta \\
	\end{split}\]
Hence the maximum likelihood estimator is the reciprocal of the mean of the sample.
\[\hat{\theta} = \sfrac{1}{\overline{x}}\]
\subsection*{Binomial Distribution}
\[X \thicksim Binomial(m, \theta) \Rightarrow f(x) =\begin{pmatrix}
		m \\ x
	\end{pmatrix}\theta^x(1 - \theta)^{m-x}\]
\[\begin{split}
		L(\theta) &= \prod_{i=1}^n f(x_i) \\
		&= \prod_{i=1}^n \begin{pmatrix}
			m \\ x_i
		\end{pmatrix}\theta^{x_i}(1 - \theta)^{m-x_i} \\
		&= \prod_{i=1}^n \begin{pmatrix}
			m \\ x_i
		\end{pmatrix} \times \prod_{i=1}^n\theta^{x_i} \times \prod_{i=1}^n(1 - \theta)^{m-x_i} \\
		&= \prod_{i=1}^n \begin{pmatrix}
			m \\ x_i
		\end{pmatrix} \times \theta^{\sum_{i=1}^nx_i} \times (1 - \theta)^{\sum_{i=1}^n m - x_i} \\
		&= \prod_{i=1}^n \begin{pmatrix}
			m \\ x_i
		\end{pmatrix} \times \theta^{\sum_{i=1}^nx_i} \times (1 - \theta)^{mn - \sum_{i=1}^n x_i} \\
	\end{split}\]
Now we find the \keyword{log likelihood}.
\[\begin{split}
		l(\theta) &= \ln L(\theta) \\
		&= \ln \left( \prod_{i=1}^n \begin{pmatrix}
			m \\ x_i
		\end{pmatrix} \times \theta^{\sum_{i=1}^nx_i} \times (1 - \theta)^{mn - \sum_{i=1}^n x_i} \right) \\
		&= \ln \prod_{i=1}^n \begin{pmatrix}
			m \\ x_i
		\end{pmatrix} + \ln \theta^{\sum_{i=1}^nx_i} + \ln (1 - \theta)^{mn - \sum_{i=1}^n x_i} \\
		&= \ln \prod_{i=1}^n \begin{pmatrix}
			m \\ x_i
		\end{pmatrix} + \sum_{i=1}^nx_i\ln \theta + \left( mn - \sum_{i=1}^n x_i \right)\ln (1 - \theta) \\
	\end{split}\]
Now we differentiate, and set equal to zero to find $\hat{\theta}$.
\[\begin{split}
		\cfrac{dl(\theta)}{d\theta} &= 0 + \sum_{i=1}^nx_i\cfrac{1}{\theta} + \left( mn - \sum_{i=1}^n x_i \right)\cfrac{1}{\theta - 1}= 0 \\
		0 & = \sum_{i=1}^nx_i\cfrac{\theta - 1}{\theta(\theta - 1)} + \left( mn - \sum_{i=1}^n x_i \right)\cfrac{\theta}{\theta(\theta - 1)} \\
		0 & = \sum_{i=1}^nx_i (\theta - 1) + \left( mn - \sum_{i=1}^n x_i \right)\theta \\
		0 & = \theta\sum_{i=1}^nx_i - \sum_{i=1}^nx_i + mn\theta - \theta\sum_{i=1}^n x_i  \\
		0 & = - \sum_{i=1}^nx_i + mn\theta  \\
		\cfrac{\sum_{i=1}^nx_i}{mn} & =\theta  \\
	\end{split}\]
Hence the maximum likelihood estimator is the sample mean divided by the number of trials (for binomial):
\[\hat{\theta} = \cfrac{\overline{x}}{m}\]

\end{document}
