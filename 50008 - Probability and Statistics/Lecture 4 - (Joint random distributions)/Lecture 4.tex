\documentclass{report}
    \title{50008 - Probability and Statistics - Lecture 4}
    \author{Oliver Killane}
    \date{01/02/22}
%===========================COMMON FORMAT & COMMANDS===========================
% This file contains commands and format to be used by every module, and is 
% included in all files.
%===============================================================================

%====================================IMPORTS====================================
\usepackage[a4paper, total={6in, 8in}]{geometry}
\usepackage{graphicx, amssymb, amsfonts, amsmath, xcolor, listings, tcolorbox, multirow, hyperref}
%===============================================================================

%====================================IMAGES=====================================
\graphicspath{{image/}}

% \centerimage{options}{image}
\newcommand{\centerimage}[2]{\begin{center}
    \includegraphics[#1]{#2}
\end{center}}
%===============================================================================

%=================================CODE LISTINGS=================================
\definecolor{codebackdrop}{gray}{0.9}
\definecolor{commentgreen}{rgb}{0,0.6,0}
\lstset{
    inputpath=code, 
    commentstyle=\color{commentgreen},
    keywordstyle=\color{blue}, 
    backgroundcolor=\color{codebackdrop}, 
    basicstyle=\footnotesize,
    frame=single,
    numbers=left,
    stepnumber=1,
    showstringspaces=false,
    breaklines=true,
    postbreak=\mbox{\textcolor{red}{$\hookrightarrow$}\space}
}

% Create a code listing for a single line
% \codeline{language}{line}{file}
\newcommand{\codeline}[3]{\lstinputlisting[language=#1, firstline = #2, lastline = #2]{#3}}

% Create a code listing for a given language & file
% \codelist{language}{file}
\newcommand{\codelist}[2]{\lstinputlisting[language=#1]{#2}}
%===============================================================================

%================================TEXT STRUCTURES================================
% Marka a word as bold
% \keyword{important word}
\newcommand{\keyword}[1]{\textbf{#1}}

% Creates a section in italics
% \question{question in italics}
\newcommand{\question}[1]{\textit{#1} \\ }

% Creates a box with title for side notes.
% \sidenote{title}{contents}
\newcommand{\sidenote}[2]{\begin{tcolorbox}[title=#1]#2\end{tcolorbox}}

% Creates an item in an itemize or enumerate, with a paragraph after
% \begin{itemize}
%     \bullpara{title}{contents}
% \end{itemize}
\newcommand{\bullpara}[2]{\item \textbf{#1} \ #2}

% Creates a compact list (very small gaps between items)
% \compitem{
%     \item item 1
%     \item item 2
%     \item ...
% }
\newcommand{\compitem}[1]{\begin{itemize}\setlength\itemsep{-0.5em}#1\end{itemize}}

% Creates a link to the lecture for use at the start of the notes document
\newcommand{\lectlink}[1]{\sidenote{Lecture Recording}{
    Lecture recording is available \href{#1}{here}
}}
%===============================================================================

\begin{document}
\maketitle

\section*{Joint Distributions}
\subsection*{CDF}
\lectlink{https://imperial.cloud.panopto.eu/Panopto/Pages/Viewer.aspx?id=ff32ac5f-7808-4e9b-a622-ae260066a944}
Suppose we have random variables $X$ and $Y$ such that:
\[X: S_X \to \mathbb{R} \ \text{ and } \ Y: S_Y \to \mathbb{R}\]
We can define $Z$ operating on sample space $S$ such that:
\[\begin{matrix}
		S = S_1 \times S_2 & S = \{(s_X, s_Y) | s_X \in S_X \land s_Y \in S_Y\} & Z = (X, Y): S \to \mathbb{R}^2
	\end{matrix}\]
Hence we have a mapping from joint random variable $Z(s)$ onto $(X(s), Y(s))$.
\\
\\ We can consider this using a graph of the sample space:
\centerimage{width=\textwidth}{joint sample space}
Hence the induced probability function for $Z$ will be:
\[F(x,y) = P_Z(X \leq x, Y \leq y) = P_Z((-\infty, x], (- \infty, y]) = P(S_{XY})\]
Hence we can use the marginals of the joint distribution to get the distribution of the two random variables:
\[F_X(x) = F(x, \infty) \ \text{ and } \ F_Y(y) = F(\infty, y)\]
To be a valid \keyword{joint cumulative distribution function}:
\compitem{
\item $\forall x,y \in \mathbb{R}. \ 0 \leq F(x,y) \leq 1$
\bullpara{Monotonicity}{\[\forall x_1,x_2,y_1,y_2 \in \mathbb{R}. \ [x_1 < x_2 \Rightarrow F(x_1,y_1) \leq F(x_2, y_1) \land y_1 < y_2 \Rightarrow F(x_1,y_1) \leq F(x_1, y_2)]\]}
\item $\forall x,y \in \mathbb{R}. \ F(x, -\infty) = F(-\infty, y) = 0$
\item $F(\infty, \infty) = 1$
}

For the probability of intervals we can use the graph mapping concept again:
\centerimage{width=0.4\textwidth}{joint interval.png}
\[P_Z(x_1 < X \leq x_2, Y \leq y) = F(x_2, y) - F(x_1, y)\]
Hence we can get the interval:
\[P_Z(x_1 < X \leq x_2, y_1 < Y \leq y_2) = F(x_2, y_2) - F(x_1, y_2) -F(x_2,y_1) + F(x_1,y_1)\]

\subsection*{PMF}
\termdef{Joint Probability Mass Function}{
	\[p(x,y) = P_Z(X = x, Y = y) \ \text{where } x,y \in \mathbb{R}\]
	We can get the original \keyword{pmfs} of the two variables as:
	\[p_X(x) = \sum_y p(x,y) \ \text{ and } \ p_Y(y) = \sum_x p(x,y)\]
	To be a valid \keyword{pmf}:
	\compitem{
		\item $\forall x,y \in \mathbb{R}. \ 0 \leq p(x,y) \leq 1$
		\item $\sum_y \sum_x p(x,y) = 1$
	}
}

\subsection*{PDF}
\lectlink{https://imperial.cloud.panopto.eu/Panopto/Pages/Viewer.aspx?id=5d0ec560-cd40-48d2-8118-ae260066a9c4}
\sidenote{Fundamental Theorem of Caculus}{
	The fundamental law that integration and differentiation and the inverse of each other (except for constant added in integration $c$, which does not affect definite integrals).
}

\termdef{Joint Probability Density Function}{
	When the variables being \textit{joined} are continuous we have $\mathbb{R} \times \mathbb{R} \to \mathbb{R}$, in this case:
	\[F(x,y) = \int_{a=-\infty}^y \int_{b=-\infty}^x f(b,a) \ db \ da\]
	The sum of the probability density function from $(x,y) \to (-\infty, -\infty)$
	\\
	\\ Hence by the fundamental theorem of calculus:
	\[f(x,y) = \cfrac{\sigma^2}{\sigma x \sigma y }F(x,y)\]
	We can differentiate to go get the PMF from the PDF.
	\\
	\\ To be valid:
	\compitem{
		\item $\forall x, y \in \mathbb{R}. f(x,y) \geq 0\ $
		\item $\int_{y=-\infty}^{\infty} \int_{x=-\infty}^{\infty} f(x,y) \ dx \ dy = 1$
	}
}

\termdef{Marginal Density Functions}{
	\[\begin{split}
			f_X(x) & = \cfrac{d}{dx} F_X(x) = \cfrac{d}{dx}F(x, \infty) \\
			& = \cfrac{d}{dx} \int_{y=-\infty}^{\infty} \int_{s = -\infty}^x f(s,y) \ ds \ dy \\
		\end{split}\]
	And likewise for y:
	\[f_Y(y) = \cfrac{d}{dy} \int_{x = -\infty}^{\infty} \int_{s = -\infty}^y f(x,s) \ ds \ dx\]
	Hence by applying the fundamental theorem of calculus:
	\[f_X(x) = \int_{y=-\infty}^{\infty} f(x,y) \ dy\]
	\[f_Y(y) = \int_{x=-\infty}^{\infty} f(x,y) \ dx\]
}
\example{Marginal pdf}{
	Given continuous variables $(X, Y) \in \mathbb{R}^2$:
	\[f(x,y) = \begin{cases}
			1 & |x| + |y| < \cfrac{1}{\sqrt{2}} \\
			0 & otherwise                       \\
		\end{cases}\]
	To determine the marginal \keyword{pdf}s for $X$ and $Y$:
	\\
	\\ First notice that: $|x| + |y| < \cfrac{1}{\sqrt{2}} \Leftrightarrow |y| < \cfrac{1}{\sqrt{2}} - |x|$.
	\\
	\\ Hence given an $x$ we can see that for the first case of the probability density function to match, $y$ must be between:
	\[\cfrac{1}{- \sqrt{2}} + |x| < y < \sqrt{2} - |x|\]
	\[\begin{split}
			f_X(x) & = \int_{y=-\infty}^{\infty} f(x,y) \ dy \\
			& = \int_{y=- \sqrt{2} + |x|}^{\sqrt{2} - |x|} 1 \ dy \\
			& = \left[ y \right]_{-\sqrt{2} + |x|}^{\sqrt{2} - |x|} \\
			& = \left( \sqrt{2} - |x| \right) - \left( -\sqrt{2} + |x| \right) \\
			& = 2 \sqrt{2} - 2|x|  \\
		\end{split}\]
	Similarly for $y$:
	\[f_Y(y) = 2 \sqrt{2} - 2|y|\]
}

\termdef{Multinomial Distribution}{
Given:
\compitem{
	\item sequence of $n$ independent and identical experiments (all same distribution, same parameters).
	\item $r$ possible outcomes for each experiment.
	\item Each probability $q_i$ is the probability of outcome $i$.
	\item The sum of all probabilities for the outcomes is 1: $\sum_{i=1}^r q_i = 1$
}
We can have a set of random variables where each $X_i$ represents the number of experiments resulting in outcome $i$.

\[P(X_1 = n_1, X_2 = n_2, \dots, X_r = n_r) = \cfrac{n!}{n_1! \times n_2! \times \dots \times n_r!} \times q_1^{n_1} \times q_2^{n_2} \times \dots \times  q_r^{n_r}\]

We know this as any sequence will have the probability $q_1^{n_1} \times q_2^{n_2} \times \dots \times q_r^{n_r}$ where $n_1 + n_2 + \dots + n_r = n$ (multiplying the probabilities in a sequence).
\\
\\ For a given number of outcomes, there are many different sequences like the above. We can determine the number of sequences as:
\[\begin{pmatrix}
		n \\ n_1
	\end{pmatrix} \begin{pmatrix}
		n - n_1 \\ n_2
	\end{pmatrix} \dots \begin{pmatrix}
		n - \sum_{i=1}^{r-1} n_i \\ n_r
	\end{pmatrix} = \cfrac{n!}{n_1! \times n_2! \times \dots \times n_r!}\]
}

\example{Party Politics}{
	Given 4 different political parties with popularities:
	\begin{center}
		\begin{tabular}{l l}
			\textbf{Party} & \textbf{Polling Percentage} \\
			Ingsoc         & 40\%                        \\
			Techno Union   & 20\%                        \\
			Norsefire      & 15\%                        \\
			Birthday Party & 25\%                        \\
		\end{tabular}
	\end{center}
	If asking 10 people of what party they prefer, what is the probability that:
	\compitem{
		\item 2 support Ingsoc
		\item 4 support the Techno Union
		\item 1 supports Norsefire
		\item 3 support the Birthday Party
	}
	\[P(X_{ingsoc} = 2, X_{techno-union} = 4, X_{norsefire} = 1, X_{birthday} = 3)\]
	\[\cfrac{10!}{2! \times 4! \times 1! \times 3!} \times (0.4)^2 \times (0.2)^4 \times (0.15)^1 \times (0.25)^3\]
	\[\cfrac{189}{25000} = 0.00756 = 0.756\%\]


}

\section*{Joint Conditional Random Variables}
\lectlink{https://imperial.cloud.panopto.eu/Panopto/Pages/Viewer.aspx?id=bd9dd283-6e6d-4643-9b8a-ae260066a98b}
Given random variables $X$ and $Y$:
\[\text{variables independent} \ \Leftrightarrow F(x,y) = F_X(x)F_Y(y)\]
(For both continuous and discrete)
\\
\\ More specifically:
\begin{center}
	\begin{tabular}{r c l}
		For Discrete Variables   & $p(x,y) = p_X(x)p_Y(y)$ & (probability mass function)    \\
		For Continuous Variables & $f(x,y) = f_X(x)f_Y(y)$ & (Probability density function) \\
	\end{tabular}
\end{center}
\example{Diamond at origin}{
	Consider \keyword{pdf}:
	\[f(x,y) = \begin{cases}
			1 & |x| + |y| < \cfrac{1}{\sqrt{2}} \\
			0 & otherwise                       \\
		\end{cases}\]
	By the previous example:
	\[\begin{split}
			f_X(x) & = 2 \sqrt{2} - 2|x| \\
			f_Y(y) & = 2 \sqrt{2} - 2|y| \\
		\end{split}\]
	Hence as $f(x,y) \neq f_X(x)f_Y(y)$ and hence $X$ and $Y$ are not independent.
}

\example{Independent variables}{
	Given two continuous random variables $X$ and $Y$:
	\[f(x,y) = \lambda_1\lambda_2 e^{-\lambda_1 x - \lambda_2 y} \ \text{ given } x,y > 0\]
	We can get the marginal \keyword{pdf} by integrating over all of y:
	\[\begin{split}
			f(x) & = \int_{y = -\infty}^{\infty} f(x,y) dy \\
			& = \int_{y = 0}^{\infty} f(x,y) dy \\
			& = \lim_{t \to \infty}\int_{y = 0}^{t} \lambda_1 \lambda_2 e^{- \lambda_1 x - \lambda_2 y} dy \\
			& = \lim_{t \to \infty}\int_{y = 0}^{t} \lambda_1 \lambda_2 e^{- \lambda_1 x} \times e^{- \lambda_2 y} dy \\
			& = \lim_{t \to \infty}\left[ -\lambda_1e^{-\lambda_1 x - \lambda_2 y} \right]_{y=0}^{y=t} \\
			& = \lim_{t \to \infty}\left( -\lambda_1e^{-\lambda_1 x - \lambda_2 t} \right) - \left( -\lambda_1e^{-\lambda_1 x - \lambda_2 0} \right)\\
			& = \lim_{t \to \infty}\left( -\lambda_1e^{-\lambda_1 x - \lambda_2 t} \right) - \left( -\lambda_1e^{-\lambda_1 x - \lambda_2 0} \right)\\
			& = 0 - \left( -\lambda_1e^{-\lambda_1 x} \right)\\
			& = \lambda_1e^{-\lambda_1 x}\\
		\end{split}\]
	We can do the same for $f_Y(y)$ to get $ \lambda_2e^{-\lambda_2 y}$.
	\\
	\\ Hence the events are independent as:
	\[\lambda_1\lambda_2 e^{-\lambda_1 x - \lambda_2 y} = \lambda_1e^{-\lambda_1 x} \times \lambda_2e^{-\lambda_2 y} \]
}

\subsection*{Conditional PMF}
For discrete random variables we can define the joint \keyword{pmf} as:
\[p_{X | Y}(x | y) = \cfrac{p(x,y)}{p_Y(y)} \ \text{ where } \forall y. p_Y(y) > 0\]
\termdef{Baye's Theorem}{
	\keyword{Baye's theorem} states that on some partition of the sample space $S$, $P_1, \dots P_k$:
	\[P(X) = \sum_{i = 1}^k P(X|E_i)P(E_i)\]
	Given each partition the probability of some $X$ occurring sums to the total probability of $X$ occurring.
	\\
	\\ Using the conditional joint \keyword{pmf} we can also express this theorem (over a single partition) as:
	\[p_{X|Y}(x|y) \times p_Y(y)= p_{Y|X}(y|x) \times p_X(x)\]
}

\termdef{Conditional PMF Marginal Joint Probabilities} {
	\[p(x) = \sum_yp_{X|Y}(x|y)p_Y(y)\]
	(Go through every y, summing the probability of x occurring with that y, multiplied by the probability of that y)
}
\subsection*{Conditional PDF}
For continuous random variables we can define the joint \keyword{pdf} as:
\[f_{X|Y}(x|y) = \cfrac{f(x,y)}{f_Y(y)}\]
\[\text{$X$ and $Y$ independent } \Leftrightarrow \forall x,y \in \mathbb{R}. \ f_{X|Y}(x,y) = f_X(x)\]
And we can now have \keyword{bayes theorem} as:
\[f_{X|Y}(x|y) = \cfrac{f_{Y|X} f_X(x)}{f_Y(y)}\]

\termdef{Conditional PDF Marginal Joint Probabilities}{
	\[f_X(x) = \int_{y = -\infty}^{\infty} f_{X|Y}(x|y)f_Y(y) \ dy\]
	and with the cumulative distribution:
	\[F_X(x) = \int_{y = -\infty}^{\infty} F_{X|Y}(x|y)f_Y(y) \ dy\]
}

\example{Independent exponential random variables}{
	Given $X \thicksim Exp(\lambda)$ and $Y \thicksim Exp(\mu)$ what is $P(X < Y)$.
	\[\begin{split}
			P(X < Y) & = \int_{x<y} f(x,y) \ dx \ dy \\
			& = \int_{y = -\infty}^{\infty} \int_{x = -\infty}^{y} f(x,y) \ dx \ dy \ \text{(go over all $y$s, for each take the $x$s that are less)} \\
			& = \int_{y = -\infty}^{\infty} \int_{x = -\infty}^{y} f_X(x)f_Y(y) \ dx \ dy \ \text{($X$ and $Y$ are independent)} \\
			& = \int_{y = -\infty}^{\infty} \int_{x = -\infty}^{y} f_X(x)f_Y(y) \ dx \ dy \ \text{($X$ and $Y$ are independent)} \\
			& = \int_{y = -\infty}^{\infty} F_X(y) \times (\mu e^{- \mu y}) \ dx \ dy \ \text{(Integrate $f_X$ to get $F_X$ and then get all below $y$)}\\
			& = \int_{y = -\infty}^{\infty} (1 - e^{- \lambda y}) \times (\mu e^{- \mu y}) \ dx \ dy  \ \text{(Substitute definitions)}\\
			& = \int_{y = 0}^{\infty} (1 - e^{- \lambda y}) \times (\mu e^{- \mu y}) \ dx \ dy \ \text{(exponential cut at 0)}\\
			& = \lim_{t \to \infty} \int_{y = 0}^{t} (1 - e^{- \lambda y}) \times (\mu e^{- \mu y}) \ dx \ dy\\
			& = \lim_{t \to \infty} \int_{y = 0}^{t} (\mu e^{- \mu y}) - e^{- \lambda y} \times (\mu e^{- \mu y})  \ dx \ dy \\
			& = \lim_{t \to \infty} \int_{y = 0}^{t} (\mu e^{- \mu y}) - \mu e^{(- \lambda - \mu) y}  \ dx \ dy \\
			& = \lim_{t \to \infty} \left[  -e^{-\mu y} + \cfrac{- \mu}{- \lambda - \mu}e^{(-\lambda -\mu)y} \right]_{y=0}^{y=t} \\
			& = \lim_{t \to \infty} \left[  -e^{-\mu y} + \cfrac{\mu}{\lambda + \mu}e^{(-\lambda -\mu)y} \right]_{y=0}^{y=t} \\
			& = \lim_{t \to \infty} \left(  -e^{-\mu t} + \cfrac{\mu}{\lambda + \mu}e^{(-\lambda -\mu)t} \right) - \left(  -e^{\mu 0} + \cfrac{\mu}{\lambda + \mu}e^{(-\lambda -\mu)0} \right) \\
			& = \left(  0 - 0 \right) - \left(  -1 + \cfrac{\mu}{\lambda + \mu} \right) \\
			& = 1 - \cfrac{\mu}{\lambda + \mu} = \cfrac{\lambda}{\lambda + \mu} \\
		\end{split}\]
}
\section*{Expectation and Variance for Joint Random Variables}
\lectlink{https://imperial.cloud.panopto.eu/Panopto/Pages/Viewer.aspx?id=2d2b060e-ff95-4a57-bd11-ae260066a91f}
\termdef{Joint Expectation}{
	Where $g$ is a \keyword{bivariat function} on the random variables $X$ and $Y$:
	\\
	\\ For \keyword{discrete variables}:
	\[E(g(X,Y)) = \sum_{y}\sum_{x}g(x,y)p(x,y)\]
	For \keyword{continuous variables}:
	\[E(g(X,Y)) = \int_{y = -\infty}^{\infty} \int_{x=-\infty}^{\infty} g(x,y) f(x,y) \ dx \ dy\]
	Hence we have the following:
	\compitem{
		\bullpara{For all}{$g(X,Y) = g_1(X) + g_2(Y) \Rightarrow E(g_1(X) + g_2(Y)) = E_X(g_1(X)) + E_Y(g_2(Y))$}
		\bullpara{If $X$ and $Y$ are independent}{$E(g_1(X) \times g_2(Y)) = E_X(g_1(X))) \times E_Y(g_2(Y))$
			\\ Hence where $g(X,Y) = X \times Y$ we have $E(XY) = E_X(X) \times E_Y(Y)$}
	}
}

\termdef{Covariance}{
Covariance measures how two random variables change with respect to one another.
\\
\\ For a single random variable we consider expected value of the difference between the mean and the value, squared.
\[\text{Expectation of } g(X) = (X - \mu_X)^2 = \sigma_X^2\]
\\
\\ For a bivariate we consider the expectation:
\[\text{Expectation of } g(X, Y) = (X - \mu_X)(Y - \mu_Y)\]
We can then defined the covariance as:
\[\begin{split}
		\sigma_{XY} = Cov(X, Y) & = E[(X - \mu_X)(Y - \mu_Y)] \\
		& = E[XY] - E_X[X] \times E_Y[Y] \\
		& = E[XY] - \mu_X\mu_Y \\
	\end{split}\]
When $X$ and $Y$ are independent so:
\[\sigma_{XY} = Cov(X, Y) = E[XY] - E_X[X] \times E_Y[Y] = E[XY] - E[XY] = 0\]
}

\termdef{Correlation}{
	Much like covariance, however is invariant to the scale of $X$ and $Y$.
	\[\rho_{XY} = Cor(X,Y) = \cfrac{\sigma_{XY}}{\sigma_X \times \sigma_Y}\]
	If the variables are independent then $\rho_{XY} = \sigma_{XY} = 0$.
}

\section*{Multivariate Normal Distribution}
\termdef{Multivariate Normal Distribution}{
	Given a random vector $X = (X_1, \dots, X_n)$ with means $\mu = (\mu_1, \dots, \mu_n)$ has joint \keyword{pdf}:
	\[f_X = \cfrac{1}{\sqrt{(2 \pi)^n det \sum}}exp(-\sfrac{1}{2}(x - \mu)^T \sum^{-1}(x - \mu))\]
	Where $\sum$ is the covariance matrix:
	\[\sum_{(i,j)} = Cov(X_i, X_j) \ \text{ where } 1 \leq i, j \leq n\]
	The covariance matrix must be \keyword{positive-definite} for a \keyword{pdf} to exist
	Note that the random variables do not need to be independent.
}

\sidenote{Positive Definite real Matrices}{
	\[M \text{ is positive-definite} \Leftrightarrow \forall x \in \mathbb{R^n} \backslash \{0\}. \ x^TMx > 0\]
}

\section*{Conditional Expectation}
\termdef{Conditional Expectation}{
	In general $E(XY) \neq E_X(X)E_Y(Y)$

	For discrete random variables the \keyword{conditional expectation} of $Y$ given that $X = x$ is:
	\[E_{Y|X}(Y|x) = \sum_y y p_{y|X}(y|x)\]

	For continuous random variables:
	\[E_{Y|X}(Y|x) = \int_{y = -\infty}^{\infty} y f_{Y|X}(y|x) \ dy\]

	In both cases the conditional expectation is a function of $x$ and not $Y$. We are getting the weighted sum over all $Y$s, for a single value ($x$) of $X$.
}

\termdef{Expectation of a Conditional Expectation}{
	We can define random variable $W$ such that:
	\[W = E_{Y|X}(Y|X)\]
	$W$ is effectively a function of the random variable $X: S \to \mathbb{R}$ by $W(s) = E_{Y|X}(Y|x)$ where $X(s) = x$.
	\\
	\\ Using this we can determine that:
	\[E_Y(Y) = E_X(E_{Y|X}(Y|X))\]
	(Expectation of Y is the same as the expectation function of X, of the expected value of Y given X)
	\\
	\\ This holds for both discrete and continuous.
	\[\int_y \int_x yf_{Y|X}(y|x)f_X(x) \ dx \ dy = \int_y \int_x yf(x,y) \ dx \ dy = \int_y y f_Y(y) \ dy \]
}

\termdef{Tower Rule}{
	The expectation of a conditional expectation rule extends to chains of expectations:
	\[\begin{split}
			E(Y) & = E_{X_1}(E_Y(Y | X_1)) \\
			& = E_{X_2}(E_{X_1}(E_Y(Y | X_1, X_2) | X_2)) \\
			& = \dots \\
			& = E_{X_n}(E_{X_{n-1}} ( \dots E_{X_1}(E_Y(Y | X_1 , \dots, X_n)| X_2 , \dots, X_n) \dots | X_n)) \\
		\end{split}\]
	This is a generalisation of the \keyword{partition rule} for conditional expectations.
}


\section*{Markov Chains}
\lectlink{https://imperial.cloud.panopto.eu/Panopto/Pages/Viewer.aspx?id=0a1c39d0-1364-48cb-803a-ae260066be68}
\termdef{Discrete Time Markov Chain (DTMC)}{
\compitem{
	\item A series of random variable modelling the state at a time step: $X_0, X_1, X_2, \dots$
	\item The state space $J$ (all states), where $J = sipp(X_i)$ (contains all states that we can be in at any step)
	\item We can take a sequence (sample path) through the states ($X_0, X_1, X_2, \dots$)
	\item We denote the state taken at step $n$ as state $J_n$
}

We use an initial probability vector $\pi$ to determine the start state:
\[\pi_0 = [\dots \text{ probability of starting in state i } \dots ]\]
We determine the probability of each next state through the transition probability matrix $r$:
\[r_{ij} = P(X_{n+1} = j | X_n = i)\]

For a markov chain the probability of being in any next state is \textbf{only} dependent on the current state (memoryless, history of previous states does not matter).
\[P(X_{i+1} = J_{n+1} | X_i = J_i) = P(X_{i+1} = J_{n+1} | X_i = J_i) = P(X_{i+1} = J_{n+1} | X_0 = J_0, \dots, X_i = J_i)\]

To get the probability we can use power of the matrix:
\[P(X_n = j | X_0 = i) = (R^n)_{ij}\]

If we have the initial probability vector we can calculate:
\[\begin{split}
		P(X_n = j) &= \sum_{i \in J}P(X_0 = i) \times P(X_n = j | X_0 = i) \\
		& = \sum_{i \in j}\pi_{0_i}(R^n)_{ij} \\
		&= (\pi_0R^n)_{ij} \\
	\end{split}\]

We can obtain the long term probabilities by using the $\infty$th step:
\[\lim_{t \to +\infty} \pi_0R^n = \pi_{\infty}\]
Note that since $\pi_{\infty}R = \pi_{\infty}$ we have eigenvector $\pi_{\infty}$ and eigenvalue $1$.
}

\example{Probabilistic Finite State Machine}{
	\centerimage{width=0.8\textwidth}{markov chain.png}
}
\example{Modelling Climate}{
	\centerimage{width=0.8\textwidth}{climate markov chain.png}
}









\end{document}
