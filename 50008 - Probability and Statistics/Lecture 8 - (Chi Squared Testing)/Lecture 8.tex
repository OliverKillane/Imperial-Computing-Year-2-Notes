\documentclass{report}
    \title{50008 - Probability and Statistics - Lecture 8}
    \author{Oliver Killane}
    \date{08/03/22}
%===========================COMMON FORMAT & COMMANDS===========================
% This file contains commands and format to be used by every module, and is 
% included in all files.
%===============================================================================

%====================================IMPORTS====================================
\usepackage[a4paper, total={6in, 8in}]{geometry}
\usepackage{graphicx, amssymb, amsfonts, amsmath, xcolor, listings, tcolorbox, multirow, hyperref}
%===============================================================================

%====================================IMAGES=====================================
\graphicspath{{image/}}

% \centerimage{options}{image}
\newcommand{\centerimage}[2]{\begin{center}
    \includegraphics[#1]{#2}
\end{center}}
%===============================================================================

%==============================SYNTAX HIGHLIGHTING==============================
\newcommand{\fun}[1]{\textcolor{blue}{\textbf{#1}}}
\newcommand{\file}[1]{\textcolor{green}{\textbf{#1}}}
\newcommand{\struct}[1]{\textcolor{orange}{\textbf{#1}}}
\newcommand{\var}[1]{\textcolor{purple}{\textbf{#1}}}
\newcommand{\const}[1]{\textcolor{red}{\textbf{#1}}}
%===============================================================================

%=================================CODE LISTINGS=================================
\definecolor{codebackdrop}{gray}{0.9}
\definecolor{commentgreen}{rgb}{0,0.6,0}
\lstset{
    inputpath=code, 
    commentstyle=\color{commentgreen},
    keywordstyle=\color{blue}, 
    backgroundcolor=\color{codebackdrop}, 
    basicstyle=\footnotesize,
    frame=single,
    numbers=left,
    stepnumber=1,
    showstringspaces=false,
    breaklines=true,
    postbreak=\mbox{\textcolor{red}{$\hookrightarrow$}\space}
}

% Create a code listing for a single line
% \codeline{language}{line}{file}
\newcommand{\codeline}[3]{\lstinputlisting[language=#1, firstline = #2, lastline = #2]{#3}}

% Create a code listing for multiple lines
% \codeline{language}{start}{end}{file}
\newcommand{\codelines}[4]{\lstinputlisting[language=#1, firstline = #2, lastline = #3]{#4}}


% Create a code listing for a given language & file
% \codelist{language}{file}
\newcommand{\codelist}[2]{\lstinputlisting[language=#1]{#2}}
%===============================================================================

%================================TEXT STRUCTURES================================
% Marka a word as bold
% \keyword{important word}
\newcommand{\keyword}[1]{\textbf{#1}}

% Creates a section in italics
% \question{question in italics}
\newcommand{\question}[1]{\textit{#1} \\ }

% Creates a box with title for side notes.
% \sidenote{title}{contents}
\newcommand{\sidenote}[2]{\begin{tcolorbox}[title=#1]#2\end{tcolorbox}}

\newcommand{\termdef}[2]{\begin{tcolorbox}[title=Definition: #1, colframe = blue]#2\end{tcolorbox}}

% Creates an item in an itemize or enumerate, with a paragraph after
% \begin{itemize}
%     \bullpara{title}{contents}
% \end{itemize}
\newcommand{\bullpara}[2]{\item \textbf{#1} \ #2}

% Creates a compact list (very small gaps between items)
% \compitem{
%     \item item 1
%     \item item 2
%     \item ...
% }
\newcommand{\compitem}[1]{\begin{itemize}\setlength\itemsep{-0.5em}#1\end{itemize}}
\newcommand{\compenum}[1]{\begin{enumerate}\setlength\itemsep{-0.5em}#1\end{enumerate}}

% Creates a link to the lecture for use at the start of the notes document
\newcommand{\lectlink}[1]{\sidenote{Lecture Recording}{
    Lecture recording is available \href{#1}{here}
}}
%===============================================================================

%================================STYLIZED PROOFS================================
%For ease in writing stylized proofs with numbers
\newcommand{\stepno}[1]{\textcolor{red}{\textbf{#1}}}

\newenvironment{proof}
{\begin{center}\begin{tabular}{r l l }}
{\end{tabular}\end{center}}

%\proofstep{step}{workings}{description}
\newcommand{\proofstep}[3]{\stepno{(#1)} & #2 & #3 \\}
%===============================================================================

%==============================UNFINISHED SECTION===============================
\newcommand{\unfinished}{\begin{huge} \textcolor{red}{\textbf{UNFINISHED!!!}} \end{huge}}
%===============================================================================
\usepackage{xfrac}
\begin{document}
    \maketitle
    
    \section*{Goodness of Fit}
        \lectlink{https://imperial.cloud.panopto.eu/Panopto/Pages/Viewer.aspx?id=a9ab82c4-178d-42c6-8031-ae4a00899447}
        \termdef{Binning}{
            Given a distribution, we can partition it into several disjoint \keyword{bins}. Essentially we are creating a pesudo-\keyword{PMF} (potentially with ranges instead of just discrete values) describing how many datapoints/the frequency we would expect to find from a distribution.
            \\
            \\ As a result, we can directly compare the expected values $E_i$ (from a distribution we are checking a sample against), with the observations $O_i$ from a sample.
            \centerimage{width=0.75\textwidth}{binning}
        }
        \termdef{Goodness of Fit/Chi-Square Statistic }{
            Denotes the difference between some expected values, and some observed.
            \\
            \\ For $n$ bins we have:
            \[X^2 = \sum_{i=1}^n\cfrac{(O_i-E_i)^2}{E_i}\]
        }
        \subsection*{Chi-Squared Test for Model Checking}
        
            Used to determine if an observed sample matches a given distribution to some significance.
            \begin{enumerate}
                \item Determine expected distribution (can use parameters estimated from the sample).
                \item Create a hypotheses based some parameters $\theta$: {
                    \[H_0 \ : \ \theta = \theta_0 \ \text{  versus  } \ H_1 \ : \ \theta \neq \theta_0\]
                }
                \item Bin the expected distribution (for comparison with the observed).
                \item Calculate the \keyword{Goodness of Fit/Chi-Square Test Statistic} $X^2$.
                \item Calculate the degrees of freedom as: {
                    \[\nu = (\text{number of possible values $X$ can take}) - (\text{number of parameters being estimated}) - 1\]}
                \item Determine the critical value using the \keyword{Chi Squared Distribution} $\chi^2_\nu$ and the significance $\alpha$ (typically using a table).
                \item If $X^2 > \chi^2_{\nu, \ 1 - \alpha}$ (test statistic larger than critical value)
            \end{enumerate}
            Note that:
            \compitem{
                \item All expected values must be larger than $5$ for a good test. Hence some bins may have to be merged.
                \item The number of values $X$ can take is typically the number of bins.
            }
            \centerimage{width=0.6\textwidth}{chi squared}
        
            \example{Adverse Drug Effects}{
                A study in the journal of the American Medical Association gives the causes of a sample of $95$ adverse drug effects as:
                \begin{center}
                    \begin{tabular}{l l}
                        \textbf{Reason} & \textbf{No. Adverse Effects} \\
                        \hline
                        Lack of Knowledge & $29$ \\
                        Rule Violation & $17$ \\
                        Faulty Dose Check & $13$ \\
                        Slips & $9$ \\
                        Other Cause & $27$ \\
                    \end{tabular}
                \end{center}
                Test if the true percentages of causes of adverse effects are different at the $5\%$ significance.
                \\
                \\ As we are checking the percentages are the same, we effectively have a discrete uniform distribution:
                \[X \thicksim U(1,5)\]
                Hence we can calculate our \keyword{null and alternative hypotheses}:
                \[H_0 \ : \ X \thicksim U(1,5) \ \text{  versus  } \ H_1 \ : \ X \not\thicksim U(1,5)\]
                Now we can bin the distribution, (no merging is required as all expected values are larger than $5$):
                \centerimage{width=0.6\textwidth}{ade example}
                It is now possible to compute goodness of fit.
                \[\begin{split}
                    X^2 &= \sum_{i=1}^n\cfrac{(O_i-E_i)^2}{E_i} \\
                    &= \cfrac{(29 - 19)^2}{19} + \cfrac{(17 - 19)^2}{19} + \cfrac{(13 - 19)^2}{19} + \cfrac{(9 - 19)^2}{19} + \cfrac{(27 - 19)^2}{19} \\
                    &= 16 \\
                \end{split}\]
                We have $\nu = 4$ as there are $5$ possible values, and no parameters were estimated from the data.
                \\
                \\ Hence we get the critical value from the chi-squared table: $\chi^2_{4, \ 0.95} = 9.49$
                \\
                \\ As $16 > 9.49$ there is sufficient evidence at the $5\%$ significance level to reject $H_0$, the percentages differ.
            }
            \lectlink{https://imperial.cloud.panopto.eu/Panopto/Pages/Viewer.aspx?id=b837a15e-d750-4e7b-975a-ae4a008d9f15}
            \example{Football Games}{
                Given the total number of goals for $2608$ football matches, determine if the number of goals scored in a match can be modelled by $X \thicksim Poisson(3.870)$ at the $5\%$ significance.
                \begin{center}
                    \begin{tabular}{l | c c c c c c c c c c c c}
                        Goals Scored ($x$) & 0 & 1 & 2 & 3 & 4 & 5 & 6 & 7 & 8 & 9 & $\geq 10$ \\
                        \hline
                        Matches ($n_x$) & 57 & 203 & 383 & 525 & 532 & 408 & 273 & 139 & 139 & 45 & 27 & 16 \\
                    \end{tabular}
                \end{center}
                Hence as we already have a distribution, we can create our hypotheses:
                \[H_0 \ : \ X \thicksim Poisson(3.870) \ \text{  versus  } \ H_1 \ : \ X \not\thicksim Poisson(3.87)\]
                We can then use the poisson distribution to calculate the expected for $2608$ football matches, for the final ($\geq 10$) we use the cumulative to get the remaining probability.
                \begin{center}
                    \begin{tabular}{l | c c c c c c c c c c c c}
                        Goals & 0 & 1 & 2 & 3 & 4 & 5 & 6 & 7 & 8 & 9 & $\geq 10$ \\
                        \hline
                        $O$ & 57 & 203 & 383 & 525 & 532 & 408 & 273 & 139 & 45 & 27 & 16 \\
                        $E$ & 54.4 & 210.5 & 407.4 & 525.5 & 508.4 & 393.5 & 253.8 & 140.3 & 67.9 & 29.2 & 17.1 \\
                        $\cfrac{(O - E)^2}{E}$ & 0.124 & 0.267 & 1.461 & 0.000 & 1.096 & 0.534 & 1.452 & 0.012 & 7.723 & 0.166 & 0.071 \\
                    \end{tabular}
                \end{center}
                Hence we get our test statistic as: $X^2 = 12.906$.
                \\
                \\ As we did not estimate any parameters from the sample, the degrees of freedom are $\nu = 11 - 1 = 10$.
                \\
                \\ The critical value is: $\chi^2_{10, \ 0.95} = 16.91$.
                \\
                \\ Hence as $12.906 < 16.91$ we there is insufficient evidence as the $5\%$ significance to reject $H_0$, the goals can be modelled as $Poisson(3.87)$.
            }
        \subsection*{Chi-Squared Test for Independence}
            \lectlink{https://imperial.cloud.panopto.eu/Panopto/Pages/Viewer.aspx?id=423cd2de-13bc-4e47-a6d6-ae4a008efe06}
            \termdef{Contingency Table}{
                A table denoting the frequency of each combination of values for $X$ and $Y$.
                \begin{center}
                    \begin{tabular}{l c | c c c c | c}
                        & & \multicolumn{4}{c|}{Possible values of $y$} & Marginal\\
                        &         & $y_1$     & $y_2$     & $\dots$ & $y_l$ & \\
                        \hline
                        \multirow{4}{*}{Possible $x$} &$x_1$    & $n_{1,1}$ & $n_{1,2}$ & $\dots$  & $n_{1,l}$ & $n_{1, \bullet}$ \\
                                                    &$x_2$    & $n_{2,1}$ & $n_{2,2}$ & $\dots$  & $n_{2,l}$ & $n_{2, \bullet}$ \\
                                                    &$\vdots$ & $\vdots$  & $\vdots$  & $\ddots$ & $\vdots$  & $\vdots$ \\
                                                    &$x_k$    & $n_{k,1}$ & $n_{k,2}$ & $\dots$  & $n_{k,l}$ & $n_{k, \bullet}$ \\
                        \hline
                        Marginal & & $n_{\bullet, 1}$ & $n_{\bullet, 2}$ & $\dots$ & $n_{\bullet, l}$ & $n$ \\
                    \end{tabular}
                \end{center}
                We can use the marginal values to determine the expected value, if the two distributions were independent.
            }
            Given a dataset of points $(x,y)_1, (x,y)_2, \dots, (x,y)_n$, we can consider it the joint distribution $P_{XY}$ of the distributions $P_X$ and $P_Y$.
            \\
            \\ To test if the distributions $P_X$ and $P_Y$ are independent from the sample (without knowing the actual distributions themselves) we can use a \keyword{contingency table}.
            \\
            \\ For the contingency table entry coordinates $0 < i \leq l$, $0 < j \leq k$:
            \[O_{i,j} = n_{i,j} \ \text{ and } E_{i,j} = \cfrac{n_{i, \bullet} \times n_{\bullet, j}}{n}\]
            Hence we can now compute the $X^2$ (\keyword{Chi Squared test statistic}) using these observed and expected values.
            \\
            \\ We compute the degrees of freedom as $\nu = (rows - 1) \times (columns - 1)$ (each row and column alone has degrees of freedom $n-1$ as they must sum to the row/column total), and can then do the \keyword{Chi-Squared Test} normally.
            \example{Fitness and Stress}{
                \begin{center}
                    \begin{tabular}{l | c c c | c}
                        & Poor Fitness & Average Fitness & Good Fitness & \\
                        \hline
                        Stress & 206 & 184 & 85 & 475 \\
                        No Stress & 36 & 28 & 10 & 74 \\
                        \hline
                        & 242 & 212 & 95 & 549 \\
                    \end{tabular}
                \end{center}
                Determine at the $5\%$ significance if there is a link between fitness and stress.
                \\
                \\ For this test the null hypothesis will be that fitness and stress are independent.
                \[H_0 \ : \ \text{Stress and fitness are independent} \ \text{  versus  } \ H_1 \ : \ \text{Stress and Fitness re not independent}\]
                Next we can calculate the expected values:
                \begin{center}
                    \begin{tabular}{l | c c c c c c | c}
                        & \multicolumn{2}{c}{Poor Fitness} & \multicolumn{2}{c}{Average Fitness} & \multicolumn{2}{c}{Good Fitness} & \\
                        & $O$ & $E$                        & $O$ & $E$                           & $O$ & $E$                        & \\
                        \hline
                        Stress    & \textcolor{blue}{206} & \textcolor{red}{209.4} & \textcolor{blue}{184} & \textcolor{red}{183.4} & \textcolor{blue}{85} & \textcolor{red}{82.2} & 475 \\
                        No Stress & \textcolor{blue}{36}  & \textcolor{red}{32.6}  & \textcolor{blue}{28}  & \textcolor{red}{28.6}  & \textcolor{blue}{10} & \textcolor{red}{12.8} & 74  \\
                        \hline
                        & \multicolumn{2}{c}{242} & \multicolumn{2}{c}{212} & \multicolumn{2}{c}{95} & 549 \\
                    \end{tabular}
                \end{center}
                We can then calculate our test statistic to be $X^2 = 1.133$.
                \\
                \\ To compute the degrees of freedom $\nu = (2 - 1) \times (3 - 1) = 2$.
                \\
                \\ Hence we can get our critical value $\chi^2_{2, \ 0.95} = 5.99$.
                \\
                \\ As $5.99 > 1.133$, there is insufficient evidence to reject $H_0$ at the $5\%$ significance level. Stress and fitness are not linked.
            }
            



\end{document}
