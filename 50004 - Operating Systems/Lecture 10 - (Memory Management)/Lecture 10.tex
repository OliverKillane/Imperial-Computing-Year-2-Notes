\documentclass{report}
    \title{50004 - Operating Systems - Lecture 10}
    \author{Oliver Killane}
    \date{22/11/21}

%===========================COMMON FORMAT & COMMANDS===========================
% This file contains commands and format to be used by every module, and is 
% included in all files.
%===============================================================================

%====================================IMPORTS====================================
\usepackage[a4paper, total={6in, 8in}]{geometry}
\usepackage{graphicx, amssymb, amsfonts, amsmath, xcolor, listings, tcolorbox, multirow, hyperref}
%===============================================================================

%====================================IMAGES=====================================
\graphicspath{{image/}}

% \centerimage{options}{image}
\newcommand{\centerimage}[2]{\begin{center}
    \includegraphics[#1]{#2}
\end{center}}
%===============================================================================

%=================================CODE LISTINGS=================================
\definecolor{codebackdrop}{gray}{0.9}
\definecolor{commentgreen}{rgb}{0,0.6,0}
\lstset{
    inputpath=code, 
    commentstyle=\color{commentgreen},
    keywordstyle=\color{blue}, 
    backgroundcolor=\color{codebackdrop}, 
    basicstyle=\footnotesize,
    frame=single,
    numbers=left,
    stepnumber=1,
    showstringspaces=false,
    breaklines=true,
    postbreak=\mbox{\textcolor{red}{$\hookrightarrow$}\space}
}

% Create a code listing for a single line
% \codeline{language}{line}{file}
\newcommand{\codeline}[3]{\lstinputlisting[language=#1, firstline = #2, lastline = #2]{#3}}

% Create a code listing for a given language & file
% \codelist{language}{file}
\newcommand{\codelist}[2]{\lstinputlisting[language=#1]{#2}}
%===============================================================================

%================================TEXT STRUCTURES================================
% Marka a word as bold
% \keyword{important word}
\newcommand{\keyword}[1]{\textbf{#1}}

% Creates a section in italics
% \question{question in italics}
\newcommand{\question}[1]{\textit{#1} \\ }

% Creates a box with title for side notes.
% \sidenote{title}{contents}
\newcommand{\sidenote}[2]{\begin{tcolorbox}[title=#1]#2\end{tcolorbox}}

% Creates an item in an itemize or enumerate, with a paragraph after
% \begin{itemize}
%     \bullpara{title}{contents}
% \end{itemize}
\newcommand{\bullpara}[2]{\item \textbf{#1} \ #2}

% Creates a compact list (very small gaps between items)
% \compitem{
%     \item item 1
%     \item item 2
%     \item ...
% }
\newcommand{\compitem}[1]{\begin{itemize}\setlength\itemsep{-0.5em}#1\end{itemize}}

% Creates a link to the lecture for use at the start of the notes document
\newcommand{\lectlink}[1]{\sidenote{Lecture Recording}{
    Lecture recording is available \href{#1}{here}
}}
%===============================================================================


%==============================SYNTAX HIGHLIGHTING==============================
\newcommand{\fun}[1]{\textcolor{blue}{\textbf{#1}}}
\newcommand{\file}[1]{\textcolor{green}{\textbf{#1}}}
\newcommand{\struct}[1]{\textcolor{orange}{\textbf{#1}}}
\newcommand{\var}[1]{\textcolor{purple}{\textbf{#1}}}
\newcommand{\const}[1]{\textcolor{red}{\textbf{#1}}}
%===============================================================================

%==============================THREAD HIGHLIGHTING==============================
\newcommand{\threada}[1]{\textcolor{green}{\textbf{#1}}}
\newcommand{\threadb}[1]{\textcolor{red}{\textbf{#1}}}
%===============================================================================

%============================DISPLAYING THREAD CODE=============================
\newcommand{\threadlist}[3]{
    \codelist{#1}{#2 init.#3}
    \begin{minipage}[t]{0.45\textwidth}
        \codelist{#1}{#2 A.#3}
    \end{minipage}
    \hfill
    \begin{minipage}[t]{0.45\textwidth}
        \codelist{#1}{#2 B.#3}
    \end{minipage}
}
%===============================================================================

\begin{document}
\maketitle
\lectlink{https://imperial.cloud.panopto.eu/Panopto/Pages/Viewer.aspx?id=7bafd1b7-0c00-4a35-9bf1-ade400bd5509}

\section*{Linux - Virtual Memory Layout}
\centerimage{width=0.8\textwidth}{linux VM layout.png}
\begin{itemize}
	\bullpara{$1:1$ Mapping}{
	\\ Can turn logical address to physical address for kernel pages by subtracting $3$GB.
	\compitem{
	\item Very efficient for kernel memory access
	\item Does not require change of page table (not changing context such as in process switching) (so TLB is not flushed when user process makes syscall).
	\item On Demand Mapping contains temporary mappings for use of more than $896$ MB of memory.
	      }
	      }
\end{itemize}
\begin{minipage}[t]{0.5\textwidth}
	\subsubsection*{Typically on IA32}
	\compitem{
		\item $4$KB page size.
		\item $4$GB virtual address space.
		\item Two-level page table (up to 3 with \keyword{Physical Address Extension}).
		\item Offset bits contain page status (dirty, readonly etc).
	}
\end{minipage}
\begin{minipage}[t]{0.5\textwidth}
	\subsubsection*{On AMD64/x86\_64}
	\compitem{
		\item Larger page sizes (e.g $4$MB).
		\item Up to four-level page table.
		\item Offset bits can contain can-execute (prevent malicious code being written to take over process).
	}
\end{minipage}

\section*{Meltdown Attack}
\codelist{C}{meltdown.c}
By speculative execution, this becomes:
\codelist{C}{meltdown speculation.c}

\section*{Demand Paging}
Only load pages from swap when the user attempts to access them.
\compitem{
	\item Lower I/O load (unused pages are never loaded)
	\item Less memory required (fewer pages resident in memory)
	\item Faster response time (Can start executing straight away, does not need to wait for all pages to be loaded first)
	\item Supports more users (lower memory usage allows this)
}
Uses a valid-invalid bit such that:
\[\begin{matrix}
		1 \Rightarrow & \text{in memory}     \\
		0 \Rightarrow & \text{not in memory} \\
	\end{matrix}\]
\compitem{
	\item All page entries are originally set to $0$.
	\item If a page with bit $0$ is accessed, page fault and trap to kernel.
	\item Kernel uses other table to determine if reference is invalid, or valid but page not in memory.
}
To handle a valid request the kernel:
\compenum{
	\item Get empty frame
	\item swap page to frame
	\item Reset tables (validation bit $= 1$)
	\item Restart last instruction
}

\subsection*{Performance}
Given a \textbf{Page Fault Rate} ($=0 \Rightarrow Never$, $=1 \Rightarrow Always$).
\[\text{Effective Access Time} = (1-p) \times \text{memory access time} + p \times (\begin{matrix}
			\text{Page Fault overhead} \\ + \\
			\text{swap page out}       \\ + \\
			\text{swap page in}        \\ + \\
			\text{restart overhead}    \\ + \\
			\text{memory access time}  \\
		\end{matrix})\]

\section*{Virtual Memory Tricks}
\begin{itemize}
	\bullpara{Copy-on-Write (COW)}{
	\\ Processes accessing identical pages use the same frame, only copy when a process wants to write/modify the page.
	\compitem{
	\item Parent and Child processes can initially share same pages in memory
	\item Efficient process creation (copy only modified pages)
	\item Free pages allopcated from a pool of zero-ed out pages
	      }

	      For example with \fun{fork}:
	      \compitem{
	\item Child's page table points to parent's pages (marked as read-only in child \& parent's page table)
	\item Protection fault causes trap by kernel.
	\item Kernel allocated new copy of the page to process alterning (that it can write to), replaces old page in page table.
	\item Both child and parent's page table sets page to Read-Write.
	      }
	      }
	      \bullpara{Memory Mapped Files}{
		      \\ Map files into the virtual address space using paging. Only need to load parts of a file when they are accessed.
		      \\
		      \\ e.g 16K page file, only access first 2 pages, so only first 2 loaded to memory.
		      \\
		      \\ Simplified programming model for I/O (can easily access stdin/out).
	      }
\end{itemize}

\section*{Page Replacement}
When out of free memory \& a new page must be created, a page must be swapped out.
\\ An example below (note victim does not have to be a page of the same process).
\centerimage{width=\textwidth}{page replacement.png}
\compenum{
	\item Access a page not loaded into memory.
	\item Page table contains bit to show page is not loaded.
	\item Update page table (must ensure no race conditions when involving multiple processes that may be running on multiple cores).
	\item Write victim to disk (swap space).
	\item Read page from disk.
	\item Restart operation, can now access page.
}
The goal is to:
\begin{itemize}
	\bullpara{Reduce the number of page faults}{\\ Avoid bringing a page back into memory many times \& in general more frames $\Rightarrow$ fewer page faults.}
	\bullpara{Prevent over-allocation of memory}{\\ Page-fault service routine should include page replacement.}
	\bullpara{Reduce redundant I/O}{\\ Use modify (dirty bit) to only load modified pages back to disk.}
\end{itemize}

\subsection*{Replacement Algorithms}
\subsubsection*{FIFO}
Replace the oldest page.
\\
\\\begin{minipage}[t]{0.5\textwidth}
	\centerline{\textbf{Advantages}}
	\compitem{
		\item Simple to implement
	}

\end{minipage}
\begin{minipage}[t]{0.5\textwidth}
	\centerline{\textbf{Disadvantages}}
	\compitem{
		\item May replace a heavily used page.
	}
\end{minipage}

Belady's Anomaly (where more frames result in more page faults), for example:
\\ \textbf{3 frames, 9 faults}
\begin{center}
	\begin{tabular}{l c c c c c c c c c c c c}
		Access:     & 1 & 2 & 3 & 4 & 1 & 2 & 5 & 1     & 2     & 3 & 4 & 5     \\
		Page Fault: & Y & Y & Y & Y & Y & Y & Y & N     & N     & Y & Y & N     \\
		Frame 0:    & 1 & 1 & 1 & 4 & 4 & 4 & 5 & $\to$ & $\to$ & 5 & 5 & $\to$ \\
		Frame 1:    & N & 2 & 2 & 2 & 1 & 1 & 1 & $\to$ & $\to$ & 3 & 3 & $\to$ \\
		Frame 2:    & N & N & 3 & 3 & 3 & 2 & 2 & $\to$ & $\to$ & 2 & 4 & $\to$ \\
	\end{tabular}
\end{center}
\textbf{4 frames, 10 faults}
\begin{center}
	\begin{tabular}{l c c c c c c c c c c c c}
		Access:     & 1 & 2 & 3 & 4 & 1     & 2     & 5 & 1 & 2 & 3 & 4 & 5 \\
		Page Fault: & Y & Y & Y & Y & N     & N     & Y & Y & Y & Y & Y & Y \\
		Frame 0:    & 1 & 1 & 1 & 1 & $\to$ & $\to$ & 5 & 5 & 5 & 5 & 4 & 4 \\
		Frame 1:    & N & 2 & 2 & 2 & $\to$ & $\to$ & 2 & 1 & 1 & 1 & 1 & 5 \\
		Frame 2:    & N & N & 3 & 3 & $\to$ & $\to$ & 3 & 3 & 2 & 2 & 2 & 2 \\
		Frame 3:    & N & N & N & 4 & $\to$ & $\to$ & 4 & 4 & 4 & 3 & 3 & 3 \\
	\end{tabular}
\end{center}
Here the reference string is such that under \keyword{FIFO} at 4 frames the next page is frequently the oldest, resulting in a high number of page faults.
\subsubsection*{Optimal Algorithm}
Replace the page that will not be used for the longest period of time. This is impossible in practice, but can be used as a benchmark for comparing other algorithms.
\\
\\ Note that in the example, pages never used again are the longest away \& lower frames are chosen when pages have equal time.
\\ \textbf{3 frames, 7 page faults}
\begin{center}
	\begin{tabular}{l c c c c c c c c c c c c}
		Access:     & 1 & 2 & 3 & 4 & 1     & 2     & 5 & 1     & 2     & 3 & 4 & 5     \\
		Page Fault: & Y & Y & Y & Y & N     & N     & Y & N     & N     & Y & Y & N     \\
		Frame 0:    & 1 & 1 & 1 & 1 & $\to$ & $\to$ & 1 & $\to$ & $\to$ & 3 & 4 & $\to$ \\
		Frame 1:    & N & 2 & 2 & 2 & $\to$ & $\to$ & 2 & $\to$ & $\to$ & 2 & 2 & $\to$ \\
		Frame 2:    & N & N & 3 & 4 & $\to$ & $\to$ & 5 & $\to$ & $\to$ & 5 & 5 & $\to$ \\
	\end{tabular}
\end{center}












\end{document}
