\documentclass{report}
    \title{50004 - Operating Systems - Lecture 5}
    \author{Oliver Killane}
    \date{02/11/21}

%===========================COMMON FORMAT & COMMANDS===========================
% This file contains commands and format to be used by every module, and is 
% included in all files.
%===============================================================================

%====================================IMPORTS====================================
\usepackage[a4paper, total={6in, 8in}]{geometry}
\usepackage{graphicx, amssymb, amsfonts, amsmath, xcolor, listings, tcolorbox, multirow, hyperref}
%===============================================================================

%====================================IMAGES=====================================
\graphicspath{{image/}}

% \centerimage{options}{image}
\newcommand{\centerimage}[2]{\begin{center}
    \includegraphics[#1]{#2}
\end{center}}
%===============================================================================

%=================================CODE LISTINGS=================================
\definecolor{codebackdrop}{gray}{0.9}
\definecolor{commentgreen}{rgb}{0,0.6,0}
\lstset{
    inputpath=code, 
    commentstyle=\color{commentgreen},
    keywordstyle=\color{blue}, 
    backgroundcolor=\color{codebackdrop}, 
    basicstyle=\footnotesize,
    frame=single,
    numbers=left,
    stepnumber=1,
    showstringspaces=false,
    breaklines=true,
    postbreak=\mbox{\textcolor{red}{$\hookrightarrow$}\space}
}

% Create a code listing for a single line
% \codeline{language}{line}{file}
\newcommand{\codeline}[3]{\lstinputlisting[language=#1, firstline = #2, lastline = #2]{#3}}

% Create a code listing for a given language & file
% \codelist{language}{file}
\newcommand{\codelist}[2]{\lstinputlisting[language=#1]{#2}}
%===============================================================================

%================================TEXT STRUCTURES================================
% Marka a word as bold
% \keyword{important word}
\newcommand{\keyword}[1]{\textbf{#1}}

% Creates a section in italics
% \question{question in italics}
\newcommand{\question}[1]{\textit{#1} \\ }

% Creates a box with title for side notes.
% \sidenote{title}{contents}
\newcommand{\sidenote}[2]{\begin{tcolorbox}[title=#1]#2\end{tcolorbox}}

% Creates an item in an itemize or enumerate, with a paragraph after
% \begin{itemize}
%     \bullpara{title}{contents}
% \end{itemize}
\newcommand{\bullpara}[2]{\item \textbf{#1} \ #2}

% Creates a compact list (very small gaps between items)
% \compitem{
%     \item item 1
%     \item item 2
%     \item ...
% }
\newcommand{\compitem}[1]{\begin{itemize}\setlength\itemsep{-0.5em}#1\end{itemize}}

% Creates a link to the lecture for use at the start of the notes document
\newcommand{\lectlink}[1]{\sidenote{Lecture Recording}{
    Lecture recording is available \href{#1}{here}
}}
%===============================================================================


%==============================SYNTAX HIGHLIGHTING==============================
\newcommand{\fun}[1]{\textcolor{blue}{\textbf{#1}}}
\newcommand{\file}[1]{\textcolor{green}{\textbf{#1}}}
\newcommand{\struct}[1]{\textcolor{orange}{\textbf{#1}}}
\newcommand{\var}[1]{\textcolor{purple}{\textbf{#1}}}
\newcommand{\const}[1]{\textcolor{red}{\textbf{#1}}}
%===============================================================================

%==============================THREAD HIGHLIGHTING==============================
\newcommand{\threada}[1]{\textcolor{green}{\textbf{#1}}}
\newcommand{\threadb}[1]{\textcolor{red}{\textbf{#1}}}
%===============================================================================

%============================DISPLAYING THREAD CODE=============================
\newcommand{\threadlist}[3]{
    \codelist{#1}{#2 init.#3}
    \begin{minipage}[t]{0.45\textwidth}
        \codelist{#1}{#2 A.#3}
    \end{minipage}
    \hfill
    \begin{minipage}[t]{0.45\textwidth}
        \codelist{#1}{#2 B.#3}
    \end{minipage}
}
%===============================================================================

\begin{document}
    \maketitle
    \sidenote{Lecture Recording}{
        Lecture recording is available \href{https://imperial.cloud.panopto.eu/Panopto/Pages/Viewer.aspx?id=d4c3f07f-16e0-4975-90cf-add400e5d365}{here}
    }

    \section*{Process Synchronisation}
        The key concepts to consider in process synchronisation are:
        \begin{itemize}
            \bullpara{Critical Sections}{
                \\ Section of code where processes access a shared resource
            }
            \bullpara{Mutual Exclusion}{
                \\ Multiple threads/processes cannot execute in a critical section simultaneously.
            }
            \bullpara{Atomic Operations}{
                \\ Operations that can occur without interruption or interleve by other threads, e.g reading and writing in 1 instruction.
            }
            \bullpara{Race Conditions}{
                \\ Where the behaviour of the program is dependent on the timing and interleving of threads.
            }
            \bullpara{Deadlock}{
                \\ Where the completion of tasks are mutually dependent, there is some loop in the dependency graph meaning a task's completion cannot occur before its completion.
            }
            \bullpara{Starvation}{}
            \bullpara{Synchronisation Mechanisms}{
                \\ Locks, Semaphores and Monitors for example.
                \\
                \\ Primitive that allow programmers to enfoce ordering of tasks, and exclusion. They are required at entry to and exit from a critical section.
            }
        \end{itemize}
    \section*{Requirements of Mutual Exclusion}
        \begin{itemize}
            \item No two processes can simultaneously be in a critical section.
            \item No process running outside the critical section can prevent another from entering the critical section.
            \item When no process is in the critical region, any process requesting permission to enter must be immediately admitted.
            \item No process requiring access to a critical section can be delayed indefinitely.
            \item No assumptions are made about the relative speed of processes.
        \end{itemize}
    \section*{Disabling Interrupts}
        \codelist{C}{intr disable.c}
        \begin{itemize}
            \item Works only on single-core Systems (other core may access shared in memory).
            \item May never release the CPU if code buggy.
            \item Interrupts could be missed while interrupts disabled.
            \item Only possible in kernel code which has the ability to disable interrupts.
        \end{itemize}
    \section*{Software Solution - Strict Alternative}
        \begin{minipage}[t]{0.5\textwidth}
            \codelist{C}{turn0.c}
        \end{minipage}
        \begin{minipage}[t]{0.5\textwidth}
            \codelist{C}{turn1.c}
        \end{minipage}
        Issues:
        \begin{itemize}
            \item Cannot run $0$ or $1$ twice in a row without the other working.
            \item Busy waiting of thread on turn variable.
            \item If the noncritical section takes a long time, the other thread is blocked as it must wait for turn to be set. (Thread outside critical region prevent critical region access).
            \item Turn must be volatile to prevent compiler optimisations causing issues.
        \end{itemize}
        \sidenote{Busy Waiting}{
            Continuously checking a value to determine if a thread can progress. It is a waste of CPU time and should only be used if the wait is short.
            \\
            \\ However as a result of actively polling the a value, it can stop waiting very quickly. Spinlocks used in OS kernels take advantage of this.
        }
    \section*{Peterson's Solution}
        \begin{minipage}[t]{0.7\textwidth}
            \codelist{C}{petersons.c}
        \end{minipage}
        \begin{minipage}[t]{0.3\textwidth}
            \codelist{C}{petersons1.c}
            \codelist{C}{petersons0.c}
        \end{minipage}
        This lock busy waits while the the other is interested and it is the other's turn.
    
    \section*{Atomic Operations}
        \begin{minipage}[t]{0.4 \textwidth}
            \codelist{C}{extract.c}
            Not atomic, other extract could be interleaved.
        \end{minipage}
        \begin{minipage}[t]{0.2 \textwidth}
            \begin{huge}
                \[\Rightarrow\]
            \end{huge}
        \end{minipage}
        \begin{minipage}[t]{0.4 \textwidth}
            \codelist{C}{extract fake atomic.c}
            Not atomic, still requires multiple instructions, between which another process could be interleaved.
        \end{minipage}
    \section*{Locks}
        \begin{minipage}[t]{0.4\textwidth}
            \codelist{C}{extract locks.c}
        \end{minipage}
        \begin{minipage}[t]{0.2 \textwidth}
            \begin{huge}
                \[\Leftarrow\]
            \end{huge}
        \end{minipage}
        \begin{minipage}[t]{0.4\textwidth}
            \codelist{C}{bad busy lock.c}
            \codelist{C}{unlock.c}
        \end{minipage}
        Here the implementation of the lock is flawed as we cannot atomically check the value of the lock, and set its value.
        \\
        \\ A working implementation is:
        \codelist{C}{lock atomic.c}
        TSL (test-and-set-lock) atomically sets the given memory location to $1$ and returns the old value. Locks of this type are called \keyword{spin locks}.
    \section*{Spin Locks}
        Spin locks should only be used when the expected wait time is short (otherwise wasteful). However it may run into the priority inversion problem.
        \sidenote{Priority Inversion}{
            When a lower priority thread is scheduled instead of a higher priority thread.
            \\
            \\ Threads $H > M > L$.
            \\ $L$ acquires a lock, however is descheduled. $H$ attempts tpo acquire the lock, but is blocked. The lock is not released as $L$ needs to run in order to release it. $M$ is higher priority than $L$, and runs instead.

        }
\end{document}
