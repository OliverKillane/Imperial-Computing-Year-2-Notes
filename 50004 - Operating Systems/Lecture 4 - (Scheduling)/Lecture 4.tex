\documentclass{report}
    \title{50004 - Operating Systems - Lecture 4}
    \author{Oliver Killane}
    \date{23/10/21}

%===========================COMMON FORMAT & COMMANDS===========================
% This file contains commands and format to be used by every module, and is 
% included in all files.
%===============================================================================

%====================================IMPORTS====================================
\usepackage[a4paper, total={6in, 8in}]{geometry}
\usepackage{graphicx, amssymb, amsfonts, amsmath, xcolor, listings, tcolorbox, multirow, hyperref}
%===============================================================================

%====================================IMAGES=====================================
\graphicspath{{image/}}

% \centerimage{options}{image}
\newcommand{\centerimage}[2]{\begin{center}
    \includegraphics[#1]{#2}
\end{center}}
%===============================================================================

%=================================CODE LISTINGS=================================
\definecolor{codebackdrop}{gray}{0.9}
\definecolor{commentgreen}{rgb}{0,0.6,0}
\lstset{
    inputpath=code, 
    commentstyle=\color{commentgreen},
    keywordstyle=\color{blue}, 
    backgroundcolor=\color{codebackdrop}, 
    basicstyle=\footnotesize,
    frame=single,
    numbers=left,
    stepnumber=1,
    showstringspaces=false,
    breaklines=true,
    postbreak=\mbox{\textcolor{red}{$\hookrightarrow$}\space}
}

% Create a code listing for a single line
% \codeline{language}{line}{file}
\newcommand{\codeline}[3]{\lstinputlisting[language=#1, firstline = #2, lastline = #2]{#3}}

% Create a code listing for a given language & file
% \codelist{language}{file}
\newcommand{\codelist}[2]{\lstinputlisting[language=#1]{#2}}
%===============================================================================

%================================TEXT STRUCTURES================================
% Marka a word as bold
% \keyword{important word}
\newcommand{\keyword}[1]{\textbf{#1}}

% Creates a section in italics
% \question{question in italics}
\newcommand{\question}[1]{\textit{#1} \\ }

% Creates a box with title for side notes.
% \sidenote{title}{contents}
\newcommand{\sidenote}[2]{\begin{tcolorbox}[title=#1]#2\end{tcolorbox}}

% Creates an item in an itemize or enumerate, with a paragraph after
% \begin{itemize}
%     \bullpara{title}{contents}
% \end{itemize}
\newcommand{\bullpara}[2]{\item \textbf{#1} \ #2}

% Creates a compact list (very small gaps between items)
% \compitem{
%     \item item 1
%     \item item 2
%     \item ...
% }
\newcommand{\compitem}[1]{\begin{itemize}\setlength\itemsep{-0.5em}#1\end{itemize}}

% Creates a link to the lecture for use at the start of the notes document
\newcommand{\lectlink}[1]{\sidenote{Lecture Recording}{
    Lecture recording is available \href{#1}{here}
}}
%===============================================================================


%==============================SYNTAX HIGHLIGHTING==============================
\newcommand{\fun}[1]{\textcolor{blue}{\textbf{#1}}}
\newcommand{\file}[1]{\textcolor{green}{\textbf{#1}}}
\newcommand{\struct}[1]{\textcolor{orange}{\textbf{#1}}}
\newcommand{\var}[1]{\textcolor{purple}{\textbf{#1}}}
\newcommand{\const}[1]{\textcolor{red}{\textbf{#1}}}
%===============================================================================

%==============================THREAD HIGHLIGHTING==============================
\newcommand{\threada}[1]{\textcolor{green}{\textbf{#1}}}
\newcommand{\threadb}[1]{\textcolor{red}{\textbf{#1}}}
%===============================================================================

%============================DISPLAYING THREAD CODE=============================
\newcommand{\threadlist}[3]{
    \codelist{#1}{#2 init.#3}
    \begin{minipage}[t]{0.45\textwidth}
        \codelist{#1}{#2 A.#3}
    \end{minipage}
    \hfill
    \begin{minipage}[t]{0.45\textwidth}
        \codelist{#1}{#2 B.#3}
    \end{minipage}
}
%===============================================================================

\begin{document}
    \maketitle
    \sidenote{Lecture Recording}{
        Lecture recording is available \href{https://imperial.cloud.panopto.eu/Panopto/Pages/Viewer.aspx?id=8ce2ab3e-8ad5-49a3-b391-adc800a6efb6}{here}
    }

    \section*{Scheduling}
        \subsubsection*{Process States}
            \centerimage{width=0.8\textwidth}{process states.png}
            Scheduling concerns determining which ready process should be run, when and for how long.
        \subsubsection*{Scheduling Goals}
            \begin{itemize}
                \bullpara{Fairness}{Comparable processes get comparable resource allocation (e.g CPU time).}
                \bullpara{Avoid Indefinite Postponement}{No process should starve (be left with no CPU time at all, effectively not running at all).}
                \bullpara{Enforce Policy}{For example enforcing certain processes (e.g system processe) are prioritised.}
                \bullpara{Maximize Resource Utilisation}{If there is a process that can be run, it should be run. Resources go beyond CPU time to resouces such as disk, or I/O.}
                \bullpara{Minimise Overhead}{The scheduling system itself should use as little resources as possible. (e.g minimise scheduling decisions and frequency of context switches).}
            \end{itemize}
            There are different priorities for different types of systems
        \subsubsection*{Scheduler Types \& Systems}
            \begin{itemize}
                \bullpara{Batch Systems}{Optimise for throughput (jobs per unit time) or turnaround time (time from submission to completion).}
                \bullpara{Interactive Systems}{Response Time is crucial (time from request issued to first response).}
                \bullpara{Real Time Systems}{Need to meet deadlines. 
                \begin{tabular}{l}
                    Soft: (recoverable) e.g reduced video quality. \\
                    Hard: (irrecoverable) e.g factory robots collide.}
                \end{tabular}
            \end{itemize}
            There are two main types of scheduler
            \begin{itemize}
                \bullpara{Non-Preemptive}{
                    \compitem{
                        \item Processes are allowed to continue to run until they become blocked, or voluntarily yield the CPU.
                        \item Typically this requires software to be trusted to not monopolise the CPU maliciously.
                        \item Bad for interactivity as a process may be waiting for a long time.
                        \item Good for batch systems as sheduler overhead is typically small.
                    }
                }
                \bullpara{Preemeptive}{
                    \compitem{
                        \item Processes can run for a maximum amount of time (time slice) before being descheduled.
                        \item Requires a clock interrupt to allow the OS to take control from the process.
                        \item Used often in interactive systems as longest wait is guarenteed to be $(\text{processes} - 1) \times \text{time slice}$.
                    }
                }
            \end{itemize}
        \subsubsection*{CPU vs I/O Bound}
            \textbf{CPU Bound}
            \compitem{
                \item Spend most of their time using the CPU (e.g data processing).
            }
            \textbf{I/O Bound}
            \compitem{
                \item Spend most of their time waiting for I/O.
                \item Use the CPU briefly before issuing an I/O request.
            }
    \section*{Schedulers}
        \subsection*{FCFS}
            \centerimage{width=0.8\textwidth}{FCFS Non-Preemeptive.png}
            \textbf{Advantages}
            \compitem{
                \item All processes are eventually scheduled.
                \item Very simple \& easy to implement.
                \item Very fast (minimal scheduling overhead).
            }
            \textbf{Disadvantages}
            \\ scheduler is unfair, for example:
            \centerimage{width=\textwidth}{FCFS Non-preemptive disadvantage.png}
        \subsubsection*{Round Robin Scheduling}
            Much like \keyword{FCFS}, however a process is preempted and added back to the ready queue when its time quantum expires (time slice).
            \centerimage{width=\textwidth}{RR scheduling.png}
            \compitem{
                \item Fair (ready jobs get equal CPU time).
                \item Response time is good for a small number of jobs.
                \item Turnaround time is low when run times differ (smaller jobs can finish quickly), but poor for similar runtimes (constant interruption \& sent to back of ready queue).
            }

            \[\text{overhead} = \cfrac{\text{context switch}}{\text{context switch} + \text{quantum}}\]
            \[\text{worst response} = \text{number of processes} \times \text{quantum}\]
            Hence the length of time quantum is a tradeoff between overhead and response time. It should be larger than the context switch time.
            \\ 
            \\ Typically from $10ms - 200ms$.
            \begin{center}
                \begin{tabular}{r l}
                    \textbf{OS} & \textbf{Time Slice} \\
                    Linux & $100ms$ \\
                    Windows Client & $20ms$ \\
                    WIndows Server & $180ms$ \\
                \end{tabular}
            \end{center}
        \subsubsection*{Shortest Job First}
            Schedule the ready thread with the shortest CPU burst remaining given we know the runtimes of each job (either provided, predicted or historical).
            \centerimage{width=\textwidth}{SJF Scheduler.png}
            Average turnaround time is optimally low as we run the jobs that are quickest first.
        \subsubsection*{Shortest Remaining Time}
            Premptive version of \keyword{SJF}, chooses the process with the shortest remaining time.
            \\
            \\ As \keyword{SJF} only makes scheduling decisions when a job is complete, when a new job is added while another job is running, it will have to wait. In \keyword{SRT} we can reschedule immediately upon receiving the shorter job. For example:
            \\
            \\ Job 1 \& 2 added simultaneously, Job 3 added after $2s$:
            \centerimage{width=0.7\textwidth}{SJF vs SRT.png}
            One issue with \keyword{SRT} is that a long job could be heavily delayed if many short jobs are added.
        \subsubsection*{Konowing Runtimes in Advance}
            Runtimes are normally not available in advance. However we can make estimates based on previous history (e.g past CPU time).
            \\
            \\ We can also use user estimates, however we may need to counteract programs cheating (by providing low estimates) by for example penalising processes that overrun their estimated time.
        \subsubsection*{Fair Share Scheduling}
            On a multi-user system, we need to take into account which processes are being run by which users.
            \\
            \\ This can be achieved in \keyword{round robin} scheduling by having a ready queue per user, and round robining on which ready queue is used.
            \subsubsection*{General Purpose Scheduling}
                \begin{itemize}
                    \bullpara{Favour short \& I/O Bound jobs}{
                        \compitem{
                            \item Once scheduled they are quickly done with and can have resources freed.
                            \item I/O bound will likely be blocked and unscheduled quickly.   
                            \item Keep reposnse times low.
                        }
                    }
                    \bullpara{Quickly Determine Job Nature}{
                        \\ By quickly adapting to determine which jobs are \keyword{CPU} and \keyword{I/O} bound we can schedule more effectively.
                    }
                \end{itemize}
    
        \subsubsection*{Priority Scheduling}
            \begin{itemize}
                \item Jobs run based on a priority. Always running the job with the highest priority.
                \item Priorities can be externally defined (user) or based on process-specific metrics (e.g expected CPU burst, previous CPU time, process parent).
                \item Can be static (unchanging) or dynamic (change during execution - e.g process aging).
            \end{itemize}
            Note: Shortest remaining time is a form of priority scheduling.
            \\
            \\ These are also used in:
            \compitem{
                \item Windows Vista, 7
                \item Mac OS X
                \item Linux 2.6 - 2.6.23
                \item Pintos!
            }
            \sidenote{Process explorer}{
                If you are on windows, you can use the process explorer to see the priority system in action on any running process \& its threads.
                \\ (https://docs.microsoft.com/en-us/sysinternals/downloads/process-explorer)
                \centerimage{width=0.7\textwidth}{process explorer.png}
            }


        \subsubsection*{Multilevel Feedback queues}
            \centerimage{width=0.5\textwidth}{multilevel feedback queue.png}
            Each priority level has its own set of runnable threads, which can be scheduled (even by different algorithms). 
            \\
            \\ We must be careful not to starve lower priority jobs. This can be done with the feedback mechanism that age a thread as it waits, and assign higher priority to threads that are waiting a long time.
            \\
            \\ We can also give I/O bound operations higher priority to ensure they are able to run as quickly they are done waiting (response time is important).
            \\
            \\ However there are some disadvantages:
            \begin{itemize}
                \bullpara{Inflexible}{
                    \\ Applications have no control over their priority, and priorities make no guarentees (e.g high priority does not matter, if all processes are running on high priority)
                }
                \bullpara{Cannot React Quickly}{
                    \\ Often Needs a warm-up period (as initially priorities of processes may not relfect what bounds them/their behaviour). This also becomes a rpoblem for realtime systems that must react quickly.
                }
                \bullpara{Cheating}{
                    \\ e.g Adding meaningless I/O to boost priority.
                    \\
                    \\ As prity is based off 'feedback', priority can be manipulated.
                }
            \end{itemize}
        
        \subsubsection*{Lottery Scheduling}
            Next job to schedule is based on probability. At each scheduling decision a random process is chosen, with probability biased towards threads we that are higher priority.
            \begin{itemize}
                \bullpara{tickets Meaningful}{
                    \\ If a process has $20\%$ of the tickets, it gets on average $20\%$ of the time.
                }
                \bullpara{Highly responsive}{
                    \\ We can provide a job more tickets to give it a higher chance of being run at the next scheduling decision.
                }
                \bullpara{No Starvation}{
                    \\ As all threads have a chance of being scheduled at every decision.
                }
                \bullpara{Donation}{
                    \\ Jobs can exchange tickets, this allows for easy priority donation, and allows cooperating jobs to achieve goals.
                }
                \bullpara{Adding a Job affects existing Jobs}{}
                \bullpara{Unpredictable}{
                    A process can be \textit{lucky} or \textit{unlucky}, and this can impact response times (e.g interactive process is very unlucky, causes an issue).
                }
            \end{itemize}
    \section*{Summary}
        \begin{itemize}
            \bullpara{Schedulers Balance Conflicting Goals}{Fairness, enforce policy, maximise resource utilisation, minimise overhead.}
            \bullpara{Different Scheduling Algorithms are Appropriate in Different Contexts}{Batch vs interactive vs real-time.}
            \bullpara{Well Studied Algorithms}{FCFS, RR, SJF, SRT, MLFQs, Lottery}
        \end{itemize}
        \centerimage{width=\textwidth}{Numbers you should know.png}
    \section*{Additions}
        A little python program for doing \keyword{RR}, \keyword{fcfs} calculations:
        \lstinputlisting[language=python]{schedulers.py}
\end{document}
