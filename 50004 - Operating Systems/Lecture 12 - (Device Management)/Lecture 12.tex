\documentclass{report}
    \title{50004 - Operating Systems - Lecture 12}
    \author{Oliver Killane}
    \date{25/11/21}
%===========================COMMON FORMAT & COMMANDS===========================
% This file contains commands and format to be used by every module, and is 
% included in all files.
%===============================================================================

%====================================IMPORTS====================================
\usepackage[a4paper, total={6in, 8in}]{geometry}
\usepackage{graphicx, amssymb, amsfonts, amsmath, xcolor, listings, tcolorbox, multirow, hyperref}
%===============================================================================

%====================================IMAGES=====================================
\graphicspath{{image/}}

% \centerimage{options}{image}
\newcommand{\centerimage}[2]{\begin{center}
    \includegraphics[#1]{#2}
\end{center}}
%===============================================================================

%==============================SYNTAX HIGHLIGHTING==============================
\newcommand{\fun}[1]{\textcolor{blue}{\textbf{#1}}}
\newcommand{\file}[1]{\textcolor{green}{\textbf{#1}}}
\newcommand{\struct}[1]{\textcolor{orange}{\textbf{#1}}}
\newcommand{\var}[1]{\textcolor{purple}{\textbf{#1}}}
\newcommand{\const}[1]{\textcolor{red}{\textbf{#1}}}
%===============================================================================

%=================================CODE LISTINGS=================================
\definecolor{codebackdrop}{gray}{0.9}
\definecolor{commentgreen}{rgb}{0,0.6,0}
\lstset{
    inputpath=code, 
    commentstyle=\color{commentgreen},
    keywordstyle=\color{blue}, 
    backgroundcolor=\color{codebackdrop}, 
    basicstyle=\footnotesize,
    frame=single,
    numbers=left,
    stepnumber=1,
    showstringspaces=false,
    breaklines=true,
    postbreak=\mbox{\textcolor{red}{$\hookrightarrow$}\space}
}

% Create a code listing for a single line
% \codeline{language}{line}{file}
\newcommand{\codeline}[3]{\lstinputlisting[language=#1, firstline = #2, lastline = #2]{#3}}

% Create a code listing for multiple lines
% \codeline{language}{start}{end}{file}
\newcommand{\codelines}[4]{\lstinputlisting[language=#1, firstline = #2, lastline = #3]{#4}}


% Create a code listing for a given language & file
% \codelist{language}{file}
\newcommand{\codelist}[2]{\lstinputlisting[language=#1]{#2}}
%===============================================================================

%================================TEXT STRUCTURES================================
% Marka a word as bold
% \keyword{important word}
\newcommand{\keyword}[1]{\textbf{#1}}

% Creates a section in italics
% \question{question in italics}
\newcommand{\question}[1]{\textit{#1} \\ }

% Creates a box with title for side notes.
% \sidenote{title}{contents}
\newcommand{\sidenote}[2]{\begin{tcolorbox}[title=#1]#2\end{tcolorbox}}

\newcommand{\termdef}[2]{\begin{tcolorbox}[title=Definition: #1, colframe = blue]#2\end{tcolorbox}}

% Creates an item in an itemize or enumerate, with a paragraph after
% \begin{itemize}
%     \bullpara{title}{contents}
% \end{itemize}
\newcommand{\bullpara}[2]{\item \textbf{#1} \ #2}

% Creates a compact list (very small gaps between items)
% \compitem{
%     \item item 1
%     \item item 2
%     \item ...
% }
\newcommand{\compitem}[1]{\begin{itemize}\setlength\itemsep{-0.5em}#1\end{itemize}}
\newcommand{\compenum}[1]{\begin{enumerate}\setlength\itemsep{-0.5em}#1\end{enumerate}}

% Creates a link to the lecture for use at the start of the notes document
\newcommand{\lectlink}[1]{\sidenote{Lecture Recording}{
    Lecture recording is available \href{#1}{here}
}}
%===============================================================================

%================================STYLIZED PROOFS================================
%For ease in writing stylized proofs with numbers
\newcommand{\stepno}[1]{\textcolor{red}{\textbf{#1}}}

\newenvironment{proof}
{\begin{center}\begin{tabular}{r l l }}
{\end{tabular}\end{center}}

%\proofstep{step}{workings}{description}
\newcommand{\proofstep}[3]{\stepno{(#1)} & #2 & #3 \\}
%===============================================================================

%==============================UNFINISHED SECTION===============================
\newcommand{\unfinished}{\begin{huge} \textcolor{red}{\textbf{UNFINISHED!!!}} \end{huge}}
%===============================================================================

%==============================THREAD HIGHLIGHTING==============================
\newcommand{\threada}[1]{\textcolor{green}{\textbf{#1}}}
\newcommand{\threadb}[1]{\textcolor{red}{\textbf{#1}}}
%===============================================================================

%============================DISPLAYING THREAD CODE=============================
\newcommand{\threadlist}[3]{
    \codelist{#1}{#2 init.#3}
    \begin{minipage}[t]{0.45\textwidth}
        \codelist{#1}{#2 A.#3}
    \end{minipage}
    \hfill
    \begin{minipage}[t]{0.45\textwidth}
        \codelist{#1}{#2 B.#3}
    \end{minipage}
}
%===============================================================================
\begin{document}
\maketitle
\lectlink{https://imperial.cloud.panopto.eu/Panopto/Pages/Viewer.aspx?id=f7672205-af40-419e-aee8-ade900e5f489}

\section*{Device Management}
\subsubsection*{Intel Example}
\centerimage{width=\textwidth}{intel example.png}
\sidenote{North \& South Bridge}{
	Here the \keyword{X58} connects the CPU to high-speed peripherals through the high bandwidth \keyword{QuickPath Interconnect} (\keyword{QPI}). This is the \keyword{North Bridge} which connects the CPU to peripherals directly.
	\\
	\\ The \keyword{ICH10} (I/O Controller Hub) supports lower speed peripherals, and connects to the CPU through the \keyword{North Bridge}.
}

\subsection*{I/O Device Management Objectives}
\begin{itemize}
	\bullpara{Fair Access to Shared Devicess}{ Prevent processes hogging resources }
	\bullpara{Exploit Parallelism}{
		\\ Can use devices in parallel (e.g send packets using network card while writing to disk), and some devices have parallelism themselves.
	}
	\bullpara{Provide uniform \& Simple view of I/O}{
	\\ Abstract devices away from processes (e.g filesystem, not disk). And use uniform interfaces (when new devices added, programs do not need to change to utilise).
	\compitem{
	\item Uniform naming and error handling.
	\item Hide complexity of device handling.
	      }
	      }
\end{itemize}

\subsection*{Device Independence}
Have the device be independent from its type (e.g terminal, disk, dvd drive) and which instance (e.g disk 1, 2, 3).
\\
\\ E.g can use the same interface for all disks connected to a system. Can read data from dvd drive, disk, terminal in a single interface (sometimes we split into classes when differences are large enough).
\subsection*{Device Variations}
\compitem{
	\bullpara{Unit of Data Transfer}{Character (bytes) or block}
	\bullpara{Supported Operations}{e.g read, write, seek}
	\bullpara{Synchronous or asynchronous}{e.g network card (send request, then get result back sometime), disk (read and block until the data is read)}
	\bullpara{Speed differences}{e.g NVMe SSD vs Tape Drive}
	\bullpara{Shareable}{e.g disks are shareable, printers are not (can print one at a time)}
	\bullpara{Types of error conditions}{e.g Disk errors, vs GPU temperature warnings}
}
\subsubsection*{Character vs Block Device}
\centerimage{width=0.8\textwidth}{character vs block.png}
Note the type (character/block) depends on the type of device.
\begin{center}
	\begin{tabular}{l l l}
		Block     & Maximize throughput & Disks, network cards etc \\
		Character & Minimize latency    & Keyboard, terminal etc   \\
	\end{tabular}
\end{center}
\textbf{Character Devices:}
\\ \begin{tabular}{l l l}
	1   & mem & Device file representing the physical memory of the os. \\
	2   & pty & Pesudoterminal (bidirectional communication channel).   \\
	180 & usb & USB devices.                                            \\
\end{tabular}
\\ \textbf{Block Devices:}
\\ \begin{tabular}{l l l}
	1 & ramdisk & access ram disk in raw mode.                                                        \\
	2 & fd      & Floppy Disk.                                                                        \\
	7 & loop    & Loop device, maps data blocks to a file in the filesystem, or another block device. \\
\end{tabular}

\subsection*{I/O Layers}
\centerimage{width=0.6\textwidth}{IO layering.png}
\subsubsection*{Interrupt Handler}
An interrupt is a signal sent from a device to the CPU to inform it that the device needs attending to (e.g device connecting, finished reading, error has occured etc.).
\\
\\ Drivers register handlers to deal with types of interrupts, when the CPU receives an interrupt, it runs the relevant handler.
\compitem{
	\item For block devices, on transfer completion signal device handler.
	\item For character devices, when character transfered, process next character.
}
For modern systems (e.g nvme storage) a hybrid approach makes use of the following:
\compitem{
	\item Polling (queue of requests and responses), very fast but uses CPU time (analogous to spin locks)
	\item Wait for interrupt (better for longer waits - another process can be scheduled)
}
\subsubsection*{Device Driver}
Handles a type of device, can control multiple devices of the same type.
\compitem{
	\item Implements block read/write
	\item Access device registers (writing control information to a device)
	\item Initiating Operations (start device e.g at boot)
	\item Scheduling requests (if a device is shared, placing in order, often for performance - order hard disk operations to do the least disk traversal).
	\item Handle Errors
}
\subsubsection*{Device Independent OS Layer}
Use standard interfaces for drivers of device types:
\compitem{
	\item Simplifies OS design.
	\item Interface to write new drivers to.
	\item No OS changes required to support new drivers, just support interfaces.
}
This layer also provides \keyword{device independence}:
\compitem{
	\bullpara{Map Logical to Physical Devices}{ (Naming and Switching)
		\\ Can map one logical device to many physical, or vice versa (e.g disk RAIDs).
	}
	\bullpara{Request Validation against device}{
		\\ Check device \& driver is working correctly.
	}
	\bullpara{Allocation}{
		\\ Determine which processes can access which logical devices.
		\\
		\\ Control \keyword{dedicated allocation} (process exclusive access to a device)
	}
	\bullpara{Buffering}{
		\\ As previously mentioned - for performance \& block size independence.
	}
	\bullpara{Error Reporting}{}
}
\subsubsection*{User-Level I/O Interface}
System call interface to allow user programs to interact (often through other 3rd party libraries).
\compitem{
	\item basic I/O operations (\fun{close}, \fun{read}, \fun{write}, \fun{seek})
	\item Sets up parameters (device independent)
	\item Can be synchronous (blocking) or asynchronous (non-blocking)
}
\sidenote{Unix files}{
	Unix accesses virtual devices as files. This allows access to devices using the normal standard input/output calls. For example the common:
	\begin{center}
		\begin{tabular}{c c}
			\textbf{File Descriptor} & \textbf{Name}   \\
			0                        & Standard Input  \\
			1                        & Standard Output \\
			2                        & Standard Error  \\
		\end{tabular}
	\end{center}
	There are also files such as \const{/dev/kdb} for keyboard access.
}
\subsection*{Device Allocation}
\begin{itemize}
	\bullpara{Dedicated Device}{ (e.g DVD writer, terminal, printer)
	\compitem{
	\item One process gets exclusive access to a device.
	\item Typically allocated for long periods of time (e.g minutes - hours: printing, or controlling a terminal display output)
	\item If another process tries to access, it fails (potentially adding to a queue of open requests).
	\item Only allocated to authorised processes (e.g don't want malicious processes blocking access to a resource).
	      }
	      }
	      \bullpara{Shared Device}{ (e.g disks, window terminals etc)
		      \\ OS can provide systems for sharing, e.g file system accessible by all processes makes use of the disk.
	      }
	      \bullpara{Spooled Device}{ (e.g printer, dvd writer - many other dedicated devices)
	      \\ A shared pool. A \keyword{daemon} process has dedicated access, processes send requests/jobs to the \keyword{daemon}.
	      \compitem{
	\item Provides sharing on non-sharable resources.
	\item Reduces I/O time resulting in greater throughput.
	      }
	      }
\end{itemize}
\subsection*{Buffered vs Unbuffered I/O}
\begin{itemize}
	\bullpara{Buffered}{
	\\ \begin{tabular}{l p{10cm}}
		Output & User data transferred to OS output buffer. Process continues and only suspends once buffer is full. \\
		Input  & OS reads ahead. Reads taken from buffer, process blocks when buffer empty.                          \\
	\end{tabular}
	\compitem{
	\item Smooths peaks in I/O traffic (allows for limited load balancing).
	\item Can allow for different data transfer unit sizes between devices (e.g buffer contains blocks).
	      }
	      }
	      \bullpara{Unbuffered}{
	      \compitem{
	\item Data Transfered directly between device and user space.
	\item Each read/write causes physical I/O (device does something, not just accessing a hidden buffer).
	\item Device handler used for every transfer.
	\item High switching overhead (e.g every read requires the driver to take over, and to do some physical action).
	      }
	      }
\end{itemize}

\section*{Device Drivers}
\subsubsection*{Memory Mapped I/O}
Device can be addressed as a special memory location.
\centerimage{width=0.8\textwidth}{memory mapped IO.png}
Hence we can use the virtual memory setup to restrict access (e.g set supervisor bit to prevent user access).
\subsection*{I/O}
\begin{itemize}
	\bullpara{Programmed I/O}{ (simple but inefficient)
		\\ Wait for device (spin), then continue execution.
	}
	\bullpara{Interrupt Driven}{ (large overhead, good for long expected waits)
		\\ Hardware sends an interrupt when operation completed, can do other work while waiting.
	}
	\bullpara{Direct Memory Access (DMA)}{ (requires hardware, but reduces CPU intervention)
		\\ A DMA controller (often in the device) waits for the device to respond, then once the full result is available, places this directly into memory.
	}
\end{itemize}

\section*{Linux}
\subsection*{Loadable Kernel Module (LKM)}
Device drivers are loadable modules that are loadeed and linked dynamically with the running kernel.
\\
\\ This requires binary compatability (module must be specific to kernel version).
\\
\\ Linux uses \keyword{Kmod}
\compitem{
	\item Kernel subsystem that manages modules without user intervention.
	\item Determines modules dependencies.
	\item Loads modules on demand (e.g load driver for network card when it is connected to the system).
}
\subsubsection*{Basic LKM module}
\codelist{C}{basic LKM module.c}
Kernel can open file, look for symbol table (generated by compiler), and call the corresponding functions.
\codelist{Bash}{load module.sh}
\subsection*{IO Management}
\sidenote{/dev /sys /proc}{
	Linux (in UNIX fashion) uses files to represent many drivers, services \& methods to collect system information. Some of the main directories for this are:
	\begin{itemize}
		\bullpara{/dev}{device file directory
			\\ Contains files for devices (virtual included).
		}
		\bullpara{/proc}{virtual filesystem for processes
			\\ Contains information on running processes, each represented by files of size 0.
			Each numbered directory is actually the \keyword{pid} of a running process.
			\\
			\\ Other files can be read to get system information such as:
			\\ \begin{tabular}{l p{10cm}}
				/proc/meminfo    & Information on the memory system including free pages, kernel stack pages etc. \\
				/proc/interrupts & Number of interrupts received for categories.                                  \\
				/proc/stat       & System status information (e.g number of processes).                           \\
				/proc/version    & OS version information.                                                        \\
			\end{tabular}
		}
		\bullpara{/sys}{
			\\ A filesystem like view of the kernel and its configuration/settings.
		}
	\end{itemize}
}
\begin{itemize}
	\bullpara{Device Classes}{Group similar types of devices (similar function, performance needs)}
	\bullpara{Identification Numbering}{ $<Major>:<Minor>$
		\\ \begin{tabular}{l l}
			\textbf{Major} & Determines which diver is controlling. (e.g IDE Disk Drive, Serial port etc.) \\
			\textbf{Minor} & Distinguishes devices of the same class. (e.g Drive Number)                   \\
		\end{tabular}
		\\ Examples from kernel documentation are \href{https://www.kernel.org/doc/Documentation/admin-guide/devices.txt}{here}.
	}
	\bullpara{Special Files}{
		\\ Most devices are represented by device/special files in the \const{/dev} directory.
		\centerimage{width=\textwidth}{dev directory.png}
	}
\end{itemize}
\subsection*{Device Access}
Device files are accessed via the \keyword{virtual file system (VFS)}.
\centerimage{width=0.6\textwidth}{VFS.png}
Most drivers implement \fun{read}, \fun{write}, \fun{seek} etc (much like a file), however all contain other operations which do not fit this abstraction.
\\
\\ Linux uses the \keyword{ioctl} (I/O Control) system call to support special tasks (e.g getting printer status, ejecting a CD drive)
\codelist{C}{ioctl.c}
\subsection*{Character Device I/O}
\compitem{
	\item Data transitted as a stream of bytes (read a byte, then another is presented)
	\item Represented by a \struct{char\_device\_struct} structure.
	\item \struct{char\_device\_struct} contains a pointer to a \struct{file\_operations} struct.
}
The \struct{file\_operations} struct:
\compitem{
	\item Maintains operations supported by the device driver.
	\item Stores function pointers to operations (read, write etc).
}
\href{https://github.com/torvalds/linux/blob/8ced7ca3570333998ad2088d5a6275701970e28e/include/linux/fs.h#L2069}{file\_operations} in the linux kernel (github).
\codelist{C}{file operations.c}

\href{https://github.com/torvalds/linux/blob/3498e7f2bb415e447354a3debef6738d9655768c/fs/char_dev.c#L34}{char\_device\_struct} in the linux kernel (github).
\codelist{C}{char device struct.c}

Note: \href{https://tldp.org/LDP/lkmpg/2.4/html/c577.htm}{this} is a basic example, very interesting!

\subsection*{Block Device I/O}
\begin{itemize}
	\bullpara{Block I/O Subsystem}{
	\\ Consists of several layers, with modularised operations (common code ine ach layer).
	\\
	\\ The tw main strategies to minimise time accessing block devices:
	\compenum{
	\item caching data.
	\item Clustering I/O Operations (store up tasks and execute several at once).
	      }
	      When given a task, check cache. If data not present, queue request on request queue for device.
	      }
	      \bullpara{Direct I/O}{
		      \\ Bypass the kernel cache when accessing a device.
		      \\
		      \\ Useful for databases and some other applications where caching can reduce performance/consistency (e.g values very rarely accessed more than once, or doing caching in use level).
	      }
\end{itemize}
\end{document}
