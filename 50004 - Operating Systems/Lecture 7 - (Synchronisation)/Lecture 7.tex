\documentclass{report}
    \title{50004 - Operating Systems - Lecture 7}
    \author{Oliver Killane}
    \date{09/11/21}

%===========================COMMON FORMAT & COMMANDS===========================
% This file contains commands and format to be used by every module, and is 
% included in all files.
%===============================================================================

%====================================IMPORTS====================================
\usepackage[a4paper, total={6in, 8in}]{geometry}
\usepackage{graphicx, amssymb, amsfonts, amsmath, xcolor, listings, tcolorbox, multirow, hyperref}
%===============================================================================

%====================================IMAGES=====================================
\graphicspath{{image/}}

% \centerimage{options}{image}
\newcommand{\centerimage}[2]{\begin{center}
    \includegraphics[#1]{#2}
\end{center}}
%===============================================================================

%==============================SYNTAX HIGHLIGHTING==============================
\newcommand{\fun}[1]{\textcolor{blue}{\textbf{#1}}}
\newcommand{\file}[1]{\textcolor{green}{\textbf{#1}}}
\newcommand{\struct}[1]{\textcolor{orange}{\textbf{#1}}}
\newcommand{\var}[1]{\textcolor{purple}{\textbf{#1}}}
\newcommand{\const}[1]{\textcolor{red}{\textbf{#1}}}
%===============================================================================

%=================================CODE LISTINGS=================================
\definecolor{codebackdrop}{gray}{0.9}
\definecolor{commentgreen}{rgb}{0,0.6,0}
\lstset{
    inputpath=code, 
    commentstyle=\color{commentgreen},
    keywordstyle=\color{blue}, 
    backgroundcolor=\color{codebackdrop}, 
    basicstyle=\footnotesize,
    frame=single,
    numbers=left,
    stepnumber=1,
    showstringspaces=false,
    breaklines=true,
    postbreak=\mbox{\textcolor{red}{$\hookrightarrow$}\space}
}

% Create a code listing for a single line
% \codeline{language}{line}{file}
\newcommand{\codeline}[3]{\lstinputlisting[language=#1, firstline = #2, lastline = #2]{#3}}

% Create a code listing for multiple lines
% \codeline{language}{start}{end}{file}
\newcommand{\codelines}[4]{\lstinputlisting[language=#1, firstline = #2, lastline = #3]{#4}}


% Create a code listing for a given language & file
% \codelist{language}{file}
\newcommand{\codelist}[2]{\lstinputlisting[language=#1]{#2}}
%===============================================================================

%================================TEXT STRUCTURES================================
% Marka a word as bold
% \keyword{important word}
\newcommand{\keyword}[1]{\textbf{#1}}

% Creates a section in italics
% \question{question in italics}
\newcommand{\question}[1]{\textit{#1} \\ }

% Creates a box with title for side notes.
% \sidenote{title}{contents}
\newcommand{\sidenote}[2]{\begin{tcolorbox}[title=#1]#2\end{tcolorbox}}

\newcommand{\termdef}[2]{\begin{tcolorbox}[title=Definition: #1, colframe = blue]#2\end{tcolorbox}}

% Creates an item in an itemize or enumerate, with a paragraph after
% \begin{itemize}
%     \bullpara{title}{contents}
% \end{itemize}
\newcommand{\bullpara}[2]{\item \textbf{#1} \ #2}

% Creates a compact list (very small gaps between items)
% \compitem{
%     \item item 1
%     \item item 2
%     \item ...
% }
\newcommand{\compitem}[1]{\begin{itemize}\setlength\itemsep{-0.5em}#1\end{itemize}}
\newcommand{\compenum}[1]{\begin{enumerate}\setlength\itemsep{-0.5em}#1\end{enumerate}}

% Creates a link to the lecture for use at the start of the notes document
\newcommand{\lectlink}[1]{\sidenote{Lecture Recording}{
    Lecture recording is available \href{#1}{here}
}}
%===============================================================================

%================================STYLIZED PROOFS================================
%For ease in writing stylized proofs with numbers
\newcommand{\stepno}[1]{\textcolor{red}{\textbf{#1}}}

\newenvironment{proof}
{\begin{center}\begin{tabular}{r l l }}
{\end{tabular}\end{center}}

%\proofstep{step}{workings}{description}
\newcommand{\proofstep}[3]{\stepno{(#1)} & #2 & #3 \\}
%===============================================================================

%==============================UNFINISHED SECTION===============================
\newcommand{\unfinished}{\begin{huge} \textcolor{red}{\textbf{UNFINISHED!!!}} \end{huge}}
%===============================================================================

%==============================THREAD HIGHLIGHTING==============================
\newcommand{\threada}[1]{\textcolor{green}{\textbf{#1}}}
\newcommand{\threadb}[1]{\textcolor{red}{\textbf{#1}}}
%===============================================================================

%============================DISPLAYING THREAD CODE=============================
\newcommand{\threadlist}[3]{
    \codelist{#1}{#2 init.#3}
    \begin{minipage}[t]{0.45\textwidth}
        \codelist{#1}{#2 A.#3}
    \end{minipage}
    \hfill
    \begin{minipage}[t]{0.45\textwidth}
        \codelist{#1}{#2 B.#3}
    \end{minipage}
}
%===============================================================================

\begin{document}
\maketitle
\lectlink{https://imperial.cloud.panopto.eu/Panopto/Pages/Viewer.aspx?id=108a0538-c84b-4fad-ab78-addb00f338d1}

\section*{Semaphores}
A blocking synchronisation mechanism invented by \keyword{Dijkstra}. Processes cooperate by means of signals:
\begin{itemize}
	\item Process stops, waiting for a signal.
	\item Process continues if it has received a specific signal.
\end{itemize}
Semaphores allow \var{i} processes to be between \fun{down} and \fun{up}. They are special variables containing
a value and a queue of waiting processes, they are accessible through the following atomic operations:
\begin{itemize}
	\bullpara{\fun{down}(\var{s})}{wait to receive signal via \keyword{semaphore} \var{s}.
		\codelist{C}{semaphore down.c}}
	\bullpara{\fun{up}(\var{s})}{transmit a signal via \keyword{semaphore} \var{s}.
		\codelist{C}{semaphore up.c}}
	\bullpara{\fun{init}(\var{s},\var{i})}{initialise semaphore \var{s} with value \var{i}
		\codelist{C}{semaphore init.c}}
\end{itemize}
\subsubsection*{Mutual Exclusion}
\codelist{C}{semaphore mutual exclusion.c}
A semaphore with an initial value of $1$ acts similarly to a \keyword{lock} (\keyword{lock} also enforces that only
the thread that acquired the access to the critical region can release it).
\subsubsection*{Ordering Events}
\threadlist{C}{semaphore ordering events}{c}
No matter the order of processes being started, process B cannot continue to the critical region until process A has called \fun{sema\_up}.
\section*{Producer/Consumer}
\begin{itemize}
	\bullpara{Producer Constraints}{
	\compitem{
	\item Items can only be deposited if there is space.
	\item Items can only be deposited in the buffer if mutual exclusion is ensured.
	      }
	      }
	      \bullpara{Consumer Constraints}{
	      \compitem{
	\item Items can only be fetched if the buffer is not empty.
	\item Items can only be fetched from the buffer if mutual exclusion is ensured.
	      }
	      }
	      \bullpara{Buffer Constraints}{
		      \\ Buffer can hold between $0$ and $N$ items.
	      }
\end{itemize}
\codelist{C}{producer consumer init.c}
\begin{minipage}[t]{0.45\textwidth}
	\codelist{C}{producer.c}
\end{minipage}
\hfill
\begin{minipage}[t]{0.45\textwidth}
	\codelist{C}{consumer.c}
\end{minipage}
\section*{Monitors}
Monitors allow processes to wait on high level conditions. Entry procedures are called from outside the monitor, internal procedures only from inside (cannot access monitor data, only monitor procedures).
\\
\\ There is an (implicit) monitor lock, with one process being in the monitor at a time.
\\
\\ The high-level conditions could be:
\begin{itemize}
	\item \textit{Some space is available in the buffer}
	\item \textit{Some data has arrived in the buffer}
	\item \textit{There is an available resource}
\end{itemize}
Signals do not accumulate, if a condition is signalled and no processes/threads are waiting for it, the signal is lost.
\\
\\ Typically a language construct, and are not supported by \keyword{C}, however we can explicitly implement them (as in \keyword{PINTOS}).
\\ Java uses synchronize, \fun{wait} and \fun{notify} with no condition variables.
\subsubsection*{Monitor Procedures}
\begin{itemize}
	\bullpara{\fun{wait}(\var{condition})}{
		Releases monitor lock, waits for \var{condition} to be signaled.
	}
	\bullpara{\fun{signal}(\var{condition})}{
		Wakes up one process waiting for \var{condition}.
	}
	\bullpara{\fun{broadcast}(\var{condition})}{
		Wakes up all processes waiting for \var{condition}.
	}
\end{itemize}
\subsubsection*{Monitor Code}
\codelist{VBScript}{producer consumer monitor.pesudocode}
\section*{Synchronisation Summary}
\begin{itemize}
	\bullpara{Lock}{
	\compitem{
	\item Reader/Writer Locks.
	\item Often exposed in monitor language construct.
	\item Within a process.
	\item 1 process/thread in the critical section.
	      }
	      }
	      \bullpara{Mutex}{
	      \compitem{
	\item Like lock, but works across processes.
	      }
	      }
	      \bullpara{Semaphore}{
	      \compitem{
	\item Like mutex, but can let $N$ processes / threads pass \& does not need the same process/thread to release as acquired.
	      }
	      }
\end{itemize}
\end{document}
