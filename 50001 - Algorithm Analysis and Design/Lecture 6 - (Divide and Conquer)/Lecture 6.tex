\documentclass{report}
    \title{50001 - Algorithm Analysis and Design - Lecture 6}
    \author{Oliver Killane}
    \date{12/11/21}

%===========================COMMON FORMAT & COMMANDS===========================
% This file contains commands and format to be used by every module, and is 
% included in all files.
%===============================================================================

%====================================IMPORTS====================================
\usepackage[a4paper, total={6in, 8in}]{geometry}
\usepackage{graphicx, amssymb, amsfonts, amsmath, xcolor, listings, tcolorbox, multirow, hyperref}
%===============================================================================

%====================================IMAGES=====================================
\graphicspath{{image/}}

% \centerimage{options}{image}
\newcommand{\centerimage}[2]{\begin{center}
    \includegraphics[#1]{#2}
\end{center}}
%===============================================================================

%=================================CODE LISTINGS=================================
\definecolor{codebackdrop}{gray}{0.9}
\definecolor{commentgreen}{rgb}{0,0.6,0}
\lstset{
    inputpath=code, 
    commentstyle=\color{commentgreen},
    keywordstyle=\color{blue}, 
    backgroundcolor=\color{codebackdrop}, 
    basicstyle=\footnotesize,
    frame=single,
    numbers=left,
    stepnumber=1,
    showstringspaces=false,
    breaklines=true,
    postbreak=\mbox{\textcolor{red}{$\hookrightarrow$}\space}
}

% Create a code listing for a single line
% \codeline{language}{line}{file}
\newcommand{\codeline}[3]{\lstinputlisting[language=#1, firstline = #2, lastline = #2]{#3}}

% Create a code listing for a given language & file
% \codelist{language}{file}
\newcommand{\codelist}[2]{\lstinputlisting[language=#1]{#2}}
%===============================================================================

%================================TEXT STRUCTURES================================
% Marka a word as bold
% \keyword{important word}
\newcommand{\keyword}[1]{\textbf{#1}}

% Creates a section in italics
% \question{question in italics}
\newcommand{\question}[1]{\textit{#1} \\ }

% Creates a box with title for side notes.
% \sidenote{title}{contents}
\newcommand{\sidenote}[2]{\begin{tcolorbox}[title=#1]#2\end{tcolorbox}}

% Creates an item in an itemize or enumerate, with a paragraph after
% \begin{itemize}
%     \bullpara{title}{contents}
% \end{itemize}
\newcommand{\bullpara}[2]{\item \textbf{#1} \ #2}

% Creates a compact list (very small gaps between items)
% \compitem{
%     \item item 1
%     \item item 2
%     \item ...
% }
\newcommand{\compitem}[1]{\begin{itemize}\setlength\itemsep{-0.5em}#1\end{itemize}}

% Creates a link to the lecture for use at the start of the notes document
\newcommand{\lectlink}[1]{\sidenote{Lecture Recording}{
    Lecture recording is available \href{#1}{here}
}}
%===============================================================================


%=============================DISPLAYING CODE STEPS=============================
\newenvironment{steps}
{\begin{tabular}{l}}
		{\end{tabular}}

\newcommand{\step}[2]{$\leadsto$ \{ #2 \} \\ $#1$ \\}
\newcommand{\start}[1]{$#1$ \\}

%===============================================================================


\begin{document}
\maketitle
\lectlink{https://imperial.cloud.panopto.eu/Panopto/Pages/Viewer.aspx?id=0cc4b21b-9502-4919-8602-adcc00d14ece}

\section*{Divide \& Conquer}
\begin{enumerate}
	\item Divide a problem into subproblems
	\item Divide and conquer subproblems into subsolutions
	\item Conquer subsolutions into a solution
\end{enumerate}
\centerimage{width=0.35\textwidth}{divide and conquer.png}

\section*{Merge Sort}
\codelist{Haskell}{merge sort.hs}
\fun{SplitAt} divides, and \fun{merge} Conquers. We can calculate the time complexity for the recurrence relations below (based on recursive structure of \fun{msort}):
\\ \begin{tabular}{l l }
	$T_{msort}(0) $ & $= 1 $                                                                                                      \\
	$T_{msort}(1) $ & $= 1 $                                                                                                      \\
	$T_{msort}(n) $ & $= T_{length}(n) + T_{splitAt}(\cfrac{n}{2}) + T_{merge}(\cfrac{n}{2}) + 2 \times T_{msort}(\cfrac{n}{2}) $ \\
\end{tabular}

We can simplify the complexity of \fun{msort}
\\ \begin{tabular}{l l }
	$T_{msort}(n) $ & $= T_{length}(n) + T_{splitAt}(\cfrac{n}{2}) + T_{merge}(\cfrac{n}{2}) + 2 \times T_{msort}(\cfrac{n}{2}) $ \\
	                & $= n + \cfrac{n}{2} + \cfrac{n}{2} + \cfrac{n}{2} + 2\times T_{msort}(\cfrac{n}{2})$                        \\
	                & $= \cfrac{5}{2} \times n + 2\times T_{msort}(\cfrac{n}{2})$                                                 \\
\end{tabular}
\sidenote{Master Theorem}{
	For an algorithm $algo$ such that:
	\[T_{algo}(n) = a \times T_{algo}(\cfrac{n}{b}) + f(n) + \text{base cases}\]
	The work at recursion level $\log_bn$ is $\Theta(a^{\log_bn})$
	To calculate the order of the time complexity:
	\begin{enumerate}
		\item Get the recurrence relation in the form above.
		\item Get the critical exponent $E$ by formula: $E = \log_ba = \cfrac{\log a}{\log b}$.
		\item Given $f(n) = n^c$ we can express the work as a geometric sum $\sum_{i=0}^{\log n}ar^i$ where $r = \cfrac{a}{b^c}$.
	\end{enumerate}
	\[r > 1 \Leftrightarrow a > b^c \Leftrightarrow \log_b a > c \Leftrightarrow E > c\]
	\[\begin{matrix}
			E < c & T_{algo}(n) \in \Theta(f(n))                                            \\
			E = c & T_{algo}(n) \in \Theta(f(n)\log_bn) = \Theta(f(n) \log n)               \\
			E > c & T_{algo}(n) \in \Theta(a^{\log_bn}) = \Theta(n^{\log_ba}) = \Theta(n^E) \\
		\end{matrix}\]
}
By master theorem we can easily see $T_{msort}(n) \in \Theta(n \log n)$.
We can also calculate it using a graph:
\centerimage{width=0.4\textwidth}{mergesort.png}

\section*{Quicksort}
\codelist{Haskell}{quick sort.hs}
Note for simplicity, we assume the lists are split into equal parts.
\\ \begin{tabular}{l l }
	$T_{qsort}(0) $ & $= 1$                                                                            \\
	$T_{qsort}(1) $ & $= 1$                                                                            \\
	$T_{qsort}(n) $ & $= T_{partition}(n-1) + T_{++}(\cfrac{n}{2}) + 2 \times T_{qsort}(\cfrac{n}{2})$ \\
\end{tabular}
\\In the worst case, the partition splits $xs$ into $(xs, [])$, we have complexity:
\\ \begin{tabular}{l l }
	$T_{qsort}(n) $ & $= T_{partition}(n-1) + T_{++}(n-1) + T_{qsort}(0) + T_{qsort}(n-1)$ \\
	                & $= 2(n-1) + n + 1 + T_{qsort}(n-1)$                                  \\
\end{tabular}
\centerimage{width=0.4\textwidth}{quicksort.png}
We can once again use master theorem, or a diagram such as below to see the complexity:
Hence in the worst case $T_{qsort}(n) \in O(n^2)$ (same as insertion sort).
\end{document}
