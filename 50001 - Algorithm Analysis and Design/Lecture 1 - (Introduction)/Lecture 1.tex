\documentclass{report}
    \title{50001 - Algorithm Analysis and Design - Lecture 1}
    \author{Oliver Killane}
    \date{12/11/21}

%===========================COMMON FORMAT & COMMANDS===========================
% This file contains commands and format to be used by every module, and is 
% included in all files.
%===============================================================================

%====================================IMPORTS====================================
\usepackage[a4paper, total={6in, 8in}]{geometry}
\usepackage{graphicx, amssymb, amsfonts, amsmath, xcolor, listings, tcolorbox, multirow, hyperref}
%===============================================================================

%====================================IMAGES=====================================
\graphicspath{{image/}}

% \centerimage{options}{image}
\newcommand{\centerimage}[2]{\begin{center}
    \includegraphics[#1]{#2}
\end{center}}
%===============================================================================

%=================================CODE LISTINGS=================================
\definecolor{codebackdrop}{gray}{0.9}
\definecolor{commentgreen}{rgb}{0,0.6,0}
\lstset{
    inputpath=code, 
    commentstyle=\color{commentgreen},
    keywordstyle=\color{blue}, 
    backgroundcolor=\color{codebackdrop}, 
    basicstyle=\footnotesize,
    frame=single,
    numbers=left,
    stepnumber=1,
    showstringspaces=false,
    breaklines=true,
    postbreak=\mbox{\textcolor{red}{$\hookrightarrow$}\space}
}

% Create a code listing for a single line
% \codeline{language}{line}{file}
\newcommand{\codeline}[3]{\lstinputlisting[language=#1, firstline = #2, lastline = #2]{#3}}

% Create a code listing for a given language & file
% \codelist{language}{file}
\newcommand{\codelist}[2]{\lstinputlisting[language=#1]{#2}}
%===============================================================================

%================================TEXT STRUCTURES================================
% Marka a word as bold
% \keyword{important word}
\newcommand{\keyword}[1]{\textbf{#1}}

% Creates a section in italics
% \question{question in italics}
\newcommand{\question}[1]{\textit{#1} \\ }

% Creates a box with title for side notes.
% \sidenote{title}{contents}
\newcommand{\sidenote}[2]{\begin{tcolorbox}[title=#1]#2\end{tcolorbox}}

% Creates an item in an itemize or enumerate, with a paragraph after
% \begin{itemize}
%     \bullpara{title}{contents}
% \end{itemize}
\newcommand{\bullpara}[2]{\item \textbf{#1} \ #2}

% Creates a compact list (very small gaps between items)
% \compitem{
%     \item item 1
%     \item item 2
%     \item ...
% }
\newcommand{\compitem}[1]{\begin{itemize}\setlength\itemsep{-0.5em}#1\end{itemize}}

% Creates a link to the lecture for use at the start of the notes document
\newcommand{\lectlink}[1]{\sidenote{Lecture Recording}{
    Lecture recording is available \href{#1}{here}
}}
%===============================================================================


%=============================DISPLAYING CODE STEPS=============================
\newenvironment{steps}
{\begin{tabular}{l}}
		{\end{tabular}}

\newcommand{\step}[2]{$\leadsto$ \{ #2 \} \\ $#1$ \\}
\newcommand{\start}[1]{$#1$ \\}

%===============================================================================


\begin{document}
\maketitle
\lectlink{https://imperial.cloud.panopto.eu/Panopto/Pages/Viewer.aspx?id=d281ea45-d9df-4f23-8578-adbf00d38370}

\section*{Introduction}
An algorithm is a method of computing a result for a given problem, at its core in a systematic/mathematical means.
\\
\\ This course extensively uses haskell instead of pesudocode to express problems, though its lessons still apply to other languages.

\section*{Fundamentals}
\sidenote{Insertion Problem}{
	Given an integer $x$ and a sorted list $ys$, produce a list containing $x:ys$ that is ordered.
	\\
	\\ Note that this can be solved by simply using $sort(x:ys)$ however this is considered wasteful as it does not exploit the fact that $ys$ is already sorted.
}
An example algorithm would be to traverse $ys$ until we find a suitable place for $x$
\codelist{Haskell}{insert.hs}
\subsubsection*{Call Steps}
In order to determine the complexity of the function, we use a \keyword{cost model} and determine what steps must be taken.
\\
\\ For example for $insert \ 4 \ [1,3,6,7]$
\\ \begin{steps}
	\start{insert \ 4 \ [1,3,6,7]}
	\step{1 : insert \ 4 \ [3,6,7]}{definition of $insert$}
	\step{1 : 3 : insert \ 4 \ [6,7]}{definition of $insert$}
	\step{1 : 3 : : 4 : [6,7]}{definition of $insert$}
\end{steps}
Hence this requires $3$ call steps.
\\
\\ We can use recurrence relations to get a generalised formula for the worst case (maximum number of calls):
\[\begin{matrix}
		T_{insert} 0 & = 1                   \\
		T_{insert} 1 & = 1 + T_{insert}(n-1) \\
	\end{matrix}\]
We can solve the recurrence:
\[\begin{matrix}
		T_{insert} (n) & = 1 + T_{insert}(n-1)                 \\
		T_{insert} (n) & = 1 + 1 + T_{insert}(n-2)             \\
		T_{insert} (n) & = 1 + 1 + \dots + 1 + T_{insert}(n-n) \\
		T_{insert} (n) & = n + T_{insert}(0)                   \\
		T_{insert} (n) & = n + 1                               \\
	\end{matrix}\]

\subsubsection*{More complex algorithms}
\codelist{Haskell}{isort.hs}
\[\begin{matrix}
		T_{isort} 0 & = 1                                    \\
		T_{isort} n & = 1 + T_{insert}(n-1) + T_{isort}(n-1) \\
	\end{matrix}\]
Hence by using our previous formula for $insert$
\[\begin{matrix}
		T_{isort} n & = 1 + n + T_{isort}(n-1) \\
	\end{matrix}\]
And by recurrence:
\[\begin{matrix}
		T_{isort} (n) & = 1 + n + (1 + n - 1) + T_{isort}(n-2)                       \\
		T_{isort} (n) & = 1 + n + (1 + n - 1) + (1 + n - 2) + \dots + T_{isort}(n-n) \\
		T_{isort} (n) & = 1 + n + (1 + n - 1) + (1 + n - 2) + \dots + T_{isort}(0)   \\
		T_{isort} (n) & = n + n + (n - 1) + (n - 2) + \dots + (n-n) + 1              \\
		T_{isort} (n) & = 1 + n + \sum_{i = 0}^n{i}                                  \\
		T_{isort} (n) & =  \sum_{i = 0}^{n+1}{i}                                     \\
		T_{isort} (n) & =  \cfrac{(n+1)\times(n+1)}{2}                               \\
	\end{matrix}\]
\end{document}
