\documentclass{report}
    \title{50001 - Algorithm Analysis and Design - Lecture 12}
    \author{Oliver Killane}
    \date{18/11/21}

%===========================COMMON FORMAT & COMMANDS===========================
% This file contains commands and format to be used by every module, and is 
% included in all files.
%===============================================================================

%====================================IMPORTS====================================
\usepackage[a4paper, total={6in, 8in}]{geometry}
\usepackage{graphicx, amssymb, amsfonts, amsmath, xcolor, listings, tcolorbox, multirow, hyperref}
%===============================================================================

%====================================IMAGES=====================================
\graphicspath{{image/}}

% \centerimage{options}{image}
\newcommand{\centerimage}[2]{\begin{center}
    \includegraphics[#1]{#2}
\end{center}}
%===============================================================================

%=================================CODE LISTINGS=================================
\definecolor{codebackdrop}{gray}{0.9}
\definecolor{commentgreen}{rgb}{0,0.6,0}
\lstset{
    inputpath=code, 
    commentstyle=\color{commentgreen},
    keywordstyle=\color{blue}, 
    backgroundcolor=\color{codebackdrop}, 
    basicstyle=\footnotesize,
    frame=single,
    numbers=left,
    stepnumber=1,
    showstringspaces=false,
    breaklines=true,
    postbreak=\mbox{\textcolor{red}{$\hookrightarrow$}\space}
}

% Create a code listing for a single line
% \codeline{language}{line}{file}
\newcommand{\codeline}[3]{\lstinputlisting[language=#1, firstline = #2, lastline = #2]{#3}}

% Create a code listing for a given language & file
% \codelist{language}{file}
\newcommand{\codelist}[2]{\lstinputlisting[language=#1]{#2}}
%===============================================================================

%================================TEXT STRUCTURES================================
% Marka a word as bold
% \keyword{important word}
\newcommand{\keyword}[1]{\textbf{#1}}

% Creates a section in italics
% \question{question in italics}
\newcommand{\question}[1]{\textit{#1} \\ }

% Creates a box with title for side notes.
% \sidenote{title}{contents}
\newcommand{\sidenote}[2]{\begin{tcolorbox}[title=#1]#2\end{tcolorbox}}

% Creates an item in an itemize or enumerate, with a paragraph after
% \begin{itemize}
%     \bullpara{title}{contents}
% \end{itemize}
\newcommand{\bullpara}[2]{\item \textbf{#1} \ #2}

% Creates a compact list (very small gaps between items)
% \compitem{
%     \item item 1
%     \item item 2
%     \item ...
% }
\newcommand{\compitem}[1]{\begin{itemize}\setlength\itemsep{-0.5em}#1\end{itemize}}

% Creates a link to the lecture for use at the start of the notes document
\newcommand{\lectlink}[1]{\sidenote{Lecture Recording}{
    Lecture recording is available \href{#1}{here}
}}
%===============================================================================


%=============================DISPLAYING CODE STEPS=============================
\newenvironment{steps}
{\begin{tabular}{l}}
		{\end{tabular}}

\newcommand{\step}[2]{$\leadsto$ \{ #2 \} \\ $#1$ \\}
\newcommand{\start}[1]{$#1$ \\}

%===============================================================================


\begin{document}
    \maketitle
    \lectlink{https://imperial.cloud.panopto.eu/Panopto/Pages/Viewer.aspx?id=abdb5a60-fd1c-44bb-afbe-ade100fb80eb}

    \section*{Red-Black Trees}
        \keyword{AVL trees} worked by storing an extra integer (height) to use in rebalancing, \keyword{red-black trees} use an extra bit to determine if a node is red or black.
        \\
        \\ In practice they are less balanced than \keyword{AVL trees} however the insertion is faster and the data structure is a little bit smaller.
        \codelist{Haskell}{red-black tree.hs}
        The structure relies on two invariances:
        \begin{enumerate}
            \item Every Red node must have a Black parent node.
            \item Every path from the root to leaf must have the same number of black nodes.
        \end{enumerate}
        \subsubsection*{Valid Red Black Trees}
            \centerimage{width=\textwidth}{red-black trees.png}
        \subsubsection*{Invalid Red Black Trees}
            \centerimage{width=\textwidth}{not red-black trees.png}
        We have an insert function that needs to rebalance the tree:
        \codelist{Haskell}{red-black insert.hs} 
        \centerimage{width=\textwidth}{balance cases.png}
    
    \section*{Counting}
        We can exploit the analogy we used with counting and trees for \keyword{RALists} here, with a difference.
        \\
        \\ Imagine a counting system that lacks zeros. We can count to 10 as:
        \begin{center}
            \begin{tabular}{l l l l l l l l l l l l l l l l l l}
                Normal:  & 1 & 2 & \dots & 9 & 10 & \dots & 11 & 12 & \dots & 19 & 20 & \dots & 101 & 102 & \dots & 110 & 111 \\
                Special: & 1 & 2 & \dots & 9 & X  & \dots & 11 & 12 & \dots & 19 & 1X & \dots & X1  & X2  & \dots & XX  & 111 \\
            \end{tabular}
        \end{center}
        We can use this with the pattern of inserting elements into a red black tree (in order) to map red black trees to an incrementing number.
\end{document}
