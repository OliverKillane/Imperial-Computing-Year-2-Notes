\documentclass{report}
    \title{50001 - Algorithm Analysis and Design - Lecture 8}
    \author{Oliver Killane}
    \date{17/11/21}

%===========================COMMON FORMAT & COMMANDS===========================
% This file contains commands and format to be used by every module, and is 
% included in all files.
%===============================================================================

%====================================IMPORTS====================================
\usepackage[a4paper, total={6in, 8in}]{geometry}
\usepackage{graphicx, amssymb, amsfonts, amsmath, xcolor, listings, tcolorbox, multirow, hyperref}
%===============================================================================

%====================================IMAGES=====================================
\graphicspath{{image/}}

% \centerimage{options}{image}
\newcommand{\centerimage}[2]{\begin{center}
    \includegraphics[#1]{#2}
\end{center}}
%===============================================================================

%=================================CODE LISTINGS=================================
\definecolor{codebackdrop}{gray}{0.9}
\definecolor{commentgreen}{rgb}{0,0.6,0}
\lstset{
    inputpath=code, 
    commentstyle=\color{commentgreen},
    keywordstyle=\color{blue}, 
    backgroundcolor=\color{codebackdrop}, 
    basicstyle=\footnotesize,
    frame=single,
    numbers=left,
    stepnumber=1,
    showstringspaces=false,
    breaklines=true,
    postbreak=\mbox{\textcolor{red}{$\hookrightarrow$}\space}
}

% Create a code listing for a single line
% \codeline{language}{line}{file}
\newcommand{\codeline}[3]{\lstinputlisting[language=#1, firstline = #2, lastline = #2]{#3}}

% Create a code listing for a given language & file
% \codelist{language}{file}
\newcommand{\codelist}[2]{\lstinputlisting[language=#1]{#2}}
%===============================================================================

%================================TEXT STRUCTURES================================
% Marka a word as bold
% \keyword{important word}
\newcommand{\keyword}[1]{\textbf{#1}}

% Creates a section in italics
% \question{question in italics}
\newcommand{\question}[1]{\textit{#1} \\ }

% Creates a box with title for side notes.
% \sidenote{title}{contents}
\newcommand{\sidenote}[2]{\begin{tcolorbox}[title=#1]#2\end{tcolorbox}}

% Creates an item in an itemize or enumerate, with a paragraph after
% \begin{itemize}
%     \bullpara{title}{contents}
% \end{itemize}
\newcommand{\bullpara}[2]{\item \textbf{#1} \ #2}

% Creates a compact list (very small gaps between items)
% \compitem{
%     \item item 1
%     \item item 2
%     \item ...
% }
\newcommand{\compitem}[1]{\begin{itemize}\setlength\itemsep{-0.5em}#1\end{itemize}}

% Creates a link to the lecture for use at the start of the notes document
\newcommand{\lectlink}[1]{\sidenote{Lecture Recording}{
    Lecture recording is available \href{#1}{here}
}}
%===============================================================================


%=============================DISPLAYING CODE STEPS=============================
\newenvironment{steps}
{\begin{tabular}{l}}
		{\end{tabular}}

\newcommand{\step}[2]{$\leadsto$ \{ #2 \} \\ $#1$ \\}
\newcommand{\start}[1]{$#1$ \\}

%===============================================================================


\begin{document}
    \maketitle
    \lectlink{https://imperial.cloud.panopto.eu/Panopto/Pages/Viewer.aspx?id=77a4fe23-79fb-4468-8f5c-add300eedc4d}

    \section*{Amortized Analysis}
        So far we have studied complexity of a single, isolated run of an algorithm. \keyword{Amortizsed Analysis} is about understanding cost in a wider context (e.g averaged over many calls to a routine).
    
    \section*{Dequeues}
        \sidenote{Dequeue}{
            A Dequeue is a double ended queue. An abstract datatype that generalises a queue. Elements can be added or removed from either end.
            \\
            \\ Common associated funtions are:
            \begin{center}
                \begin{tabular}{l l}
                    \fun{snoc} & Insert element at the back of the queue. \\
                    \fun{cons} & Insert element at the front of the queue. \\
                    \fun{eject} & Remove last element. \\
                    \fun{pop} & remove fist element. \\
                    \fun{peek} & Examine but do not remove first element. \\
                \end{tabular}
            \end{center}
            Dequeues are also called head-tail linked lists or symmetric lists.
        }
        we use a dequeue when we want to reduce the time taken to perform certain operations.
        \subsubsection*{List Operation Complexity}
            \centerimage{width=0.4\textwidth}{list anatomy.png}
            \begin{minipage}[t]{0.4\textwidth}
                \begin{center}
                    \begin{tabular}{l l}
                        \fun{cons} & O(1) \\
                        \fun{head} & O(1) \\
                        \fun{tail} & O(1) \\
                    \end{tabular}
                \end{center}
            \end{minipage}
            \hfill
            \begin{minipage}[t]{0.4\textwidth}
                \begin{center}
                    \begin{tabular}{l l}
                        \fun{snoc} & O(1) \\
                        \fun{last} & O(1) \\
                        \fun{init} & O(1) \\
                    \end{tabular}
                \end{center}
            \end{minipage}
        \subsubsection*{Dequeue Structure}
            To achieve $O(1)$ complexity in the \fun{snoc}, \fun{init} and \fun{last} we use two lists.
            \centerimage{width=0.7\textwidth}{dequeue structure.png}
            One list starts contains the start of the list, and the other the end (reversed).
            \\
            \\ We keep two invariants for $Dequeue \ us \ sv$:
            \[null \ us \Rightarrow null \ sv \lor single \ sv\]
            \[null \ sv \Rightarrow null \ us \lor single \ us\]
            In other words, if one list is empty, the other can contain at most 1 element.
            \\
            \\ An example implementation in haskell is below:
            \codelist{Haskell}{dequeue declaration.hs}
            
            When considering the cost of \fun{tail} and \fun{init} we must consider that there are two possibilities:
            \begin{center}
                \begin{tabular}{l l p{7cm}}
                    High Cost & $\begin{matrix}
                        init (Dequeue us [s]) \\
                        tail (Dequeue [u] sv) \\
                    \end{matrix}$
                    & This operation is $O(n)$ complexity due to the \fun{spitAt} and \fun{reverse} operation done on half of a list. \\
                    Low Cost & $\begin{matrix}
                        init (Dequeue us (s:sv)) \\
                        tail (Dequeue (u:us) sv) \\
                    \end{matrix}$ & Low cost $O(1)$ operation as it requires only a pattern match on the first element. \\
                \end{tabular}
            \end{center}
            As both of these operations rebalance the \keyword{Dequeue} to to be balanced (half the queue on each list), we these operations can have an amortized cost of $O(1)$.
            \\
            \\ We know this as the average cost is of order $O(1)$. The $O(n)$ cost is incurred every $n/2$ calls to \fun{tail}/\fun{init}.
            \centerimage{width=0.6\textwidth}{dequeue tail.png}
        
        \section*{Amortization}
            The complexity of \fun{tail} is an example of \keyword{Amortized analysis}, where operation's wider context are considered when calculating the complexity.
            \[xs_0 \overset{op_0}{\leadsto} xs_1 \overset{op_1}{\leadsto} xs_2 \overset{op_2}{\leadsto} xs_3 \overset{op_3}{\leadsto} \dots \overset{op_{n-1}}{\leadsto} xs_n\]
            We defined 3 parts:
            \begin{enumerate}
                \bullpara{Cost Function}{
                    \\ $C_{op_i}(xs_i)$ determines the cost of operation $op_i$ on data $xs_i$. Estimating how many steps it takes for each operation to execute.
                }
                \bullpara{Amortized Cost Function}{
                    \\ $A_{op_i}(xs_i)$ for each operation $op_i$ on data $xs_i$.
                }
                \bullpara{Size Function}{
                    \\ $S(xs)$ that calculates the size of data $xs$
                }
            \end{enumerate}
            We define these functions with the goal to show that:
            \[C_{op_i}(xs_i) \leq A_{op_i} (xs_i) + S(xs_i) - S(xs_{i+1})\]
            The cost of the operation is smaller than the amortized cost, plus the difference in size of the data structure before and after the operation.
            \\
            \\ Once this is shown, we can infer that:
            \[\sum^{n-1}_{i=0}C_{op_i}(xs_i) \leq \sum^{n-1}_{i=0}A_{op_i}(xs_i) + S(xs_0) - S(xs_n)\]
            Furthermore when $S(xs_0) = 0$ this implies:
            \[\sum^{n-1}_{i=0}C_{op_i}(xs_i) \leq \sum^{n-1}_{i=0}A_{op_i}(xs_i) - S(xs_n) \Rightarrow \sum^{n-1}_{i=0}C_{op_i}(xs_i) \leq \sum^{n-1}_{i=0}A_{op_i}(xs_i)\]
            This means the cost of the operations is less than the sum of the amortized costs.
            \\
            \\ For example, if $A_{op_i}(xs) = 1$ then the total cost will be bounded by $O(n)$.
            \\
            \\ We can use the \keyword{dequeue} operations as an example:
            \[\begin{matrix}
                C_{cons}(xs) = 1 & C_{snoc}(xs) = 1 & C_{head}(xs) = 1 & C_{last}(xs) = 1 \\
            \end{matrix}\]
            For tail we can do the following:
            \\ \begin{proof}
                \proofstep{1}{$C_{tail}(Dequeue \ us \ sv) = length \ sv$}{Create a cost function of \fun{tail}.}
                \proofstep{2}{$A_{op}(xs) = 2$}{Create an arbitrary cost function.}
                \proofstep{3}{$S(Dequeue \ us \ sv) = |length \ us - length \ sv|$}{Create a size function for \keyword{dequeue}.}
                \proofstep{4}{$Dequeue \ us \ sv \text{ where } length \ sv = k$}{Worst case where $us$ is a singleton}
                \proofstep{5}{$\begin{matrix}
                    S(Dequeue \ us \ sv) = k - 1 \\
                    S(Dequeue \ us' \ sv') = 1 \\
                \end{matrix}$}{Size of the next data structure can be at most $1$.}
                \proofstep{6}{$C_{tail}(Dequeue \ us \ sv) = k$}{Calculate the worst case cost of \fun{tail}.}
                \proofstep{7}{$k \leq 2 + (k-1) - 1 = k + 2$}{As this inequality holds, the time complexity of all $n$ instructions is $O(n)$.}
            \end{proof}
            As the time complexity of all $n$ instructions together is $O(n)$, the amortized cost of a single instruction is $O(1)$.
\end{document}
