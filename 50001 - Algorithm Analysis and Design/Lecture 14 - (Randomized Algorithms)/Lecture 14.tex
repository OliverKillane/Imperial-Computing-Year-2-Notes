\documentclass{report}
    \title{50001 - Algorithm Analysis and Design - Lecture 14}
    \author{Oliver Killane}
    \date{28/11/21}
%===========================COMMON FORMAT & COMMANDS===========================
% This file contains commands and format to be used by every module, and is 
% included in all files.
%===============================================================================

%====================================IMPORTS====================================
\usepackage[a4paper, total={6in, 8in}]{geometry}
\usepackage{graphicx, amssymb, amsfonts, amsmath, xcolor, listings, tcolorbox, multirow, hyperref}
%===============================================================================

%====================================IMAGES=====================================
\graphicspath{{image/}}

% \centerimage{options}{image}
\newcommand{\centerimage}[2]{\begin{center}
    \includegraphics[#1]{#2}
\end{center}}
%===============================================================================

%=================================CODE LISTINGS=================================
\definecolor{codebackdrop}{gray}{0.9}
\definecolor{commentgreen}{rgb}{0,0.6,0}
\lstset{
    inputpath=code, 
    commentstyle=\color{commentgreen},
    keywordstyle=\color{blue}, 
    backgroundcolor=\color{codebackdrop}, 
    basicstyle=\footnotesize,
    frame=single,
    numbers=left,
    stepnumber=1,
    showstringspaces=false,
    breaklines=true,
    postbreak=\mbox{\textcolor{red}{$\hookrightarrow$}\space}
}

% Create a code listing for a single line
% \codeline{language}{line}{file}
\newcommand{\codeline}[3]{\lstinputlisting[language=#1, firstline = #2, lastline = #2]{#3}}

% Create a code listing for a given language & file
% \codelist{language}{file}
\newcommand{\codelist}[2]{\lstinputlisting[language=#1]{#2}}
%===============================================================================

%================================TEXT STRUCTURES================================
% Marka a word as bold
% \keyword{important word}
\newcommand{\keyword}[1]{\textbf{#1}}

% Creates a section in italics
% \question{question in italics}
\newcommand{\question}[1]{\textit{#1} \\ }

% Creates a box with title for side notes.
% \sidenote{title}{contents}
\newcommand{\sidenote}[2]{\begin{tcolorbox}[title=#1]#2\end{tcolorbox}}

% Creates an item in an itemize or enumerate, with a paragraph after
% \begin{itemize}
%     \bullpara{title}{contents}
% \end{itemize}
\newcommand{\bullpara}[2]{\item \textbf{#1} \ #2}

% Creates a compact list (very small gaps between items)
% \compitem{
%     \item item 1
%     \item item 2
%     \item ...
% }
\newcommand{\compitem}[1]{\begin{itemize}\setlength\itemsep{-0.5em}#1\end{itemize}}

% Creates a link to the lecture for use at the start of the notes document
\newcommand{\lectlink}[1]{\sidenote{Lecture Recording}{
    Lecture recording is available \href{#1}{here}
}}
%===============================================================================


%=============================DISPLAYING CODE STEPS=============================
\newenvironment{steps}
{\begin{tabular}{l}}
		{\end{tabular}}

\newcommand{\step}[2]{$\leadsto$ \{ #2 \} \\ $#1$ \\}
\newcommand{\start}[1]{$#1$ \\}

%===============================================================================

\begin{document}
    \maketitle
    \lectlink{https://imperial.cloud.panopto.eu/Panopto/Pages/Viewer.aspx?id=a562661f-da77-4d28-85a6-ade80168e6e5}

    \section*{Randomized Algorithms}
        An algorithm that uses random values to produce a result.
        \begin{center}
            \begin{tabular}{l l l}
                \textbf{Algorithm Type} & \textbf{Running time} & \textbf{Correct Result} \\
                Monte Carlo & Predicatable & Unpredictably \\
                Las Vegas & Unpredictable & Predictably \\
            \end{tabular}
        \end{center}
    
    \section*{Random Generation}
        
        Functions are deterministic (always map same inputs to same outputs), this is known as \keyword{Leibniz's law} or the \keyword{Law of indiscernibles}:
        \[x = y \Rightarrow f x = f y\]
        We can exhibit pesudo random behaviour using an input that varies \begin{tabular}{l l}
            explicitly & (e.g Random numbers through seeds) \\
            implicitly & (e.g Microphone or camera noise) \\
        \end{tabular}

        \subsection*{Inside IO Monad}
            We can use basic random through the IO monad like this:
            \codelist{Haskell}{simple random.hs}
            However using the \struct{IO monad} is too specific, we may want to use random numbers in other contexts.
        
        \subsection*{StdGen}
            In haskell we can use \struct{Stdgen}.
            \codelist{Haskell}{stdgen.hs}
            By passing the newly generated \struct{StdGen} we can generate new values based on the original seed.
            \centerimage{width=\textwidth}{random seeds.png}

        \subsection*{With Random Monad}
            Rather than passing \struct{StdGen} seeds through the program, we can use the \struct{MonadRandom} monad which internally uses this value.

        
    \section*{Randomized $\pi$}
        \centerimage{width=0.6\textwidth}{randomized pi.png}
        (Monte Carlo Algorithm - known number of samples, known running time per sample) To estimate $\pi$, find the proportion of randomly selected spots that are within the circle.
        \begin{center}
            \begin{tabular}{l l}
                Area of square & $2 \times 2 = 4$ \\
                Area of circle & $\pi \times 1^2 = \pi$ \\
                Probability in circle & $\cfrac{\pi}{4}$ \\
            \end{tabular}
        \end{center}
        Once we have the proportion, we can multiply by 4 to get an estimate of $\pi$.
        \codelist{Haskell}{montePi.hs}

    \section*{Treaps}
        Simultaneously a \keyword{Tree} and a \keyword{Heap}. Stores values in order, while promoting higher priority nodes to the top of the tree.
        \centerimage{width=0.8\textwidth}{treap.png}
        \codelist{Haskell}{treap.hs}
        \centerimage{width=0.8\textwidth}{lnode and rnode.png}
\end{document}
