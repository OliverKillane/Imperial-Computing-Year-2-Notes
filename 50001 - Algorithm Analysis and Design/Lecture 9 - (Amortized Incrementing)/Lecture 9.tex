\documentclass{report}
    \title{50001 - Algorithm Analysis and Design - Lecture 9}
    \author{Oliver Killane}
    \date{17/11/21}

%===========================COMMON FORMAT & COMMANDS===========================
% This file contains commands and format to be used by every module, and is 
% included in all files.
%===============================================================================

%====================================IMPORTS====================================
\usepackage[a4paper, total={6in, 8in}]{geometry}
\usepackage{graphicx, amssymb, amsfonts, amsmath, xcolor, listings, tcolorbox, multirow, hyperref}
%===============================================================================

%====================================IMAGES=====================================
\graphicspath{{image/}}

% \centerimage{options}{image}
\newcommand{\centerimage}[2]{\begin{center}
    \includegraphics[#1]{#2}
\end{center}}
%===============================================================================

%=================================CODE LISTINGS=================================
\definecolor{codebackdrop}{gray}{0.9}
\definecolor{commentgreen}{rgb}{0,0.6,0}
\lstset{
    inputpath=code, 
    commentstyle=\color{commentgreen},
    keywordstyle=\color{blue}, 
    backgroundcolor=\color{codebackdrop}, 
    basicstyle=\footnotesize,
    frame=single,
    numbers=left,
    stepnumber=1,
    showstringspaces=false,
    breaklines=true,
    postbreak=\mbox{\textcolor{red}{$\hookrightarrow$}\space}
}

% Create a code listing for a single line
% \codeline{language}{line}{file}
\newcommand{\codeline}[3]{\lstinputlisting[language=#1, firstline = #2, lastline = #2]{#3}}

% Create a code listing for a given language & file
% \codelist{language}{file}
\newcommand{\codelist}[2]{\lstinputlisting[language=#1]{#2}}
%===============================================================================

%================================TEXT STRUCTURES================================
% Marka a word as bold
% \keyword{important word}
\newcommand{\keyword}[1]{\textbf{#1}}

% Creates a section in italics
% \question{question in italics}
\newcommand{\question}[1]{\textit{#1} \\ }

% Creates a box with title for side notes.
% \sidenote{title}{contents}
\newcommand{\sidenote}[2]{\begin{tcolorbox}[title=#1]#2\end{tcolorbox}}

% Creates an item in an itemize or enumerate, with a paragraph after
% \begin{itemize}
%     \bullpara{title}{contents}
% \end{itemize}
\newcommand{\bullpara}[2]{\item \textbf{#1} \ #2}

% Creates a compact list (very small gaps between items)
% \compitem{
%     \item item 1
%     \item item 2
%     \item ...
% }
\newcommand{\compitem}[1]{\begin{itemize}\setlength\itemsep{-0.5em}#1\end{itemize}}

% Creates a link to the lecture for use at the start of the notes document
\newcommand{\lectlink}[1]{\sidenote{Lecture Recording}{
    Lecture recording is available \href{#1}{here}
}}
%===============================================================================


%=============================DISPLAYING CODE STEPS=============================
\newenvironment{steps}
{\begin{tabular}{l}}
		{\end{tabular}}

\newcommand{\step}[2]{$\leadsto$ \{ #2 \} \\ $#1$ \\}
\newcommand{\start}[1]{$#1$ \\}

%===============================================================================


\begin{document}
\maketitle
\lectlink{https://imperial.cloud.panopto.eu/Panopto/Pages/Viewer.aspx?id=83376d33-8332-4d64-abce-add60165e432}
\section*{Amortization}
The complexity of \fun{tail} is an example of \keyword{Amortized analysis}, where operation's wider context are considered when calculating the complexity.
\[xs_0 \overset{op_0}{\leadsto} xs_1 \overset{op_1}{\leadsto} xs_2 \overset{op_2}{\leadsto} xs_3 \overset{op_3}{\leadsto} \dots \overset{op_{n-1}}{\leadsto} xs_n\]
We defined 3 parts:
\begin{enumerate}
	\bullpara{Cost Function}{
		\\ $C_{op_i}(xs_i)$ determines the cost of operation $op_i$ on data $xs_i$. Estimating how many steps it takes for each operation to execute.
	}
	\bullpara{Amortized Cost Function}{
		\\ $A_{op_i}(xs_i)$ for each operation $op_i$ on data $xs_i$.
	}
	\bullpara{Size Function}{
		\\ $S(xs)$ that calculates the size of data $xs$
	}
\end{enumerate}
We define these functions with the goal to show that:
\[C_{op_i}(xs_i) \leq A_{op_i} (xs_i) + S(xs_i) - S(xs_{i+1})\]
The cost of the operation is smaller than the amortized cost, plus the difference in size of the data structure before and after the operation.
\\
\\ Once this is shown, we can infer that:
\[\sum^{n-1}_{i=0}C_{op_i}(xs_i) \leq \sum^{n-1}_{i=0}A_{op_i}(xs_i) + S(xs_0) - S(xs_n)\]
Furthermore when $S(xs_0) = 0$ this implies:
\[\sum^{n-1}_{i=0}C_{op_i}(xs_i) \leq \sum^{n-1}_{i=0}A_{op_i}(xs_i) - S(xs_n) \Rightarrow \sum^{n-1}_{i=0}C_{op_i}(xs_i) \leq \sum^{n-1}_{i=0}A_{op_i}(xs_i)\]
This means the cost of the operations is less than the sum of the amortized costs.
\\
\\ For example, if $A_{op_i}(xs) = 1$ then the total cost will be bounded by $O(n)$.

\subsubsection*{Tail example}
\[\begin{matrix}
		C_{cons}(xs) = 1 & C_{snoc}(xs) = 1 & C_{head}(xs) = 1 & C_{last}(xs) = 1 \\
	\end{matrix}\]
For tail we can do the following:
\\ \begin{proof}
	\proofstep{1}{$C_{tail}(Dequeue \ us \ sv) = length \ sv$}{Create a cost function of \fun{tail}.}
	\proofstep{2}{$A_{op}(xs) = 2$}{Create an arbitrary cost function.}
	\proofstep{3}{$S(Dequeue \ us \ sv) = |length \ us - length \ sv|$}{Create a size function for \keyword{dequeue}.}
	\proofstep{4}{$Dequeue \ us \ sv \text{ where } length \ sv = k$}{Worst case where $us$ is a singleton}
	\proofstep{5}{$\begin{matrix}
				S(Dequeue \ us \ sv) = k - 1 \\
				S(Dequeue \ us' \ sv') = 1   \\
			\end{matrix}$}{Size of the next data structure can be at most $1$.}
	\proofstep{6}{$C_{tail}(Dequeue \ us \ sv) = k$}{Calculate the worst case cost of \fun{tail}.}
	\proofstep{7}{$k \leq 2 + (k-1) - 1 = k + 2$}{As this inequality holds, the time complexity of all $n$ instructions is $O(n)$.}
\end{proof}
As the time complexity of all $n$ instructions together is $O(n)$, the amortized cost of a single instruction is $O(1)$.

\subsubsection*{About the size function}
We want to balance the size function such that:
\compitem{
	\item The size function is $0$ to start with.
	\item The size between operations is large enough to prove the inequality.
}
The size function is arbitrary, if you cannot choose a size function that satisfied the goal inequality, then you're probably making a mistake

\subsubsection*{Peano Numbers}
\codelist{Haskell}{peano.hs}
This shows how similar operations of similarly structured data can be.

\section*{Binary Numbers}
\codelist{Haskell}{binary.hs}
we can do amortized analysis on incr:
\begin{proof}
	\proofstep{1}{$C_{incr}(bs) = t + 1$ where $t = length \ (takeWhile \ (==I) \ bs)$}{Create a cost function.}
	\proofstep{2}{$A_{incr}(bs) = 2$}{Create Amortized Cost}
	\proofstep{3}{$S(bs) = length . filter \ (==I) \ \$ \ bs $}{Create size function.}
	\proofstep{4}{Given $bs' = incr \ bs$ we show $C_{incr}(bs) \leq A_{incr}{bs} + S(bs) - S(bs')$}{Setup inequality}
	\proofstep{5}{$t + 1 \leq 2 + S(bs) - (S(bs) - t + 1)$}{Subsitiute in inequality}
	\proofstep{6}{$t + 1 \leq 1 + t$}{Hence inequality holds}
	\proofstep{7}{$S(start) = 0$}{Start size is zero.}
\end{proof}
Hence The sum of $C$ is smaller than the sum of $A$, as this is over $n$ operations and $\sum A = 2n$, $C_{incr}(bs) \in O(1)$.
\end{document}
