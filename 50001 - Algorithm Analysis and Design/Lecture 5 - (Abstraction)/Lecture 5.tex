\documentclass{report}
    \title{50001 - Algorithm Analysis and Design - Lecture 5}
    \author{Oliver Killane}
    \date{12/11/21}

%===========================COMMON FORMAT & COMMANDS===========================
% This file contains commands and format to be used by every module, and is 
% included in all files.
%===============================================================================

%====================================IMPORTS====================================
\usepackage[a4paper, total={6in, 8in}]{geometry}
\usepackage{graphicx, amssymb, amsfonts, amsmath, xcolor, listings, tcolorbox, multirow, hyperref}
%===============================================================================

%====================================IMAGES=====================================
\graphicspath{{image/}}

% \centerimage{options}{image}
\newcommand{\centerimage}[2]{\begin{center}
    \includegraphics[#1]{#2}
\end{center}}
%===============================================================================

%=================================CODE LISTINGS=================================
\definecolor{codebackdrop}{gray}{0.9}
\definecolor{commentgreen}{rgb}{0,0.6,0}
\lstset{
    inputpath=code, 
    commentstyle=\color{commentgreen},
    keywordstyle=\color{blue}, 
    backgroundcolor=\color{codebackdrop}, 
    basicstyle=\footnotesize,
    frame=single,
    numbers=left,
    stepnumber=1,
    showstringspaces=false,
    breaklines=true,
    postbreak=\mbox{\textcolor{red}{$\hookrightarrow$}\space}
}

% Create a code listing for a single line
% \codeline{language}{line}{file}
\newcommand{\codeline}[3]{\lstinputlisting[language=#1, firstline = #2, lastline = #2]{#3}}

% Create a code listing for a given language & file
% \codelist{language}{file}
\newcommand{\codelist}[2]{\lstinputlisting[language=#1]{#2}}
%===============================================================================

%================================TEXT STRUCTURES================================
% Marka a word as bold
% \keyword{important word}
\newcommand{\keyword}[1]{\textbf{#1}}

% Creates a section in italics
% \question{question in italics}
\newcommand{\question}[1]{\textit{#1} \\ }

% Creates a box with title for side notes.
% \sidenote{title}{contents}
\newcommand{\sidenote}[2]{\begin{tcolorbox}[title=#1]#2\end{tcolorbox}}

% Creates an item in an itemize or enumerate, with a paragraph after
% \begin{itemize}
%     \bullpara{title}{contents}
% \end{itemize}
\newcommand{\bullpara}[2]{\item \textbf{#1} \ #2}

% Creates a compact list (very small gaps between items)
% \compitem{
%     \item item 1
%     \item item 2
%     \item ...
% }
\newcommand{\compitem}[1]{\begin{itemize}\setlength\itemsep{-0.5em}#1\end{itemize}}

% Creates a link to the lecture for use at the start of the notes document
\newcommand{\lectlink}[1]{\sidenote{Lecture Recording}{
    Lecture recording is available \href{#1}{here}
}}
%===============================================================================


%=============================DISPLAYING CODE STEPS=============================
\newenvironment{steps}
{\begin{tabular}{l}}
		{\end{tabular}}

\newcommand{\step}[2]{$\leadsto$ \{ #2 \} \\ $#1$ \\}
\newcommand{\start}[1]{$#1$ \\}

%===============================================================================


\begin{document}
\maketitle
\lectlink{https://imperial.cloud.panopto.eu/Panopto/Pages/Viewer.aspx?id=7cfe76ce-2e38-465f-a1d0-adc800b8a764}

\section*{DLists Continued\dots}
\subsubsection*{Monoids (again)}
A \keyword{monoid} is a triple $(M, \diamond, \epsilon)$ where $\diamond$ is associative and of type $M \to M \to M$, and $x \diamond \epsilon \equiv x$.
\codelist{Haskell}{monoid.hs}
A haskell typeclass can then be instantiated for many other data types. For example the \keyword{monoid} $(\mathbb{Z}, +, 0)$ (note that we cannot enforce \keyword{monoid} properties through haskell, unlike languages such as \keyword{agda}).
\codelist{Haskell}{integer monoid.hs}
Likewise we can abstract Lists to a class (which we can instantiate for DLists).

\subsubsection*{List Class}
\codelist{Haskell}{list class.hs}
$[a]$ is out abstract list type, and $list a$ is our concrete type.
\\
\\ It is critical to ensure that $toList \ \bullet \ fromList \equiv id$
\\ But in general $fromList \ \bullet \ toList \not\equiv id$ (this is as the internal representation may change
and much information about the internal representaion cannot be preserved by toList, for example an unbalanced tree
changed to a list maybe be balanced when converted back to a tree).
\\
\\ We also included $normalise \ :: \ fromList \ \bullet \ toList$ as a useful tool to reset the internal structure (for example to rebalanced the tree representation of a list)

\section*{Haskell Implementation}
To prevent conflicts due to Prelude functions already being defined we can use:
\codelist{Haskell}{import prelude.hs}
To help ensure correctness we can use \keyword{Quickcheck} to check properties
\codelist{Bash}{get quickcheck.sh}
Then can use quickcheck to define properties we want to test:
\codelist{Haskell}{use quickcheck.hs}
\codelist{Bash}{run quickcheck.sh}
\end{document}
