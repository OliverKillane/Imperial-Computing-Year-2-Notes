\documentclass{report}
    \title{50001 - Algorithm Analysis and Design - Lecture 11}
    \author{Oliver Killane}
    \date{18/11/21}

%===========================COMMON FORMAT & COMMANDS===========================
% This file contains commands and format to be used by every module, and is 
% included in all files.
%===============================================================================

%====================================IMPORTS====================================
\usepackage[a4paper, total={6in, 8in}]{geometry}
\usepackage{graphicx, amssymb, amsfonts, amsmath, xcolor, listings, tcolorbox, multirow, hyperref}
%===============================================================================

%====================================IMAGES=====================================
\graphicspath{{image/}}

% \centerimage{options}{image}
\newcommand{\centerimage}[2]{\begin{center}
    \includegraphics[#1]{#2}
\end{center}}
%===============================================================================

%=================================CODE LISTINGS=================================
\definecolor{codebackdrop}{gray}{0.9}
\definecolor{commentgreen}{rgb}{0,0.6,0}
\lstset{
    inputpath=code, 
    commentstyle=\color{commentgreen},
    keywordstyle=\color{blue}, 
    backgroundcolor=\color{codebackdrop}, 
    basicstyle=\footnotesize,
    frame=single,
    numbers=left,
    stepnumber=1,
    showstringspaces=false,
    breaklines=true,
    postbreak=\mbox{\textcolor{red}{$\hookrightarrow$}\space}
}

% Create a code listing for a single line
% \codeline{language}{line}{file}
\newcommand{\codeline}[3]{\lstinputlisting[language=#1, firstline = #2, lastline = #2]{#3}}

% Create a code listing for a given language & file
% \codelist{language}{file}
\newcommand{\codelist}[2]{\lstinputlisting[language=#1]{#2}}
%===============================================================================

%================================TEXT STRUCTURES================================
% Marka a word as bold
% \keyword{important word}
\newcommand{\keyword}[1]{\textbf{#1}}

% Creates a section in italics
% \question{question in italics}
\newcommand{\question}[1]{\textit{#1} \\ }

% Creates a box with title for side notes.
% \sidenote{title}{contents}
\newcommand{\sidenote}[2]{\begin{tcolorbox}[title=#1]#2\end{tcolorbox}}

% Creates an item in an itemize or enumerate, with a paragraph after
% \begin{itemize}
%     \bullpara{title}{contents}
% \end{itemize}
\newcommand{\bullpara}[2]{\item \textbf{#1} \ #2}

% Creates a compact list (very small gaps between items)
% \compitem{
%     \item item 1
%     \item item 2
%     \item ...
% }
\newcommand{\compitem}[1]{\begin{itemize}\setlength\itemsep{-0.5em}#1\end{itemize}}

% Creates a link to the lecture for use at the start of the notes document
\newcommand{\lectlink}[1]{\sidenote{Lecture Recording}{
    Lecture recording is available \href{#1}{here}
}}
%===============================================================================


%=============================DISPLAYING CODE STEPS=============================
\newenvironment{steps}
{\begin{tabular}{l}}
		{\end{tabular}}

\newcommand{\step}[2]{$\leadsto$ \{ #2 \} \\ $#1$ \\}
\newcommand{\start}[1]{$#1$ \\}

%===============================================================================


\begin{document}
\maketitle
\lectlink{https://imperial.cloud.panopto.eu/Panopto/Pages/Viewer.aspx?id=678e8f17-9eb3-4469-96d0-adde00d597c2}

\section*{Equality}
\codelist{Haskell}{equality.hs}
Eq is the typeclass for equality, any instance of this class should ensure the equality satisfied the laws:
\begin{center}
	\begin{tabular}{r l}
		reflexivity  & $x == x$                                 \\
		transitivity & $x == y \land y == z \Rightarrow x == z$ \\
		symmetry     & $x == y \Rightarrow y == x$              \\
	\end{tabular}
\end{center}
We also expect the idescernability of identicals (Leibniz Law):
\[x == y \Rightarrow f \ x == f \ y\]
For a set-like interface we have a member function:
\codelist{Haskell}{member.hs}
If we assume only that Eq holds, the complexity is $O(n)$ as we must potentially check all members of the set. To get around this, we use ordering.

\section*{Orderings}
The Ord typeclass allows us to check for inequalities:
\codelist{Haskell}{order.hs}
We must try to ensure certain properties hold, for example for to have a partial order we require:
\begin{center}
	\begin{tabular}{r l}
		reflexivity  & $x \leq x$                                     \\
		transitivity & $x \leq y \land y \leq z \Rightarrow x \leq z$ \\
		antisymmetry & $x \leq y \land y \leq x \Rightarrow x == y$   \\
	\end{tabular}
\end{center}
There are also total orders (all elements in the set are ordered compared to all others), for which we add the constraint:
\begin{center}
	\begin{tabular}{r l}
		connexity & $x \leq y \lor y \leq x$ \\
	\end{tabular}
\end{center}

\section*{Ordered Sets}
\codelist{Haskell}{ordset.hs}
We can implement this class for Trees:
\codelist{Haskell}{search tree.hs}
However the worst case here is still $O(n)$ as we do not balance the tree as more members are inserted. If the members are added in order, the tree devolves to a linked list.
\\
\\ We need a way to create a tree that self balances.

\section*{Binary Search Trees (AVL Trees)}
\codelist{Haskell}{avl tree.hs}
When inserting into the tree we must keep the tree balanced such that no subtree's left is more than one higher than its' right.
\codelist{Haskell}{avl tree insert.hs}
We must rebalance the tree after insertion, this must consider the following cases:
\centerimage{width=\textwidth}{htree balanced.png}
When the tree's balanced invariant has been broken, we must follow these cases:
\centerimage{width=\textwidth}{htree rotate right.png}
\codelist{Haskell}{avl rotate right.hs}
When the right subtree's right subtree is higher:
\centerimage{width=\textwidth}{htree rotate left.png}
\codelist{Haskell}{avl rotate left.hs}
We can use rotate left and rotate right to for the two cases where the right subtree is 2 or more higher than the left.
\codelist{Haskell}{avl balance.hs}
\end{document}
