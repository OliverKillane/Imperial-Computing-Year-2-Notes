\documentclass{report}
    \title{50001 - Algorithm Analysis and Design - Lecture 8}
    \author{Oliver Killane}
    \date{17/11/21}

%===========================COMMON FORMAT & COMMANDS===========================
% This file contains commands and format to be used by every module, and is 
% included in all files.
%===============================================================================

%====================================IMPORTS====================================
\usepackage[a4paper, total={6in, 8in}]{geometry}
\usepackage{graphicx, amssymb, amsfonts, amsmath, xcolor, listings, tcolorbox, multirow, hyperref}
%===============================================================================

%====================================IMAGES=====================================
\graphicspath{{image/}}

% \centerimage{options}{image}
\newcommand{\centerimage}[2]{\begin{center}
    \includegraphics[#1]{#2}
\end{center}}
%===============================================================================

%=================================CODE LISTINGS=================================
\definecolor{codebackdrop}{gray}{0.9}
\definecolor{commentgreen}{rgb}{0,0.6,0}
\lstset{
    inputpath=code, 
    commentstyle=\color{commentgreen},
    keywordstyle=\color{blue}, 
    backgroundcolor=\color{codebackdrop}, 
    basicstyle=\footnotesize,
    frame=single,
    numbers=left,
    stepnumber=1,
    showstringspaces=false,
    breaklines=true,
    postbreak=\mbox{\textcolor{red}{$\hookrightarrow$}\space}
}

% Create a code listing for a single line
% \codeline{language}{line}{file}
\newcommand{\codeline}[3]{\lstinputlisting[language=#1, firstline = #2, lastline = #2]{#3}}

% Create a code listing for a given language & file
% \codelist{language}{file}
\newcommand{\codelist}[2]{\lstinputlisting[language=#1]{#2}}
%===============================================================================

%================================TEXT STRUCTURES================================
% Marka a word as bold
% \keyword{important word}
\newcommand{\keyword}[1]{\textbf{#1}}

% Creates a section in italics
% \question{question in italics}
\newcommand{\question}[1]{\textit{#1} \\ }

% Creates a box with title for side notes.
% \sidenote{title}{contents}
\newcommand{\sidenote}[2]{\begin{tcolorbox}[title=#1]#2\end{tcolorbox}}

% Creates an item in an itemize or enumerate, with a paragraph after
% \begin{itemize}
%     \bullpara{title}{contents}
% \end{itemize}
\newcommand{\bullpara}[2]{\item \textbf{#1} \ #2}

% Creates a compact list (very small gaps between items)
% \compitem{
%     \item item 1
%     \item item 2
%     \item ...
% }
\newcommand{\compitem}[1]{\begin{itemize}\setlength\itemsep{-0.5em}#1\end{itemize}}

% Creates a link to the lecture for use at the start of the notes document
\newcommand{\lectlink}[1]{\sidenote{Lecture Recording}{
    Lecture recording is available \href{#1}{here}
}}
%===============================================================================


%=============================DISPLAYING CODE STEPS=============================
\newenvironment{steps}
{\begin{tabular}{l}}
		{\end{tabular}}

\newcommand{\step}[2]{$\leadsto$ \{ #2 \} \\ $#1$ \\}
\newcommand{\start}[1]{$#1$ \\}

%===============================================================================


\begin{document}
\maketitle
\lectlink{https://imperial.cloud.panopto.eu/Panopto/Pages/Viewer.aspx?id=77a4fe23-79fb-4468-8f5c-add300eedc4d}

\section*{Amortized Analysis}
So far we have studied complexity of a single, isolated run of an algorithm. \keyword{Amortizsed Analysis} is about understanding cost in a wider context (e.g averaged over many calls to a routine).

\section*{Dequeues}
\sidenote{Dequeue}{
	A Dequeue is a double ended queue. An abstract datatype that generalises a queue. Elements can be added or removed from either end.
	\\
	\\ Common associated funtions are:
	\begin{center}
		\begin{tabular}{l l}
			\fun{snoc}  & Insert element at the back of the queue.  \\
			\fun{cons}  & Insert element at the front of the queue. \\
			\fun{eject} & Remove last element.                      \\
			\fun{pop}   & remove fist element.                      \\
			\fun{peek}  & Examine but do not remove first element.  \\
		\end{tabular}
	\end{center}
	Dequeues are also called head-tail linked lists or symmetric lists.
}
we use a dequeue when we want to reduce the time taken to perform certain operations.
\subsubsection*{List Operation Complexity}
\centerimage{width=0.4\textwidth}{list anatomy.png}
\begin{minipage}[t]{0.4\textwidth}
	\begin{center}
		\begin{tabular}{l l}
			\fun{cons} & O(1) \\
			\fun{head} & O(1) \\
			\fun{tail} & O(1) \\
		\end{tabular}
	\end{center}
\end{minipage}
\hfill
\begin{minipage}[t]{0.4\textwidth}
	\begin{center}
		\begin{tabular}{l l}
			\fun{snoc} & O(1) \\
			\fun{last} & O(1) \\
			\fun{init} & O(1) \\
		\end{tabular}
	\end{center}
\end{minipage}
\subsubsection*{Dequeue Structure}
To achieve $O(1)$ complexity in the \fun{snoc}, \fun{init} and \fun{last} we use two lists.
\centerimage{width=0.7\textwidth}{dequeue structure.png}
One list starts contains the start of the list, and the other the end (reversed).
\\
\\ We keep two invariants for $Dequeue \ us \ sv$:
\[null \ us \Rightarrow null \ sv \lor single \ sv\]
\[null \ sv \Rightarrow null \ us \lor single \ us\]
In other words, if one list is empty, the other can contain at most 1 element.
\\
\\ An example implementation in haskell is below:
\codelist{Haskell}{dequeue declaration.hs}

When considering the cost of \fun{tail} and \fun{init} we must consider that there are two possibilities:
\begin{center}
	\begin{tabular}{l l p{7cm}}
		High Cost & $\begin{matrix}
				init (Dequeue \ us \ [s]) \\
				tail (Dequeue \ [u] \ sv) \\
			\end{matrix}$
		          & This operation is $O(n)$ complexity due to the \fun{spitAt} and \fun{reverse} operation done on half of a list.                                                                                       \\
		Low Cost  & $\begin{matrix}
				init (Dequeue \ us \ (s:sv)) \\
				tail (Dequeue \ (u:us) \ sv) \\
			\end{matrix}$                                                                                     & Low cost $O(1)$ operation as it requires only a pattern match on the first element. \\
	\end{tabular}
\end{center}
As both of these operations rebalance the \keyword{Dequeue} to to be balanced (half the queue on each list), we these operations can have an amortized cost of $O(1)$.
\\
\\ We know this as the average cost is of order $O(1)$. The $O(n)$ cost is incurred every $n/2$ calls to \fun{tail}/\fun{init}.
\centerimage{width=0.6\textwidth}{dequeue tail.png}
\end{document}
