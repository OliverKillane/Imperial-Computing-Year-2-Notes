\documentclass{report}
    \title{50001 - Algorithm Analysis and Design - Lecture 10}
    \author{Oliver Killane}
    \date{18/11/21}

%===========================COMMON FORMAT & COMMANDS===========================
% This file contains commands and format to be used by every module, and is 
% included in all files.
%===============================================================================

%====================================IMPORTS====================================
\usepackage[a4paper, total={6in, 8in}]{geometry}
\usepackage{graphicx, amssymb, amsfonts, amsmath, xcolor, listings, tcolorbox, multirow, hyperref}
%===============================================================================

%====================================IMAGES=====================================
\graphicspath{{image/}}

% \centerimage{options}{image}
\newcommand{\centerimage}[2]{\begin{center}
    \includegraphics[#1]{#2}
\end{center}}
%===============================================================================

%=================================CODE LISTINGS=================================
\definecolor{codebackdrop}{gray}{0.9}
\definecolor{commentgreen}{rgb}{0,0.6,0}
\lstset{
    inputpath=code, 
    commentstyle=\color{commentgreen},
    keywordstyle=\color{blue}, 
    backgroundcolor=\color{codebackdrop}, 
    basicstyle=\footnotesize,
    frame=single,
    numbers=left,
    stepnumber=1,
    showstringspaces=false,
    breaklines=true,
    postbreak=\mbox{\textcolor{red}{$\hookrightarrow$}\space}
}

% Create a code listing for a single line
% \codeline{language}{line}{file}
\newcommand{\codeline}[3]{\lstinputlisting[language=#1, firstline = #2, lastline = #2]{#3}}

% Create a code listing for a given language & file
% \codelist{language}{file}
\newcommand{\codelist}[2]{\lstinputlisting[language=#1]{#2}}
%===============================================================================

%================================TEXT STRUCTURES================================
% Marka a word as bold
% \keyword{important word}
\newcommand{\keyword}[1]{\textbf{#1}}

% Creates a section in italics
% \question{question in italics}
\newcommand{\question}[1]{\textit{#1} \\ }

% Creates a box with title for side notes.
% \sidenote{title}{contents}
\newcommand{\sidenote}[2]{\begin{tcolorbox}[title=#1]#2\end{tcolorbox}}

% Creates an item in an itemize or enumerate, with a paragraph after
% \begin{itemize}
%     \bullpara{title}{contents}
% \end{itemize}
\newcommand{\bullpara}[2]{\item \textbf{#1} \ #2}

% Creates a compact list (very small gaps between items)
% \compitem{
%     \item item 1
%     \item item 2
%     \item ...
% }
\newcommand{\compitem}[1]{\begin{itemize}\setlength\itemsep{-0.5em}#1\end{itemize}}

% Creates a link to the lecture for use at the start of the notes document
\newcommand{\lectlink}[1]{\sidenote{Lecture Recording}{
    Lecture recording is available \href{#1}{here}
}}
%===============================================================================


%=============================DISPLAYING CODE STEPS=============================
\newenvironment{steps}
{\begin{tabular}{l}}
		{\end{tabular}}

\newcommand{\step}[2]{$\leadsto$ \{ #2 \} \\ $#1$ \\}
\newcommand{\start}[1]{$#1$ \\}

%===============================================================================


\begin{document}
    \maketitle
    \lectlink{https://imperial.cloud.panopto.eu/Panopto/Pages/Viewer.aspx?id=62e2dbd5-4fba-4758-b2a6-addb016ec088}

    \section*{List lookup}
        \codelist{Haskell}{list lookup.hs}
        As you ca  see \fun{!!} costs $O(n)$ as it may traverse the entire list.
        \\
        \\ If we want this access to be faster, we can use trees:
        \codelist{Haskell}{list tree.hs}
        This costs $O(\log n)$ as each recursive call acts on half of the remaining list.
        \\
        \\ However we hav difficulty with insertion:
        \begin{center}
            \begin{tabular}{l l l}
                Insert Quickly & $n : t = node (Leaf n) t$ & Effectively becomes a linked list, $\log n$ search time ruined. \\
                Insert Slowly & Rebalance tree (e.g \keyword{AVL} tree) & Complex and no longer $O(1)$ insert. \\
            \end{tabular}
        \end{center}
    
    \section*{Random Access Lists}
        A list containing elements that are either nothing, or a perfect tree with size the same as $2^{index}$.
        \centerimage{width=\textwidth}{RAList.png}
        The empty tree can be represented by a $Tip$ value (from the notes), or using type $Maybe(Tree)$ (from the lecture) where $Tree \ a = Leaf \ x \ | \ Node \ n \ l \ r$.
        \\
        \\ When we add to a tree, we add to the first element of the \struct{RAList}, if the invariant is breached (no longer perfect tree of size $2^0$) it can be combined with the next list over (if empty, place, else combine and repeat).
        \\
        \\ This way while the worst case insert is $O(n)$, our amortized complexity is $O(1)$ much as with increment.
        \codelist{Haskell}{RAList.hs}
\end{document}
