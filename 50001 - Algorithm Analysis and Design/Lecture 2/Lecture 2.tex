\documentclass{report}
    \title{50001 - Algorithm Analysis and Design - Lecture 2}
    \author{Oliver Killane}
    \date{12/11/21}

%===========================COMMON FORMAT & COMMANDS===========================
% This file contains commands and format to be used by every module, and is 
% included in all files.
%===============================================================================

%====================================IMPORTS====================================
\usepackage[a4paper, total={6in, 8in}]{geometry}
\usepackage{graphicx, amssymb, amsfonts, amsmath, xcolor, listings, tcolorbox, multirow, hyperref}
%===============================================================================

%====================================IMAGES=====================================
\graphicspath{{image/}}

% \centerimage{options}{image}
\newcommand{\centerimage}[2]{\begin{center}
    \includegraphics[#1]{#2}
\end{center}}
%===============================================================================

%=================================CODE LISTINGS=================================
\definecolor{codebackdrop}{gray}{0.9}
\definecolor{commentgreen}{rgb}{0,0.6,0}
\lstset{
    inputpath=code, 
    commentstyle=\color{commentgreen},
    keywordstyle=\color{blue}, 
    backgroundcolor=\color{codebackdrop}, 
    basicstyle=\footnotesize,
    frame=single,
    numbers=left,
    stepnumber=1,
    showstringspaces=false,
    breaklines=true,
    postbreak=\mbox{\textcolor{red}{$\hookrightarrow$}\space}
}

% Create a code listing for a single line
% \codeline{language}{line}{file}
\newcommand{\codeline}[3]{\lstinputlisting[language=#1, firstline = #2, lastline = #2]{#3}}

% Create a code listing for a given language & file
% \codelist{language}{file}
\newcommand{\codelist}[2]{\lstinputlisting[language=#1]{#2}}
%===============================================================================

%================================TEXT STRUCTURES================================
% Marka a word as bold
% \keyword{important word}
\newcommand{\keyword}[1]{\textbf{#1}}

% Creates a section in italics
% \question{question in italics}
\newcommand{\question}[1]{\textit{#1} \\ }

% Creates a box with title for side notes.
% \sidenote{title}{contents}
\newcommand{\sidenote}[2]{\begin{tcolorbox}[title=#1]#2\end{tcolorbox}}

% Creates an item in an itemize or enumerate, with a paragraph after
% \begin{itemize}
%     \bullpara{title}{contents}
% \end{itemize}
\newcommand{\bullpara}[2]{\item \textbf{#1} \ #2}

% Creates a compact list (very small gaps between items)
% \compitem{
%     \item item 1
%     \item item 2
%     \item ...
% }
\newcommand{\compitem}[1]{\begin{itemize}\setlength\itemsep{-0.5em}#1\end{itemize}}

% Creates a link to the lecture for use at the start of the notes document
\newcommand{\lectlink}[1]{\sidenote{Lecture Recording}{
    Lecture recording is available \href{#1}{here}
}}
%===============================================================================


%=============================DISPLAYING CODE STEPS=============================
\newenvironment{steps}
{\begin{tabular}{l}}
		{\end{tabular}}

\newcommand{\step}[2]{$\leadsto$ \{ #2 \} \\ $#1$ \\}
\newcommand{\start}[1]{$#1$ \\}

%===============================================================================


\newcommand{\cond}[3]{\text{if } #1 \text{ then } #2 \text{ else } #3}

\begin{document}
    \maketitle
    \lectlink{https://imperial.cloud.panopto.eu/Panopto/Pages/Viewer.aspx?id=c8abc020-d7c8-44ad-b1df-adc100991181}

    \section*{Evaluation \& Cost Models}
        \codelist{Haskell}{minimum.hs}
        When analysing the cost of \fun{minimum} we must consider how the function is evaluated.
        \\
        \\ For example we could shortcut the once \fun{isort} has determined the first element (the minimum) of the list.
        \\
        \sidenote{Cost Model}{
            A model to determine the time taken to execute a program. 
            \\
            \\ The model assigns cost to different operations (e.g comparisons, calls, memory reads/writes)
            \\
            \\ A very generalised cost model assigns cost based on the number of \keyword{reductions} required to evaluate a program.
        }
    
    \section*{Small While Language}
        We can define a small language of expressions as follows:
        \[e ::= x \ | \ k \ | \ f \ e_1 \dots e_n \ | \ \cond{e}{e_1}{e_2}\]
        where $k$ means constant and $x$ is the variable form.
        \\ Infix functions such as $+, -, \times$ are written normally, and are expressions also as they can be used in the form $(+) \ e_1 \ e_2$.
        \\
        \\ There are also several primitve constants: $True, False, 0, 1, 2, \dots$
        \\ List constants and operations are also primitive: $[], (:), null, head, tail$
    
    \section*{Evaluation Order}
        \begin{itemize}
            \bullpara{Applicative Order}{ Strict evaluation 
                \\The leftmost, innermost reducible expression is evaluated first.
                \\ e.g for $fst (3 \times 2, 1 + 2)$
                \\ \begin{steps}
                    \start{fst (3 \times 2, 1 + 2)}
                    \step{ fst (6, 1 + 2)}{Definition of $\times$}
                    \step{fst (6, 3)}{ Definition of $+$}
                    \step{6}{Definition of $fst$}
                \end{steps}
            }
            \bullpara{Normal Order}{ Lazy evaluation 
                \\ The leftmost outer reducible is evaluated first. Efectively evaluating the function beforte its arguments.
                \\ e.g for $fst (0, 1 + 2)$
                \\ \begin{steps}
                    \start{fst (3 \times 2, 1 + 2)}
                    \step{3 \times 2}{ Definition of $fst$}
                    \step{6}{ Definition of $\times$}
                \end{steps}
            }
        \end{itemize}
        For a given program, if \keyword{applicative} and \keyword{normal} terminate, then they produce the same value in normal form.
        \\
        \\ However there are some programs where \keyword{normal} evaluation terminates, but \keyword{applicative} will not.
        \begin{minipage}[t]{0.4\textwidth}
            \centerline{\textbf{Applicative}}
            \begin{steps}
                \start{fst (0, \text{crazy nonesense})}
                \step{CRASH!}{ By lack of definition for crazy nonesense}
            \end{steps}
        \end{minipage}
        \hfill
        \begin{minipage}[t]{0.4\textwidth}
            \centerline{\textbf{Normal}}
            \begin{steps}
                \start{fst (0, \text{crazy nonesense})}
                \step{0}{ Definition of $fst$}
            \end{steps}
        \end{minipage}
        The program may by syntactically correct, but have an error such as zero-division which will not be evaluated and hence not result in improper termination under \keyword{normal} order.
        \begin{large} \[\text{Applicative Terminates} \Rightarrow \text{Normal Terminates}\] \end{large}
    
    \section*{Cost Model for Small While}
        We can evaluate a cost model for the small while language by creating a function $T$ to assign cost to expressions.
        \\\begin{tabular}{p{0.25\textwidth} p{0.35\textwidth} p{0.4\textwidth}}
            \textbf{Type} & \textbf{Function} & \textbf{Explanation} \\
            \hline
            non-primitive function & $\begin{matrix}
                f \ a_1 \ \dots \ a_n = e \\
                T(f) \ a_1 \ \dots \ a_n = T(e) + 1 \\
            \end{matrix}$ & Given we have already computed all argument, the cost of the function is the cost of the expression it produces, and a single call. \\
            \hline
            primitive function & $T(f) \ x \dots \ x_n = 0$ & Primitive functions are assumed to be free. \\
            \hline
            Variable & $T(x) = 0$ & accessing variables is free. \\
            \hline
            Application & $T(f \ e_1 \ \dots \ e_n) = T(f) \ e_1 \ \dots \ e_n + T(e_1) + \dots + T(e_n)$ & When applying a function we must consider both its cost, and the cost of all argument expressions. \\
            \hline
            Conditional & $T(\cond{p}{e_1}{e_2}) = T(p) + \cond{p}{T(e_1)}{T(e_2)}$ & Cost of condition and of the resulting expression. \\
        \end{tabular}
    
    \section*{Cost Model Example}
        Given the function:
        \[mul \ m \ n = \cond{m=0}{0}{n + mul(m-1)n}\]
        Evaluate $T(mul \ 3 100)$
        \\
        \\ \unfinished
        \begin{proof}
            \proofstep{1}{$mul \ 3 \ 100$}{}
            \proofstep{2}{$T(\cond{3=0}{0}{100 + mul \ (3-1) \ 100}) + 1$}{By Rule for non-primitive functions}
            \proofstep{3}{$T(3=0) + T(100 + mul \ (3-1) \ 100) + 1$}{By rule for conditionals}
            \proofstep{3}{$0 + T(100 + mul \ (3-1) \ 100) + 1$}{By primitive functions}
            \proofstep{3}{$T(+) (100 \ mul \ (3-1) \ 100) + T(100) + T(mul \ (3-1) \ 100) + 1$}{By application rule}
            \proofstep{3}{$0 + T(100) + T(mul \ (3-1) \ 100) + 1$}{By rule for primitive functions}
            \proofstep{3}{$T(mul \ (3-1) \ 100) + 1$}{By rule for constants}
            \proofstep{3}{$T(mul) \ (3-1) \ 100 + T(3 - 1) + T(100) + 1$}{By application rule}
            \proofstep{3}{$T(mul) \ (3-1) \ 100 + T(-) 3 \ 1 + T(100) + 1$}{By application rule}
            \proofstep{3}{$T(mul) \ (2) \ 100 + 1$}{By application rule}
        \end{proof}
\end{document}
