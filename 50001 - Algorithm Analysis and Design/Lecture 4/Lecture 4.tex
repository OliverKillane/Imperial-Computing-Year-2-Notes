\documentclass{report}
    \title{50001 - Algorithm Analysis and Design - Lecture 4}
    \author{Oliver Killane}
    \date{12/11/21}

%===========================COMMON FORMAT & COMMANDS===========================
% This file contains commands and format to be used by every module, and is 
% included in all files.
%===============================================================================

%====================================IMPORTS====================================
\usepackage[a4paper, total={6in, 8in}]{geometry}
\usepackage{graphicx, amssymb, amsfonts, amsmath, xcolor, listings, tcolorbox, multirow, hyperref}
%===============================================================================

%====================================IMAGES=====================================
\graphicspath{{image/}}

% \centerimage{options}{image}
\newcommand{\centerimage}[2]{\begin{center}
    \includegraphics[#1]{#2}
\end{center}}
%===============================================================================

%=================================CODE LISTINGS=================================
\definecolor{codebackdrop}{gray}{0.9}
\definecolor{commentgreen}{rgb}{0,0.6,0}
\lstset{
    inputpath=code, 
    commentstyle=\color{commentgreen},
    keywordstyle=\color{blue}, 
    backgroundcolor=\color{codebackdrop}, 
    basicstyle=\footnotesize,
    frame=single,
    numbers=left,
    stepnumber=1,
    showstringspaces=false,
    breaklines=true,
    postbreak=\mbox{\textcolor{red}{$\hookrightarrow$}\space}
}

% Create a code listing for a single line
% \codeline{language}{line}{file}
\newcommand{\codeline}[3]{\lstinputlisting[language=#1, firstline = #2, lastline = #2]{#3}}

% Create a code listing for a given language & file
% \codelist{language}{file}
\newcommand{\codelist}[2]{\lstinputlisting[language=#1]{#2}}
%===============================================================================

%================================TEXT STRUCTURES================================
% Marka a word as bold
% \keyword{important word}
\newcommand{\keyword}[1]{\textbf{#1}}

% Creates a section in italics
% \question{question in italics}
\newcommand{\question}[1]{\textit{#1} \\ }

% Creates a box with title for side notes.
% \sidenote{title}{contents}
\newcommand{\sidenote}[2]{\begin{tcolorbox}[title=#1]#2\end{tcolorbox}}

% Creates an item in an itemize or enumerate, with a paragraph after
% \begin{itemize}
%     \bullpara{title}{contents}
% \end{itemize}
\newcommand{\bullpara}[2]{\item \textbf{#1} \ #2}

% Creates a compact list (very small gaps between items)
% \compitem{
%     \item item 1
%     \item item 2
%     \item ...
% }
\newcommand{\compitem}[1]{\begin{itemize}\setlength\itemsep{-0.5em}#1\end{itemize}}

% Creates a link to the lecture for use at the start of the notes document
\newcommand{\lectlink}[1]{\sidenote{Lecture Recording}{
    Lecture recording is available \href{#1}{here}
}}
%===============================================================================


%=============================DISPLAYING CODE STEPS=============================
\newenvironment{steps}
{\begin{tabular}{l}}
		{\end{tabular}}

\newcommand{\step}[2]{$\leadsto$ \{ #2 \} \\ $#1$ \\}
\newcommand{\start}[1]{$#1$ \\}

%===============================================================================


\begin{document}
    \maketitle
    \lectlink{https://imperial.cloud.panopto.eu/Panopto/Pages/Viewer.aspx?id=8ddef027-50cb-4256-9115-adc50139006e}

    \section*{Lists}
        \codelist{Haskell}{list.hs}
        \centerimage{width=0.6\textwidth}{list anatomy.png}
        Lists in \keyword{Haskell} are a persistent data structure, meaning that when operations are applied to lists the original list is maintained (not mutated).

        \subsubsection*{Append}
            We can append lists, by traversing over the first list, copying values (this ensures both argument lists are preserved).
            \codelist{Haskell}{append.hs}
            \centerimage{width=\textwidth}{list structure.png}
            As the entire first list must be traversed, the cost of $xs ++ ys$ is $T_{(++)}(n) \in O(n)$ where $n = length \ xs$
        
        \subsubsection*{Foldr}
            \codelist{Haskell}{foldr.hs}
            \centerimage{width=0.8\textwidth}{foldr.png}
            As you can see, $foldr \ (:) \ [] \equiv id$.
            \\ Foldr can also be expressed through bracketing
            \[foldr \ f \ k \ [x_1, x_2, \dots, x_n] \equiv f \ x_1 \ (f \ x_2 \ ( \dots (f \ x_n \ k) \dots))\]

            \sidenote{Associativity}{
                \keyword{Associativity} determines how operations are grouped in the absence of brackets.
                \[\begin{matrix}
                    a \spadesuit b \spadesuit c & \text{unbracketed statement} \\
                    ((a) \spadesuit b) \spadesuit c & \spadesuit \text{ is left associative} \\
                    a \spadesuit (b \spadesuit (c)) & \spadesuit \text{ is right associative} \\
                \end{matrix}\]
                If $\spadesuit$ is associative, then the right \& left associative versions are equivalent.
            }
            \fun{foldr} applies functions in a right-associative scheme.
        
        \subsubsection*{Foldl}
            \codelist{Haskell}{foldl.hs}
            \centerimage{width=0.8\textwidth}{foldl.png}
            As you can see $foldl \ (snoc) \ [] \equiv id$.
            \\ Foldl can be expressed through bracketing
            \[foldl \ f \ k \ [x_1, x_2, \dots, x_n] \equiv f \ (\dots (f \ (f \ k \ x_1) x_2) \dots x_n)\]

    \section*{Monoids}
        Consider the case when for some $\bigstar $ and $\epsilon$: $foldr \ \bigstar \ \epsilon \equiv foldl \ \bigstar \ \epsilon$.
        For this to be possible for $\bigstar \ :: \ a \to a \to a$ and $\epsilon \ :: \ a$.
        \begin{center}
            \begin{tabular}{r l}
                $\bigstar$ must be associative & $x \ \bigstar \ (y \ \bigstar \ z) \equiv (s \ \bigstar \ y) \ \bigstar z$ \\
                $\epsilon$ must have no effect & $\epsilon \ \bigstar \ n = n$ \\
            \end{tabular}
        \end{center}
        These properties for a \keyword{monoid} $(a, \bigstar, \epsilon)$.
        \\ Other example include:
        \[\begin{matrix}
            (lists, ++, []) & (\mathbb{N}, +, 0) & (\mathbb{N}, \times, 1) & (bool, \land, true) \\
            (bool, \lor, false) & (\mathbb{R}, max, \infty) & (\mathbb{R}, min, -\infty) & (Universal \ set, \cup, \emptyset) \\
        \end{matrix}\]
        We can also find monoids of functions:
        \[(a \to a, (.), id)\]
        as $(id \ . \ g) x \equiv id (g \ x)$ and $((f \ . \ g) \ . h) x = f(g(h \ x))$
    
    \section*{Concat}
        We can easily define concat recursively as:
        \codelist{Haskell}{concat.hs}
        We can also notice that $([[a]], (++), [])$ is a monoid, so we can use \fun{foldr} or \fun{foldl}
        \\\begin{minipage}[t]{0.45\textwidth}
            \codelist{Haskell}{concatr.hs}
        \end{minipage}
        \hfill
        \begin{minipage}[t]{0.45\textwidth}
            \codelist{Haskell}{concatl.hs}
        \end{minipage}
        as \fun{(++)} makes a copy of the first argument (to ensure persistent data), if we apply is in a left associative bracketing scheme we will have to make larger \& larger copies.
        \[( \dots (((( [ \ ] \text{++}_{0} \  xs_1) \text{++}_{m} \ xs_2) \text{++}_{2m} \ xs_3) \text{++}_{3m}  \ xs_4 \dots) \text{++}_{mn} \ xs_n\]
        Hence where $n = length \ xss$ and $m = length \ xs_1 = length \ xs_2 = \dots = length \ xs_n$.
        \[\begin{matrix}
            T_{concatl}(m,n) \in O(n^2m) \\
            T_{concatr}(m,n) \in O(nm) \\
        \end{matrix}\]
    
    \section*{DLists}
        Instead of storing a list, we store a composition of functions that build up a list.

        \[ \begin{matrix}
            xs_1 \text{++} xs_2 \text{++} xs_3 \text{++} \dots \text{++} xs_n \\
            \Downarrow \\
            f \ xs_1 \bullet f \ xs_2 \bullet f \ xs_3 \bullet \dots \bullet f \ xs_n \\
            \Downarrow \\
            (xs_1 \ \text{++}) \bullet (xs_2 \ \text{++}) \bullet (xs_3 \ \text{++}) \bullet \dots \bullet (xs_n \ \text{++}) \\
        \end{matrix}\]
        We can then apply this function on the empty list $[ \ ]$ to get the resulting list.
        \codelist{haskell}{DList.hs}
        We can form a \keyword{monoid} of ($DList$,++,$DList \ id$).
\end{document}
