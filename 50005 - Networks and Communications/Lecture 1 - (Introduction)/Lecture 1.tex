\documentclass{report}
    \title{50005 - Networks and Communications - Lecture 1}
    \author{Oliver Killane}
    \date{18/01/22}
\input{../50005 common.tex}
\begin{document}
    \maketitle
    \lectlink{https://imperial.cloud.panopto.eu/Panopto/Pages/Viewer.aspx?id=249f18dd-a0ec-45ff-a009-ae1d01349e0f}

    \section*{Introduction}
        \subsection*{Module Timetable}
            \begin{center}
                \begin{tabular}{c l}
                    \textbf{Week} & \textbf{Topic} \\
                    1 & Introduction \\
                    2 & Basic Concepts \\
                    3 & Application Layer \\
                    4 & Transport Layer \\
                    5 & Security \\
                    6 & Network Layer \\
                    7 & Practical Applications \\
                    8 & Data Link Layer \\
                    9 & Physical Layer \& Coding \& Network Simulation \\
                    10 & The Future \& Revision \\
                \end{tabular}
            \end{center}
            Coursework runs from Monday $7/2 \to$ Friday $25/2$.
            \\
            \\ Exam occurs in the summer term, covering principals, design tradeoffs (not a low-level/technical style).

        \lectlink{https://imperial.cloud.panopto.eu/Panopto/Pages/Viewer.aspx?id=151a8c0a-6a89-454d-a75e-ae1d013510aa}

        \subsection*{Lectures}
            \compitem{
                \item All lectures are pre-recorded and released a week before the Monday Q\&A session.
                \item Monday Q\&A session is recorded and run on teams.
            }
        
        \subsection*{How to not fail the module}
            \compitem{
                \item Attend Q\&A sessions.
                \item Complete the weekly worksheets.
                \item Read up on the links in the slides.
                \item Complete the coursework.
                \item Ask and help to answer questions on EdStem.
                \item Revise during term.
            }
        \subsection*{Bibliography}
            Books to aide with the course, you are not expected to read them completely however you may find them useful to address topics from the slides.
            \begin{itemize}
                \bullpara{"Computer Networks" by Andres S. Tanenbaum}{ 5th or 4th edition suffice.}
                \bullpara{"Computer Networking: A Top-Down Approach" James Kurose and Keith Ross}{ 7th or 6th edition suffice, E=Book available on Imperial College Library Website.}
            \end{itemize}
    
    \section*{What is Computer Networking}
        \termdef{Computer Networking}{
            The process of interconnecting computer systems via telecommunications methods to share data and resources.
        }
        \compitem{
            \item Networks are becoming eprvasive (everywhere, always on).
            \item Most mainstream softwarew systems are distributed (cloud computing).
            \item Peformance often depends on network usage (can be a bottleneck or on critical path)
        }
        \sidenote{First Internet Connection}{
            Arpanet (first part of the internet) was created on september 1st 1969 with a single node.
            \\
            \\ First message as "login", after "lo" was transmitted it crashed, but sent the resul after rebooting an hour later.
            \\
            \\ Greatly expanded afterwords, connecting several universities.
            \centerimage{width=\textwidth}{arpanet sept 1971.png}
        }
        
        \subsection*{Vocations}
            \begin{itemize}
                \bullpara{Network Engineer/Architect}{ Design, build and maintain networks.}
                \bullpara{Server Application Developer}{ Server Backend and communication for cloud applications.}
                \bullpara{Network Software Engineer}{ Networks + Software Engineering }
                \bullpara{Data Center / Cloud Platform Admin}{ Networks + Cloud Computing }
                \bullpara{Network Secuity Engineer}{ Networks + Computer Security }
            \end{itemize}

\end{document}
