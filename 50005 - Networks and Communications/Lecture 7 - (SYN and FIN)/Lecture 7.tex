\documentclass{report}
    \title{50005 - Networks and Communications - Lecture 7}
    \author{Oliver Killane}
    \date{18/04/22}
%===========================COMMON FORMAT & COMMANDS===========================
% This file contains commands and format to be used by every module, and is 
% included in all files.
%===============================================================================

%====================================IMPORTS====================================
\usepackage[a4paper, total={6in, 8in}]{geometry}
\usepackage{graphicx, amssymb, amsfonts, amsmath, xcolor, listings, tcolorbox, multirow, hyperref}
%===============================================================================

%====================================IMAGES=====================================
\graphicspath{{image/}}

% \centerimage{options}{image}
\newcommand{\centerimage}[2]{\begin{center}
    \includegraphics[#1]{#2}
\end{center}}
%===============================================================================

%=================================CODE LISTINGS=================================
\definecolor{codebackdrop}{gray}{0.9}
\definecolor{commentgreen}{rgb}{0,0.6,0}
\lstset{
    inputpath=code, 
    commentstyle=\color{commentgreen},
    keywordstyle=\color{blue}, 
    backgroundcolor=\color{codebackdrop}, 
    basicstyle=\footnotesize,
    frame=single,
    numbers=left,
    stepnumber=1,
    showstringspaces=false,
    breaklines=true,
    postbreak=\mbox{\textcolor{red}{$\hookrightarrow$}\space}
}

% Create a code listing for a single line
% \codeline{language}{line}{file}
\newcommand{\codeline}[3]{\lstinputlisting[language=#1, firstline = #2, lastline = #2]{#3}}

% Create a code listing for a given language & file
% \codelist{language}{file}
\newcommand{\codelist}[2]{\lstinputlisting[language=#1]{#2}}
%===============================================================================

%================================TEXT STRUCTURES================================
% Marka a word as bold
% \keyword{important word}
\newcommand{\keyword}[1]{\textbf{#1}}

% Creates a section in italics
% \question{question in italics}
\newcommand{\question}[1]{\textit{#1} \\ }

% Creates a box with title for side notes.
% \sidenote{title}{contents}
\newcommand{\sidenote}[2]{\begin{tcolorbox}[title=#1]#2\end{tcolorbox}}

% Creates an item in an itemize or enumerate, with a paragraph after
% \begin{itemize}
%     \bullpara{title}{contents}
% \end{itemize}
\newcommand{\bullpara}[2]{\item \textbf{#1} \ #2}

% Creates a compact list (very small gaps between items)
% \compitem{
%     \item item 1
%     \item item 2
%     \item ...
% }
\newcommand{\compitem}[1]{\begin{itemize}\setlength\itemsep{-0.5em}#1\end{itemize}}

% Creates a link to the lecture for use at the start of the notes document
\newcommand{\lectlink}[1]{\sidenote{Lecture Recording}{
    Lecture recording is available \href{#1}{here}
}}
%===============================================================================

\begin{document}
\maketitle

\section*{WireShark}
\lectlink{https://imperial.cloud.panopto.eu/Panopto/Pages/Viewer.aspx?id=bd933f5f-688c-460a-af86-ae3d0100e28c}
\termdef{Credential Resuse/Stuffing}{
    Using previously leaked/found eail-passoword combinations on other services. Useful when users reuse passwords for mutliple services.
}
\termdef{Network Monitoring / Packet Sniffing}{
    Listening on a network and reading packets where you (your \keyword{NIC}) are not the intended recipient.
}
\termdef{Code/SQL Injection}{
    Executing code on a system by passing it through normal data collection.
    \\
    \\ e.g if text entered in a website is directly substituted into a database query, by writing SQL ocde in the text entry, we can alter (or add another query) to the query generated.
}
\termdef{Session/Cookie Hijacking}{
    Using the cookie/authentication token form another user's session to get authenticated.
}
\termdef{Wardriving}{
    Identifying and compomising unsecured wireless networks.
    \\
    \\ e.g installing spyware on usecured home routers.
}
\termdef{Wireshark}{
    Wireshark is a network protocol analyser. It allows users to capture, analyse \& deconstruct packets to analyse traffic on a network. 
}
\subsection*{WireShark Modes}
\halfsplit{
    \centerline{\keyword{Promiscuous Mode}}
    \compitem{
        \item Works for Wired and wireless.
        \item \keyword{NIC} does not drop packets, retains all received packets.
        \item When wireless, only listens on the connected network.
        \item Some \keyword{NIC}s ignore this (considered \textit{impolite} and easily abused).
    }
}{
    \centerline{\keyword{Monitor Mode}}
    \compitem{
        \item Only works on wireless networks.
        \item \keyword{NIC} listens on all networks in range/that it can receive from.
        \item Wifi networks secured with authentication (e.g password) will appear scrambled (encryption).
        \item Most \keyword{NICs} do not support this, may require new drivers or a special \keyword{NIC}.
        \item \keyword{WinPcap} (windows) does not support though \keyword{AirPcap} and \keyword{Npcap} on linux do.
    }
}

\sidenote{Sniffing Ethics}{
    When monitoring a network, it needs to be a netywork you have permission to monitor (either wired or wireless)
}
\lectlink{https://imperial.cloud.panopto.eu/Panopto/Pages/Viewer.aspx?id=5481751f-1e54-487a-8f58-ae3d0101379a}
\subsection*{WireShark Packet Capture}
\begin{center}
    \begin{tabular}{l p{0.7\textwidth}}
        \textbf{Location} & \textbf{Can Capture} \\
        \hline
        \textbf{Hub} & Local traffic, Broadcast/Multicast, (Promiscuous Mode) Entire Network. \\
        \textbf{Switch} & Local Traffic, Broadcast/Multicast, (Promiscuous Mode) Network connected to the same switch port. \\
        \textbf{WLAN} & Local Traffic, Broadcast/Multicast, (Promiscuous Mode) Entire WLAN, (Monitor Mode) All wireless packets physically receivable/in range. \\
    \end{tabular}
\end{center}
We can provide wireshark with authentication to allow it to decrypt packets on for protected networks (e.g provide the RSA key for SSL, or password for WPA/WEP).

\subsection*{WireShark Display Filters}
Can hide or select packets based on contents, destination \& source address and more. And can build up complex filters.
\example{WireShark Capture Filter}{
    \codelist{Python}{wireshark filter.py}
    \href{https://wiki.wireshark.org/DisplayFilters\#examples}{More Examples} and \href{https://www.wireshark.org/docs/wsug_html_chunked/ChWorkBuildDisplayFilterSection.html}{filter building tutorial}.
}

\subsection*{NMAP}
\termdef{NMAP}{
    A network scanning tool which uses sends raw \keyword{IP} packets and monitors responses \& determine the services provided by the network and its hosts.
    \\
    \\ It can be used to detect vulnerable hosts on a network.
}
We can scan networks using the gui, or by using the command line utility:
\codelist{Bash}{namp examples.sh}

\end{document}
